\documentclass[a4paper,12pt]{article}
\usepackage[latin1]{inputenc}
\usepackage[spanish]{babel}
\usepackage{bm}
\usepackage{graphicx}
\usepackage{amsmath}
\usepackage{enumerate}
%\documentstyle[12pt,bezier]{articulo}
\setlength{\textheight}{235mm}
\setlength{\textwidth}{168mm}
\setlength{\oddsidemargin}{0pt}
\pagestyle{empty}

\begin{document}
\mbox{}\vspace*{-45mm}

{\centering
{\small\sc Escuela T�cnica Superior de Ingenieros de Caminos, Canales y
Puertos (Madrid)}\\*[4mm]
{\Large\bf M�todo de los Elementos Finitos}\\*[4mm]
PR�CTICA 6: Elasticidad lineal (modelos axisim�tricos)\\*[4mm]
}
%%%%%
\noindent
Obtener mediante un modelo de elementos finitos axisim�trico
el campo de desplazamientos y la distribuci�n de tensiones radiales y
circunferenciales en la pared de un cilindro sometido a presi�n interna $p$.
El radio
interior del cilindro es $a=0.5$ m., el radio exterior $b=1.0$ m. y
la presi�n $p=3 \cdot 10^8$ Pa. Considerar un material el�stico lineal con
m�dulo de elasticidad $E=2.1\cdot 10^{11}$ y coeficiente de Poisson $\nu=0.3$.
Comparar los resultados obtenidos con la soluci�n anal�tica:
\begin{align}
\sigma_{rr}&=-p \frac{{( b/r) }^2 -1}
{{( b/a)}^2 -1} \\
\sigma_{\theta \theta}&=
p \frac{{( b/r) }^2 +1}
{{( b/a)}^2 -1} \\
\sigma_{zz}&=\nu(\sigma_{rr}+\sigma_{\theta \theta})
=\nu p \frac{2}
{{( b/a)}^2 -1} \\
u_r&=\frac{(1+\nu)p}{E [{(b/a)}^2-1 ]} \left [ 
(1-2 \nu) r +\frac{b^2}{r} \right ]
\end{align}
\end{document}
