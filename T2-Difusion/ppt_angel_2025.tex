\documentclass{beamer}
\usepackage{ae}
\usepackage[utf8]{inputenc}
\usepackage[spanish]{babel}
\usepackage{amsmath,amsfonts}
\usepackage{bm}
\mode<presentation>
{
  \usetheme{PaloAlto}
  % or ...

%  \setbeamercovered{transparent}
  % or whatever (possibly just delete it)
  \usecolortheme{default}
}

\title[Modelos de difusión]{
MÉTODO DE LOS ELEMENTOS FINITOS \\
Modelos de difusión}
\author[F.G y A.Y]{Felipe Gabaldón Castillo \newline Ángel Yagüe Hernán (angel.yague@upm.es)}
\date{Madrid, 4 de noviembre de 2025}
\pgfdeclareimage[height=0.5cm]{logo-upm}{upm}
\logo{\pgfuseimage{logo-upm}}

\AtBeginSection[]
{
  \begin{frame}<beamer>
    \frametitle{Índice}
    \tableofcontents[currentsection]
  \end{frame}
}
%%%%%%%%%%%%%%%%%%%%%%%%%%%%%%%%%%%%%%%%%%%%%%%%%%%%%%%%%%%%%%%%%%%%%%%%%%%%%%
\newcommand{\norm}[1]{\lVert #1 \rVert}
\begin{document}
\begin{frame}
  \titlepage
\end{frame}
% --- DIAPOSITIVA DE CONTENIDO ---
\begin{frame}
  \frametitle{Contenido de la clase}  
  \tableofcontents
  
 % \begin{block}{Logística}
  %    \begin{itemize}
   %     \item \textbf{Tiempo Total:} 120 Minutos
    %    \item \textbf{Materiales:} Presentación (PDF 2), Capítulo del Libro (PDF 1), Pizarra.
    %\end{itemize}
  %\end{block}
\end{frame}
%%%%%%%%%%%%%%%%%%%%%%%%%%%%%%%%%%%%%%%%%%%%%%%%%%%%%%%%%%%%%%%%%%%%%%%%%%%%%%%%
% --- MÓDULO 1 ---%%%%%%%%%%%%%%%%%%%%%%%%%%%%%%%%%%%%%%%%%%%%%%%%%%%%%%%%%%%%%%%%%%%%%%%%%%%%%%
\section[M1: Introducción]{MÓDULO 1: Introducción y Motivación}

\begin{frame}
  \frametitle{La ecuación de Poisson}
  
  %\begin{pizarra}
      %Todo el problema de Poisson se puede resumir en estas cuatro líneas:
      \begin{enumerate}
          \item \textbf{Ecuación de Balance:} $div~q = f$
          
          \item \textbf{Ecuación Constitutiva:} $q = -k\nabla u$
          
          \item \textbf{Condición Esencial (Dirichlet):} $u = \overline{u}$ en $\partial_u\Omega$
          
          \item \textbf{Condición Natural (Neumann):} $q \cdot n = \overline{q}$ en $\partial_t\Omega$
      \end{enumerate}

      \vspace{5mm}
      siendo:
\vspace{-5mm}
\begin{align*}
u  &:\textrm{ función potencial (variable básica)} \\
\bm{q}&:\textrm{ vector flujo} \\
f&:\textrm{ fuente} \\
\bm{k}&:\textrm{ tensor constitutivo} \\
\end{align*}

El interés que tiene estudiar aquí la ecuación de Poisson es el
amplio número de modelos físicos que se rigen por esta ecuación, y que
la variable básica es un escalar
  %\end{pizarra}
\end{frame}




%%%%%%%%%%%%%%%%%%%%%%%%%%%%%%%%%%%%%%%%%%%%%%%%%%%%%%%%%%%%%%%%%%%%%%%%%%%%%%
\begin{frame}
\frametitle{Conducción de calor}
\begin{align*}
\operatorname{div} \bm{q}&=f  \\
\bm{q}&=-\bm{k} \bm{\nabla}_{x} u
\end{align*}
\begin{align*}
u  &:\textrm{\color{red}{Temperatura}} \\
\bm{q}&:\textrm{\color{red}{Flujo de calor por unidad de superficie}} \\
f&:\textrm{\color{red}{Calor generado por unidad de volumen}} \\
\bm{k}&:\textrm{\color{red}{Matriz de conductividades (Ley de Fourier)}}
\end{align*}
\end{frame}
%%%%%%%%%%%%%%%%%%%%%%%%%%%%%%%%%%%%%%%%%%%%%%%%%%%%%%%%%%%%%%%%%%%%%%%%%%%%%%
\begin{frame}
\frametitle{Conducción de calor}
\begin{center}
\includegraphics[width=0.65\textwidth]{building_envelope_ir.jpg}
\end{center}
\end{frame}
%%%%%%%%%%%%%%%%%%%%%%%%%%%%%%%%%%%%%%%%%%%%%%%%%%%%%%%%%%%%%%%%%%%%%%%%%%%%%%
\begin{frame}
\frametitle{Flujo en medios porosos}
\begin{align*}
\operatorname{div} \bm{q}&=f  \\
\bm{q}&=-\bm{k} \bm{\nabla}_{x} u
\end{align*}
\begin{align*}
u  &:\textrm{\color{blue}{Altura piezométrica }} (u=\frac{p}{\gamma}+z) \\
\bm{q}&:\textrm{\color{blue}{Velocidad}}\\
f&:\textrm{\color{blue}{Caudal suministrado por unidad de volumen}}\\
\bm{k}&:\textrm{\color{blue}{Matriz de permeabilidades (Ley de Darcy)}}
\end{align*}
\end{frame}
%%%%%%%%%%%%%%%%%%%%%%%%%%%%%%%%%%%%%%%%%%%%%%%%%%%%%%%%%%%%%%%%%%%%%%%%%%%%%%
\begin{frame}
\frametitle{Flujo en medios porosos}
\begin{center}
\includegraphics[width=\textwidth]{Deltares_IMOD.jpg}
\end{center}
\end{frame}
%%%%%%%%%%%%%%%%%%%%%%%%%%%%%%%%%%%%%%%%%%%%%%%%%%%%%%%%%%%%%%%%%%%%%%%%%%%%%%
\begin{frame}
\frametitle{Electrostática}
\begin{align*}
\operatorname{div} \bm{q}&=f  \\
\bm{q}&=-\bm{k} \bm{\nabla}_{x} u
\end{align*}
\begin{align*}
u  &:\textrm{\color{green}{Potencial eléctrico}} \\
\bm{q}&:\textrm{\color{green}{Intensidad de campo eléctrico}} \\
f&:\textrm{\color{green}{Carga eléctrica generada por unidad de volumen}} \\
\bm{k}&:\textrm{\color{green}{Matriz de permitividades (Ley de Gauss)}}
\end{align*}
\end{frame}
%%%%%%%%%%%%%%%%%%%%%%%%%%%%%%%%%%%%%%%%%%%%%%%%%%%%%%%%%%%%%%%%%%%%%%%%%%%%%%
\begin{frame}
\frametitle{Electrostática}
\begin{center}
\includegraphics[width=0.4\textwidth]{Model_Geometry_600_514.jpg}
\includegraphics[width=0.6\textwidth]{electric_fields_contour_600_480.jpg}
\end{center}
\end{frame}
%%%%%%%%%%%%%%%%%%%%%%%%%%%%%%%%%%%%%%%%%%%%%%%%%%%%%%%%%%%%%%%%%%%%%%%%%%%%%%
\begin{frame}
\frametitle{Torsión uniforme}
\begin{align*}
\operatorname{div} \bm{q}&=f  \\
\bm{q}&=-\bm{k} \bm{\nabla}_{x} u
\end{align*}
\begin{align*}
u  &:\textrm{\color{magenta}{ Función de tensiones (constante en }} \partial \Omega) \\
\bm{q}&:\textrm{\color{magenta}{ Tensiones tangenciales}} \\
f&: \textrm{ }2 G \theta, \textrm{\color{magenta}{ siendo }} 
\theta \textrm{\color{magenta}{ el giro por unidad de longitud y}} \\
 &\mbox{ }G \textrm{\color{magenta}{ el módulo de cortante}} \\
\sigma_{i3}&= \frac{\partial u}{\partial x_i}, 
\textrm{\color{magenta}{ sería la ecuación constitutiva}}
\end{align*}
\begin{itemize}
\item Otros modelos físicos: flujo irrotacional de fluidos ideales,
lubricación de cojinetes, magnetostática, presiones hidrodinámicas sobre
superficies en movimiento, etc.
\end{itemize}
\end{frame}
%%%%%%%%%%%%%%%%%%%%%%%%%%%%%%%%%%%%%%%%%%%%%%%%%%%%%%%%%%%%%%%%%%%%%%%%%%%%%%
\begin{frame}
\frametitle{Torsión uniforme}
\begin{center}
\includegraphics[height=0.95\textheight]{torsion_uni.pdf}
\end{center}
\end{frame}
%%%%%%%%%%%%%%%%%%%%%%%%%%%%%%%%%%%%%%%%%%%%%%%%%%%%%%%%%%%%%%%%%%%%%%%%%%%%%%










\begin{frame}
  \frametitle{Aplicaciones en Ingeniería Civil}
  
  \begin{block}{Flujo en Medios Porosos (Geotecnia)}
    \begin{itemize}
        \item \textbf{¡El más importante para ing. civil!}
        \item $u \rightarrow h$ (Altura piezométrica)
        \item $q \rightarrow v$ (Velocidad del flujo)
        \item $k \rightarrow k$ (Tensor de permeabilidad)
        \item Ley Constitutiva: \textbf{Ley de Darcy} $v = -k\nabla h$
        \item Si $f=0 \implies$ \textbf{Ecuación de Laplace} $\nabla^{2}h=0$
    \end{itemize}
  \end{block}
  
  \begin{block}{Transmisión de Calor (Eficiencia)}
    \begin{itemize}
        \item $u \rightarrow T$ (Temperatura)
        \item $q \rightarrow q$ (Flujo de calor)
        \item $k \rightarrow k$ (Conductividad térmica)
        \item Ley Constitutiva: \textbf{Ley de Fourier}
    \end{itemize}
  \end{block}
\end{frame}

\begin{frame}
  \frametitle{Aplicaciones en Ingeniería Civil (cont.)}
  \begin{block}{Torsión de Saint-Venant (Estructuras)}
    \begin{itemize}
        \item \textbf{¡Importante para estructuras!}
        \item $u \rightarrow \psi$ (Función de tensiones de Prandtl)
        \item La ''fuente'' $f$ es constante: $f = -2G\theta$
    \end{itemize}
  \end{block}
  
  \begin{block}{Otros}
      \begin{itemize}
          \item Electrostática
          \item ...
      \end{itemize}
  \end{block}
\end{frame}

%%%%%%%%%%%%%%%%%%%%%%%%%%%%%%%%%%%%%%%%%%%%%%%%%%%%%%%%%%%%%%%%%%%%%%%%%%%%%%%%
% --- MÓDULO 2 ---
%%%%%%%%%%%%%%%%%%%%%%%%%%%%%%%%%%%%%%%%%%%%%%%%%%%%%%%%%%%%%%%%%%%%%%%%%%%%%%%%

\section[M2: Formulación del problema de contorno]{MÓDULO 2: Formulación del problema de contorno}

\begin{frame}
  \frametitle{Formulación Fuerte}
  \begin{itemize}
      \item Es el sistema de PDEs que acabamos de ver.
      \item Se llama ''fuerte'' porque exige que la solución $u$ sea muy ''suave'' (continua y dos veces derivable, $C^2$).
  \end{itemize}
  
  \begin{alertblock}{Problema: ¿Qué pasa en la junta de dos materiales?}
    La derivada de la solución ''salta'' en la junta (ej. $k_1 \neq k_2$). La segunda derivada no existe.
    \vspace{1em}
    $\implies$ \textbf{Necesitamos una versión que permita soluciones ''menos suaves''.}
  \end{alertblock}
\end{frame}

\begin{frame}
\frametitle{Formulación Fuerte}

Sea $\overline{\Omega}=\Omega \cup \partial \Omega$ el dominio ocupado
por un medio conductivo ($\overline{\Omega} \subset \mathbb{R}^n$), cuyo
contorno $\partial \Omega$ admite la descomposición
$\partial \Omega=\partial_u \Omega \cup \partial_t \Omega$,
$\partial_u \Omega \cap \partial_t \Omega= \emptyset$. La formulación fuerte
del problema se establece en los siguientes términos:

Dados $f:\Omega \rightarrow \mathbb{R},\quad
\overline{u} :\partial_u \Omega \rightarrow \mathbb{R}, \quad
\overline{q} :\partial_t \Omega \rightarrow \mathbb{R}$, encontrar el
campo de temperaturas $u:\overline{\Omega}\rightarrow \mathbb{R}$ que
cumple:
\begin{align}
\operatorname{div} \bm{q}&=f \textrm{ en } \Omega\\
u&=\overline{u} \textrm{ en } \partial_u \Omega \\
\bm{q}\cdot \bm{n} &=\overline{q} \textrm{ en } \partial_t \Omega
\end{align}
con $\bm{q}=-\bm{k} \bm{\nabla} u$
\end{frame}
\begin{frame}
\frametitle{Formulación débil}

Dados $f:\Omega \rightarrow \mathbb{R}$ y las funciones
$\overline{u} :\partial_u \Omega \rightarrow \mathbb{R}, \quad
\overline{q} :\partial_t \Omega \rightarrow \mathbb{R}$, encontrar el
campo de temperaturas $u \in \delta \, \mid \, \forall \delta u \in
\nu$ que cumple:
\begin{equation}
\int_{\Omega} 
\left( \bm{q} \cdot \bm{\nabla} \delta u \, + f \delta u \right) \,
d \Omega - \int_{\partial_t \Omega} \overline{q} \delta u \, d \Gamma=0
\end{equation}
siendo:
\begin{align}
\delta&=\left\{
u \in H^1(\Omega,\mathbb{R}) \mid u(\bm{x})=\overline{u}
\; \quad \forall \; \bm{x} \in \partial_u \Omega \right\} \label{trial} \\
{\cal V}&=\left\{
\delta u \in H^1(\Omega,\mathbb{R}) \mid \delta u(\bm{x})=0
\; \quad \forall \; \bm{x} \in \partial_u \Omega \right\} \label{weight}
\end{align}
y $H^1(\Omega,\mathbb{R})$ el espacio de Sobolev de orden $1$ y grado $2$:
$$
H^1=\left\{ u:\Omega \rightarrow \mathbb{R} \quad \mid \quad
\int_{\Omega} \norm{u}_{2,1} \, d\Omega <\infty \right\}
$$
\end{frame}
%%%%%%%%%%%%%%%%%%%%%%%%%%%%%%%%%%%%%%%%%%%%%%%%%%%%%%%%%%%%%%%%%%%%%%%%%%%%%%

\begin{frame}
  \frametitle{Formulación Débil}
  
  \begin{itemize}
      \item Transformamos la PDE (puntual) en una Ecuación Integral (promediada).
      \item Esto ''baja el orden'' de las derivadas y relaja los requisitos de continuidad.
      \item Es la base del Método de los Elementos Finitos (MEF).
  \end{itemize}

  \begin{block}{Proceso: (Fuerte $\implies$ Débil)}
    Vamos a derivar la forma débil paso a paso.
  \end{block}
\end{frame}

% --- INICIO DE LA DEMOSTRACIÓN (VARIOS FRAMES) ---
\begin{frame}[allowframebreaks]
  \frametitle{Derivación (Fuerte $\implies$ Débil): Pasos 1 y 2}  

  %\begin{pizarra}
    \begin{enumerate}
      \item \textbf{Partir de la Ecuación de Balance:}
      \[ div~q - f = 0 \quad \text{en } \Omega \]
      
      \item \textbf{Multiplicar por una ''función de peso'' $\delta u$ e integrar:}
      \begin{equation*}
          \int_{\Omega} \delta u (div~q - f) d\Omega = 0
      \end{equation*}
      Esto debe ser cierto para \textit{toda} función $\delta u$ (que cumpla ciertos requisitos).
      
      \item \textbf{Separar los términos:}
      \[ \int_{\Omega} \delta u~div~q~d\Omega - \int_{\Omega} \delta u~f~d\Omega = 0 \]
    \end{enumerate}
%  \end{pizarra}
\end{frame}








\begin{frame}
  \frametitle{Derivación (Fuerte $\implies$ Débil): Pasos 3 y 4}
  
%\begin{pizarra}
    \begin{enumerate}
     % \setcounter{enumi}{3}{
      \item \textbf{Integración por Partes / T. Green:}
      Usamos la identidad:
      \[ \delta u~div(q) = div(\delta u~q) - \nabla\delta u \cdot q \]
      
      \item \textbf{Sustituir y aplicar Teorema de la Divergencia:}
      El Teorema de la Divergencia dice: $\int_{\Omega} div(\mathbf{v}) d\Omega = \int_{\partial\Omega} (\mathbf{v} \cdot n) d\Gamma$.
      Aplicándolo al término $div(\delta u~q)$:
      \begin{equation*}
          \int_{\partial\Omega} (\delta u~q) \cdot n~d\Gamma - \int_{\Omega} \nabla\delta u \cdot q~d\Omega - \int_{\Omega} \delta u~f~d\Omega = 0
      \end{equation*}
      \textbf{¡Hemos bajado el orden de las derivadas!}
    \end{enumerate}
 % \end{pizarra}
\end{frame}

\begin{frame}
  \frametitle{Derivación (Fuerte $\implies$ Débil): Pasos 5 y 6}
  
 % \begin{pizarra}
    \begin{enumerate}
  %    \setcounter{enumi}{5}{
      \item \textbf{Introducir Ecuación Constitutiva ($q = -k\nabla u$):}
      Reordenando y sustituyendo (con $q \cdot n = \overline{q}$):
      \begin{equation*}
          \int_{\Omega} (\nabla\delta u \cdot k\nabla u) d\Omega = \int_{\Omega} \delta u~f~d\Omega + \int_{\partial\Omega} \delta u~\overline{q}~d\Gamma
      \end{equation*}
      
      \item \textbf{Manejar Contornos (Esencial vs. Natural):}
      \begin{itemize}
          \item En $\partial_u\Omega$ (Dirichlet), exigimos $\delta u = 0$. La integral de contorno desaparece allí.
          \item En $\partial_t\Omega$ (Neumann), $\overline{q}$ es conocido y se mantiene.
      \end{itemize}
    \end{enumerate}
 % \end{pizarra}
\end{frame}

\begin{frame}
  \frametitle{Formulación Débil: ¡Resultado Final!}
  \begin{pizarra}
    \textbf{Formulación Débil Final:}
    
    Encontrar $u$ (tal que $u=\overline{u}$ en $\partial_u\Omega$) que cumpla para \textit{toda} $\delta u$ (tal que $\delta u=0$ en $\partial_u\Omega$):
    \begin{equation*}
        \int_{\Omega} (\nabla\delta u \cdot k\nabla u) d\Omega = \int_{\Omega} \delta u~f~d\Omega + \int_{\partial_t\Omega} \delta u~\overline{q}~d\Gamma
    \end{equation*}
  \end{pizarra}
\end{frame}

\begin{frame}
  \frametitle{Equivalencia (Débil $\implies$ Fuerte)}
  
 % \begin{pizarra}
    \textbf{Objetivo: Probar que (Débil $\implies$ Fuerte)}
    
    Asumimos que (W) es cierta. (W1) nos da (S2) gratis.
    Partimos de (W2) e integramos por partes \textit{al revés}:
    \begin{equation*}
        \int_{\Omega} q \cdot \nabla\delta u~d\Omega = -\int_{\Omega} (div~q) \delta u~d\Omega + \int_{\partial\Omega} (q \cdot n) \delta u~d\Gamma
    \end{equation*}
    Sustituyendo en (W2) y reordenando:
    \begin{equation*}
        \int_{\Omega} (f - div~q) \delta u~d\Omega + \int_{\partial_t\Omega} (q \cdot n - \overline{q}) \delta u~d\Gamma = 0
    \end{equation*}
    (Recordando que $\delta u = 0$ en $\partial_u\Omega$).
 % \end{pizarra}
\end{frame}

\begin{frame}
  \frametitle{Equivalencia (Débil $\implies$ Fuerte): Pasos Finales}
  
 % \begin{pizarra}
    \begin{enumerate}
      \item \textbf{Demostrar (S1) $div~q = f$:}
      \begin{itemize}
          \item Escogemos una $\delta u$ que sea CERO en $\partial_t\Omega$.
          \item El segundo término se anula $\implies \int_{\Omega} (f - div~q) \delta u~d\Omega = 0$.
          \item Por Lema Fundamental $\implies$ \textbf{$f - div~q = 0$}.
      \end{itemize}
      
      \item \textbf{Demostrar (S3) $q \cdot n = \overline{q}$:}
      \begin{itemize}
          \item Ahora que sabemos que $f - div~q = 0$, el primer término es cero.
          \item Nos queda: $\int_{\partial_t\Omega} (q \cdot n - \overline{q}) \delta u~d\Gamma = 0$.
          \item Esto vale para $\delta u$ que \textit{no} son cero en $\partial_t\Omega$.
          \item Por Lema Fundamental $\implies$ \textbf{$q \cdot n - \overline{q} = 0$}.
      \end{itemize}
    \end{enumerate}
    \textbf{Conclusión: Las dos formulaciones son 100\% equivalentes.}
 % \end{pizarra}
\end{frame}
% --- FIN DE LA DEMOSTRACIÓN ---
%%%%%%%%%%%%%%%%%%%%%%%%%%%%%%%%%%%%%%%%%%%%%%%%%%%%%%%%%%%%%%%%%%%%%%%%%%%%%%%%
% --- MÓDULO 3 ---
%%%%%%%%%%%%%%%%%%%%%%%%%%%%%%%%%%%%%%%%%%%%%%%%%%%%%%%%%%%%%%%%%%%%%%%%%%%%%%%%
\section[M3: Formulación de Galerkin]{MÓDULO 3: Discretización. Formulación de Galerkin}

\begin{frame}
\frametitle{Formulación de Galerkin}
Sean los conjuntos $\nu^h$ y $\delta^h$ aproximaciones de
dimensión finita de $\nu$ y $\delta$ respectivamente. Partiendo de
la descomposición (método de Bubnov-Galerkin):
\begin{equation}
u^h=v^h+\overline{u}^h
\end{equation}
donde $v^h \in \nu^h$ y $\overline{u}^h=\overline{u}$ en $\partial_u \Omega$
(``aproximadamente''), la formulación de Galerkin se expresa en los
siguientes términos:

Dados $f:\Omega \rightarrow \mathbb{R}$ y las funciones
$\overline{u} :\partial_u \Omega \rightarrow \mathbb{R}, \quad
\overline{q} :\partial_t \Omega \rightarrow \mathbb{R}$, encontrar
$u^h=v^h+\overline{u}^h \in \delta^h$ tal que
$\forall \delta u^h \in \nu^h$ se cumple:
\begin{multline}
\int_{\Omega}
\bm{\nabla}^T \delta u^h \cdot \bm{k} \bm{\nabla} v^h \, d \Omega=
\int_{\Omega}
f \delta u^h \, d \Omega -
\int_{\partial_t \Omega} q \delta u^h \, d \Gamma- \\
\int_{\Omega}
\bm{\nabla}^T \delta u^h \cdot \bm{k} \bm{\nabla} \overline{u}^h \, d \Omega
\label{calorgal}
\end{multline}
\end{frame}

\begin{frame}
  \frametitle{La Idea de Galerkin}
  \begin{itemize}
      \item El problema débil es de \textbf{dimensión infinita} (debe cumplirse para infinitas $\delta u$).
      \item \textbf{La aproximación (MEF):}
      \begin{enumerate}
          \item Discretizamos el dominio $\Omega$ en $n_{elm}$ elementos finitos.
          \item Definimos $n_{nod}$ nodos.
      \end{enumerate}
  \end{itemize}
  \begin{block}{La idea: Interpolar la Solución}
    \begin{pizarra}
      Aproximamos la solución $u$ (desconocida) y la función de peso $\delta u$:
      \begin{align*}
          u(x) \approx u^h(x) &= \sum_{A=1}^{n_{nod}} N_A(x) d_A \\
          \delta u(x) \approx \delta u^h(x) &= \sum_{B=1}^{n_{nod}} N_B(x) c_B
      \end{align*}
      \begin{itemize}
          \item $N_A(x)$ = \textbf{Funciones de Forma} (conocidas, ej: lineales)
          \item $d_A$ = \textbf{Incógnitas} (valor de $u$ en el nodo A)
          \item $c_B$ = Coeficientes arbitrarios
      \end{itemize}
    \end{pizarra}
  \end{block}
\end{frame}

\begin{frame}
  \frametitle{Sustitución}
  
  \begin{block}{De 1 Ecuación Integral a $n_{eq}$ Ecuaciones Algebraicas}
    \begin{enumerate}
        \item Sustituimos $u^h$ y $\delta u^h$ en la forma débil.
        \item Obtenemos una ecuación muy grande que depende de $c_B$.
        \item Como la ecuación debe ser cierta para \textit{cualquier} valor de $c_B$, el término que multiplica a $c_B$ debe ser cero.
        \item Esto nos da una ecuación \textit{para cada nodo $B$}.
        \item \textbf{Resultado:} Un sistema de $n_{eq}$ ecuaciones lineales con $n_{eq}$ incógnitas ($d_A$).
    \end{enumerate}
  \end{block}
\end{frame}


%%%%%%%%%%%%%%%%%%%%%%%%%%%%%%%%%%%%%%%%%%%%%%%%%%%%%%%%%%%%%%%%%%%%%%%%%%%%%%%%
% --- MÓDULO 4 ---
%%%%%%%%%%%%%%%%%%%%%%%%%%%%%%%%%%%%%%%%%%%%%%%%%%%%%%%%%%%%%%%%%%%%%%%%%%%%%%%%
\section[M4: Formulación de elementos finitos]{MÓDULO 4: Formulación de elementos finitos}
\begin{frame}
\frametitle{Formulación matricial}
Sean $\eta=\{ 1,2,\ldots,n_{\textrm{nod}} \}$,
$\eta_u \subset \eta$ el conjunto de nodos con temperaturas prescritas
y $\eta-\eta_u$ el conjunto complementario de $\eta_u$ en $\eta$:
$\circ(\eta-\eta_u)=n_{\textrm{eq}}$.

Los elemento de $\nu^h$ en (\ref{calorgal}) se expresan:
\begin{equation}
\delta u^h=\sum_{A \in \eta-\eta_u} c_A N_A \qquad
v^h=\sum_{A \in \eta-\eta_u} d_A N_A \label{termnuh}
\end{equation}
y por otra parte:
\begin{equation}
\overline{u}^h=\sum_{A \in \eta_u} \overline{u}_A N_A
\qquad \textrm{siendo } \overline{u}_A=\overline{u}(x_A) \label{termuh}
\end{equation}
\end{frame}
%%%%%%%%%%%%%%%%%%%%%%%%%%%%%%%%%%%%%%%%%%%%%%%%%%%%%%%%%%%%%%%%%%%%%%%%%%%%%%
\begin{frame}
\frametitle{Formulación matricial}

Sustituyendo (\ref{termnuh}, \ref{termuh}) en (\ref{calorgal}), y teniendo
en cuenta que los coeficientes $c_A$ son arbitrarios, resulta el 
sistema de ecuaciones:
\begin{multline}
\sum_{B \in \eta-\eta_u} \int_{\Omega} \bm{\nabla} N_A \cdot \bm{k} \bm{\nabla} N_B \, d_B=
\int_{\Omega} N_A f \, d\Omega+
\int_{\partial_t \Omega} N_A \overline{q} \, d\Gamma \\
- \sum_{B \in \eta_u} \int_{\Omega} \bm{\nabla} N_A \cdot \bm{k} \bm{\nabla} N_B \, u_B
\quad (A \in \eta-\eta_u) \label{siseqcal}
\end{multline}
Para expresar en forma matricial este sistema
es necesario establecer la numeración global de las $\eta-\eta_u$ ecuaciones
que lo constituyen:
\begin{equation}
id(A)=\left\{
\begin{array}{ll}
P & \textrm{ si } A \in \eta-\eta_u \\
0 & \textrm{ si } A \in \eta_u
\end{array}
\right. \label{idmcalor}
\end{equation}
siendo $P$ el número de la ecuación global correspondiente al nodo $A$, y tal
que $1 \leq P \leq n_{\rm eq}$. La dimensión de la matriz $\bm{id}$ es $n_{\rm nod}$.
\end{frame}
%%%%%%%%%%%%%%%%%%%%%%%%%%%%%%%%%%%%%%%%%%%%%%%%%%%%%%%%%%%%%%%%%%%%%%%%%%%%%%
\begin{frame}
\frametitle{Formulación matricial}
La expresión matricial del sistema de ecuaciones (\ref{siseqcal}) es:
\begin{equation}
\mathbf{K} \bm{d}=\bm{F}
\end{equation}
donde:
\begin{equation}
\bm{K}= [K_{PQ}]                                        ,\quad
\bm{d}= \{d_Q\}                                              ,\quad
\bm{F}= \{F_P\}                                              ,\quad
1 \leq P,Q \leq n_{\textrm{eq}}
\end{equation}
siendo:
\begin{align}
K_{PQ}&=\int_{\Omega} k_{ij} \frac{\partial N_A}{\partial x_i}
                             \frac{\partial N_B}{\partial x_j}
                             d \Omega                        ,\quad
P=id(A)                                                      ,\quad
Q=id(B)
                                                             \label{kglpoi} \\
F_P&= \int_{\Omega} N_A f \, d\Omega+
      \int_{\partial_t \Omega} N_A \overline{q} \, d\Gamma
     -                                                       \nonumber \\
   &\mbox{ } \sum_{B \in \eta_u} \left(
      \int_{\Omega} k_{ij} \frac{\partial N_A}{\partial x_i}
      \frac{\partial N_B}{\partial x_j} d \Omega
                          \right) \overline{u}_B
\end{align}
{\em La matriz de rigidez es simétrica.}
\end{frame}

\begin{frame}[allowframebreaks]
  \frametitle{Obtención de $Kd=F$}
  
 % \begin{pizarra}
    La ecuación para el nodo $B$ es:
    \begin{align*}
        \sum_{A \in \eta-\eta_u} \left[ \int_{\Omega} (\nabla N_B) \cdot k (\nabla N_A) d\Omega \right] d_A = & \int_{\Omega} N_B f d\Omega \\
        & + \int_{\partial_t\Omega} N_B \overline{q} d\Gamma \\
        & - \int_{\Omega} (\nabla N_B) \cdot k \nabla(\overline{u}^h) d\Omega
    \end{align*}
    
    Esto es, exactamente, la fila $B$ de un sistema matricial:
    \[ \sum_A K_{BA} d_A = F_B \implies \mathbf{K} \mathbf{d} = \mathbf{F} \]
 % \end{pizarra}
\end{frame}

\begin{frame}
  \frametitle{Definición de Términos}
  
%  \begin{pizarra}
    \begin{itemize}
        \item \textbf{Vector de Incógnitas $\mathbf{d}$:}
        \[ \{d_A\} \]
        (Los valores de $u$ en los nodos que buscamos).
        
        \item \textbf{Matriz de Rigidez $\mathbf{K}$:}
        \begin{equation*}
            K_{BA} = \int_{\Omega} \nabla N_B \cdot k \nabla N_A d\Omega
        \end{equation*}
        (Interacción entre nodos $B$ y $A$. Es simétrica).
        
        \item \textbf{Vector de Fuerzas $\mathbf{F}$:}
        \begin{align*}
            F_B = & \int_{\Omega} N_B f d\Omega \quad \text{(Fuentes internas)} \\
                  & + \int_{\partial_t\Omega} N_B \overline{q} d\Gamma \quad \text{(Flujo de contorno)} \\
                  & - \int_{\Omega} \nabla N_B \cdot k \nabla \overline{u}^h d\Omega \quad \text{(Efecto de C. Dirichlet)}
        \end{align*}
    \end{itemize}
%  \end{pizarra}
\end{frame}

\begin{frame}
  \frametitle{Formulación de Elementos Finitos}
  
  \begin{block}{La Clave Computacional del MEF}
    Calcular $\int_{\Omega}$ es difícil. Pero la integral del todo es la suma de las partes (elementos):
    \[ \int_{\Omega} (\dots) d\Omega = \sum_{e=1}^{n_{elm}} \int_{\Omega^e} (\dots) d\Omega \]
  \end{block}
  
  \begin{itemize}
      \item Calculamos una \textbf{matriz de rigidez elemental} $k^e$ para cada elemento.
      \item Calculamos un \textbf{vector de fuerzas elemental} $f^e$ para cada elemento.
      \item (Dentro de un elemento simple, como un triángulo, estas integrales son fáciles).
      \item Luego, ''sumamos'' todas las matrices elementales para obtener las globales $K$ y $F$.
      \item $K = \sum_e K^e$ (Ensamblaje)
  \end{itemize}
\end{frame}
\begin{frame}
\frametitle{Formulación de elementos finitos}

Considerando un elemento genérico $e$ de $n_{\textrm{nen}}$ nodos, las
expresiones de la matriz de rigidez elemental y del vector de fuerzas, son
respectivamente:
\begin{align*}
\mathbf{K}^e&=\int_{\Omega^e} \mathbf{B}^T \mathbf{k} \mathbf{B} \, d \Omega \\
\bm{F}^e&=\int_{\Omega^e} \bm{N} f \, d\Omega 
-\int_{\partial_t \Omega^e} \bm{N} \overline{q} \, d\Gamma-
\sum_{b=1}^{n_{\textrm{nen}}} \left(
\int_{\Omega^e} \bm{\nabla} N_a \cdot \bm{k} \bm{\nabla} N_b \, d \Omega \right)
\overline{u}_b
\end{align*}
siendo:
$
\mathbf{B}=\left[
\bm{B}_1,\,\bm{B}_2,\ldots,\bm{B}_{\textrm{nen}}
\right], \qquad
\bm{B}_a=\bm{\nabla}N_a
$
A partir de los vectores y matrices elementales se obtienen los
globales mediante un operador de ensamble ${\Large \bf{\sf A}[\cdot]}$:
\begin{align}
\mathbf{K} &=
\stackrel{{n_{\rm numel}}}{\underset{e=1}{\mbox{\Large{\bf {\sf A}}}}}
\mathbf{k}^e \label{assemK} \\
    \bm{F} &=
\stackrel{{n_{\rm numel}}}{\underset{e=1}{\mbox{\Large{\bf {\sf A}}}}}
\bm{f}^e \label{assemF}
\end{align}
\end{frame}
%%%%%%%%%%%%%%%%%%%%%%%%%%%%%%%%%%%%%%%%%%%%%%%%%%%%%%%%%%%%%%%%%%%%%%%%%%%%%%%%
% --- MÓDULO 5 ---
%%%%%%%%%%%%%%%%%%%%%%%%%%%%%%%%%%%%%%%%%%%%%%%%%%%%%%%%%%%%%%%%%%%%%%%%%%%%%%%%
\section[M5: Ensamblaje de las ecuaciones]{MÓDULO 5: Ensamblaje de las ecuaciones}

\begin{frame}
\frametitle{Ensamblaje de las ecuaciones}
En las expresiones (\ref{assemK}) y (\ref{assemF}), el operador
${\mbox{\LARGE{\bf {\sf A}}}}$
genera la matriz de rigidez y el vector de fuerzas globales, a partir de
las matrices y vectores calculados en cada elemento. A este proceso se le
denomina {\em ensamble} o {\em ensamblaje}.
\begin{align*}
IEN(\underbrace{a}_{\textrm{nodo local}},\underbrace{e}_{\textrm{elemento}})&=
\underbrace{A}_{\textrm{nodo global}} \\
ID(A)&=\underbrace{N}_{\textrm{ecuación}} \\
LM(a,e)&=ID(IEN(a,e))
\end{align*}
Las matrices $IEN$ e $ID$ se construyen a partir de los datos de
entrada (conectividad y condiciones de contorno)
\end{frame}

\begin{frame}
\frametitle{Ensamblaje de las ecuaciones. Ejemplo}
\parbox{0.20\textwidth}{
\includegraphics{ensam}
} \hfill
\parbox{0.30\textwidth}{ \begin{footnotesize}
\begin{align*}
ID&=(
\begin{array}{cccccccccccc}
$0$ & $1$ & $2$ & $0$ & $3$ & $4$ & $0$ & $5$ & $6$ & $0$ & $7$ & $8$
\end{array}
) \\
IEN&=\left(
\begin{array}{cccccc}
$1$ & $2$ & $4$ & $5$ & $ 7$ & $ 8$ \\
$2$ & $3$ & $5$ & $6$ & $ 8$ & $ 9$ \\
$5$ & $6$ & $8$ & $9$ & $11$ & $12$ \\
$4$ & $5$ & $7$ & $8$ & $10$ & $11$
\end{array}
\right) \\
LM&=\left(
\begin{array}{cccccc}
$0$ & $1$ & $0$ & $3$ & $ 0$ & $ 5$ \\
$1$ & $2$ & $3$ & $4$ & $ 5$ & $ 6$ \\
$3$ & $4$ & $5$ & $6$ & $ 7$ & $ 8$ \\
$0$ & $3$ & $0$ & $5$ & $ 0$ & $ 7$
\end{array}
\right)
\end{align*}
\end{footnotesize}}
\end{frame}
\newpage
%%%%%%%%%%%%%%%%%%%%%%%%%%%%%%%%%%%%%%%%%%%%%%%%%%%%%%%%%%%%%%%%%%%%%%%%%%%%%%
\begin{frame}
\frametitle{Ensamblaje de las ecuaciones. Ejemplo}
Por ejemplo, para el elemento $3$:
$$
LM(a,3)=\left(
\begin{array}{c}
0 \\
3 \\
5 \\
0
\end{array}
\right) \Rightarrow \qquad
\begin{array}{c}
K_{33} \leftarrow K_{33}+K^e_{22} \\
K_{35} \leftarrow K_{35}+K^e_{23} \\
K_{53} \leftarrow K_{53}+K^e_{32} \\
K_{55} \leftarrow K_{55}+K^e_{33} \\
F_3 \leftarrow F_3 + F^e_2 \\
F_5 \leftarrow F_5 + F^e_3
\end{array}
$$
\end{frame}
%%%%%%%%%%%%%%%%%%%%%%%%%%%%%%%%%%%%%%%%%%%%%%%%%%%%%%%%%%%%%%%%%%%%%%%%%%%%%%






%%%%%%%%%%%%%%%%%%%%%%%%%%%%%%%%%%%%%%%%%%%%%%%%%%%%%%%%%%%%%%%%%%%%%%%%%%%%%%%%
% --- MÓDULO 6 ---
%%%%%%%%%%%%%%%%%%%%%%%%%%%%%%%%%%%%%%%%%%%%%%%%%%%%%%%%%%%%%%%%%%%%%%%%%%%%%%%%
\section[M6: Resumen]{MÓDULO 6: Conclusión y Resumen}

\begin{frame}
  \frametitle{Resumen del Flujo de Trabajo del MEF}
  
  \begin{block}{El Proceso Completo}
    \begin{enumerate}
      \item \textbf{Física} (Ej: Flujo Poroso) $\implies$ \textbf{Modelo (Fuerte)} (PDEs).
      \item \textbf{Forma Débil} (Integración por partes).
      \item \textbf{Discretización (Galerkin)} (Aproximar $u \approx \sum N_A d_A$).
      \item \textbf{Sistema Matricial} (Obtener $K_{BA}$ y $F_B$ de las integrales).
      \item \textbf{Ensamblaje} (Construir $K = \sum k^e$ y $F = \sum f^e$).
      \item \textbf{Resolver} (Resolver el sistema lineal $Kd=F$ para $d$).
      \item \textbf{Post-Proceso} (Calcular flujos, gradientes, etc.).
    \end{enumerate}
  \end{block}
\end{frame}

\begin{frame}
  \frametitle{Convergencia}
  
  \begin{itemize}
      \item La solución aproximada $u^h$ \textit{converge} a la solución real $u$ a medida que la malla se refina ($h \rightarrow 0$).
      
      \item Para ello, las funciones de forma $N_A$ deben cumplir:
      \begin{itemize}
          \item \textbf{Completitud:} Capaces de representar un campo constante y un gradiente lineal.
          \item \textbf{Compatibilidad:} Continuidad $C^0$ entre los elementos.
      \end{itemize}
  \end{itemize}
\end{frame}

\begin{frame}
  \frametitle{Fin}
  \begin{center}
    \Huge Gracias por la atención
  \end{center}
\end{frame}
%%%%%%%%%%%%%%%%%%%%%%%%%%%%%%%%%%%%%%%%%%%%%%%%%%%%%%%%%%%%%%%%%%%%%%%%%%%%%%

%%%%%%%%%%%%%%%%%%%%%%%%%%%%%%%%%%%%%%%%%%%%%%%%%%%%%%%%%%%%%%%%%%%%%%%%%%%%%%
\end{document}
