\documentclass[a4paper]{article}

\usepackage[utf8]{inputenc}    
\usepackage[T1]{fontenc}
\usepackage[spanish]{babel}

\usepackage{bm}
\usepackage{amsxtra} 
\usepackage{amssymb}% to get the \mathbb alphabet
\usepackage{amsmath}

\usepackage{moodle}

\spanishdecimal{.}

\begin{document}
	
\begin{quiz}{FO-1-20-21}

%%%%%%%%%%%%%%%%%%%%%%%%%%%%%%%%%%%%%%%%%%%%%%%%%%%%%%%%%%%%%%%%%%%%%%%%%%%%%
\begin{multi}{p1}
En un problema lineal de conducci\'on de calor (modelo de difusi\'on) discretizado con
una malla de $N$ nodos, con $n_u$ valores conocidos de la temperatura en los
nodos, y $n_t$ valores conocidos del flujo en direccion normal, el sistema
lineal de ecuaciones resultante de la formulaci\'on de elementos finitos:
\item* Tiene $N-n_u$ grados de libertad
\item[fraction=-33.333] Tiene $N-n_t$ grados de libertad
\item[fraction=-33.333] Tiene $N-n_u-n_t$ grados de libertad
\item[fraction=-33.333] Tiene $N+n_u-n_t$ grados de libertad
\item[fraction=0] Respuesta en blanco
\end{multi}
%%%%%%%%%%%%%%%%%%%%%%%%%%%%%%%%%%%%%%%%%%%%%%%%%%%%%%%%%%%%%%%%%%%%%%%%%%%%%
\begin{multi}{p2}
El valor de la temperatura en el punto $A$ de la figura vale:
\item* $318 \; ^{\circ}$K
\item[fraction=-33.333] $427 \; ^{\circ}$K
\item[fraction=-33.333] $365 \; ^{\circ}$K
\item[fraction=-33.333] $275 \; ^{\circ}$K
\item[fraction=0] Respuesta en blanco
\end{multi}
%%%%%%%%%%%%%%%%%%%%%%%%%%%%%%%%%%%%%%%%%%%%%%%%%%%%%%%%%%%%%%%%%%%%%%%%%%%%%
\begin{multi}{p3}
En el punto $A$ de la figura, el valor calculado de la componte $x$ del flujo de calor es:
\item* $1268$ W/m$^2$
\item[fraction=-33.333] $3970$ W/m$^2$
\item[fraction=-33.333] $2176$ W/m$^2$
\item[fraction=-33.333] $765$ W/m$^2$
\item[fraction=0] Respuesta en blanco
\end{multi}
%%%%%%%%%%%%%%%%%%%%%%%%%%%%%%%%%%%%%%%%%%%%%%%%%%%%%%%%%%%%%%%%%%%%%%%%%%%%%
\begin{multi}{p4}
En el punto $A$ de la figura, el valor calculado de la componte $y$ del flujo de calor es:
\item* $-3970$ W/m$^2$
\item[fraction=-33.333] $-1268$ W/m$^2$
\item[fraction=-33.333] $-2176$ W/m$^2$
\item[fraction=-33.333] $-765$ W/m$^2$
\item[fraction=0] Respuesta en blanco
\end{multi}
%%%%%%%%%%%%%%%%%%%%%%%%%%%%%%%%%%%%%%%%%%%%%%%%%%%%%%%%%%%%%%%%%%%%%%%%%%%%%
\begin{multi}{p5}
El valor m\'aximo, en valor absoluto, del flujo de calor en direcci\'on $y$ se obtiene en:
\item* La zona pr\'oxima al punto $A$ de la figura
\item[fraction=-33.333] El borde vertical con la temperatura impuesta
\item[fraction=-33.333] El borde horizontal con la temperatura impuesta
\item[fraction=-33.333] Ninguna de las otras respuestas es correcta
\item[fraction=0] Respuesta en blanco
\end{multi}
%%%%%%%%%%%%%%%%%%%%%%%%%%%%%%%%%%%%%%%%%%%%%%%%%%%%%%%%%%%%%%%%%%%%%%%%%%%%%
\begin{multi}{p6}
Los valores de la componente $x$ del vector flujo est\'an
comprendidos aproximadamente entre:
\item* $294$ y $7210$ W/m$^2$
\item[fraction=-33.333] $-294$ y $-7210$ W/m$^2$
\item[fraction=-33.333] $55.3$ y $3970$ W/m$^2$
\item[fraction=-33.333] $-55.3$ y $-3970$ W/m$^2$
\item[fraction=0] Respuesta en blanco
\end{multi}
%%%%%%%%%%%%%%%%%%%%%%%%%%%%%%%%%%%%%%%%%%%%%%%%%%%%%%%%%%%%%%%%%%%%%%%%%%%%%
\begin{multi}{p7}
El n\'umero de grados de libertad del modelo es:
\item* $944$
\item[fraction=-33.333] $900$
\item[fraction=-33.333] $992$
\item[fraction=-33.333] Ninguna de las otras respuestas es correcta
\item[fraction=0] Respuesta en blanco
\end{multi}
%%%%%%%%%%%%%%%%%%%%%%%%%%%%%%%%%%%%%%%%%%%%%%%%%%%%%%%%%%%%%%%%%%%%%%%%%%%%%
\begin{multi}{p8}
	En la formulaci\'on d\'ebil del problema de conducci\'on de calor las funciones
	de prueba $\delta u$  verifican:
	\item* $\delta u=0$ en la parte del contorno con la temperatura impuesta
	\item[fraction=-33.333] $\delta u=0$ en la parte del contorno con el flujo
	en direcci\'on normal impuesto
	\item[fraction=-33.333] $\delta u=0$ en todo el contorno
	\item[fraction=-33.333] Las dem\'as respuestas son incorrectas
	\item[fraction=0] Respuesta en blanco
\end{multi}
%%%%%%%%%%%%%%%%%%%%%%%%%%%%%%%%%%%%%%%%%%%%%%%%%%%%%%%%%%%%%%%%%%%%%%%%%%%%%
\begin{multi}{p9}
En un problema lineal de conducci\'on de calor (modelo de difusi\'on) la ley
constitutiva (ley de Fourier) relaciona:
	\item* El vector flujo de calor y el gradiente de la temperatura
	\item[fraction=-33.333] El vector flujo de calor y la temperatura
	\item[fraction=-33.333] El flujo de calor en direcci\'on normal al contorno y la temperatura
	\item[fraction=-33.333] Ninguna de las otras respuestas es correcta
	\item[fraction=0] Respuesta en blanco
\end{multi}
%%%%%%%%%%%%%%%%%%%%%%%%%%%%%%%%%%%%%%%%%%%%%%%%%%%%%%%%%%%%%%%%%%%%%%%%%%%%%
\begin{multi}{p10}
	Despu\'es de unir los dos bloques, el n\'umero del nodo situado en el punto $A$ de la figura es:
	\item* $481$
	\item[fraction=-33.333] $496$
	\item[fraction=-33.333] $502$
	\item[fraction=-33.333] $475$
	\item[fraction=0] Respuesta en blanco
\end{multi}
%%%%%%%%%%%%%%%%%%%%%%%%%%%%%%%%%%%%%%%%%%%%%%%%%%%%%%%%%%%%%%%%%%%%%%%%%%%%%

\end{quiz}

\end{document}
