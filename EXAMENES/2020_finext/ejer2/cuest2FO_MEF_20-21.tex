\documentclass[a4paper]{article}

\usepackage[utf8]{inputenc}    
\usepackage[T1]{fontenc}
\usepackage[spanish]{babel}

\usepackage{bm}
\usepackage{amsxtra} 
\usepackage{amssymb}% to get the \mathbb alphabet
\usepackage{amsmath}

\usepackage{moodle}

\spanishdecimal{.}

\begin{document}

\begin{quiz}{FO-2-20-21}

%%%%%%%%%%%%%%%%%%%%%%%%%%%%%%%%%%%%%%%%%%%%%%%%%%%%%%%%%%%%%%%%%%%%%%%%%%%%%
\begin{multi}{p1}
Para integrar de forma exacta el polinomio $p(x)=x^7+x^3+1$ utilizando
una cuadratura de Gauss, el m\'inimo n\'umero de puntos de integraci\'on necesario
es:
	\item* $4$
	\item[fraction=-33.333] $3$
	\item[fraction=-33.333] $5$
	\item[fraction=-33.333] $7$
	\item[fraction=0] Respuesta en blanco
\end{multi}
%%%%%%%%%%%%%%%%%%%%%%%%%%%%%%%%%%%%%%%%%%%%%%%%%%%%%%%%%%%%%%%%%%%%%%%%%%%%%
\begin{multi}{p2}
Los elementos isoparam\'etricos se caracterizan por:
	\item* Las funciones de interpolaci\'on de las coordenadas
	son las mismas que las funciones de interpolaci\'on de los desplazamientos
	\item[fraction=-33.333] La dimensi\'on de la matriz de rigidez del elemento
	es $n \times n$, siendo $n$ el n\'umero de nodos de dicho elemento
	\item[fraction=-33.333] Los nodos del elemento necesariamente deben estar
	situados en los v\'ertices o en los lados del mismo, no pudiendo haber nodos
	en su interior
	\item[fraction=-33.333] Verifican el requisito de complitud pero no pasan
	la prueba de la parcela.
	\item[fraction=0] Respuesta en blanco
\end{multi}
%%%%%%%%%%%%%%%%%%%%%%%%%%%%%%%%%%%%%%%%%%%%%%%%%%%%%%%%%%%%%%%%%%%%%%%%%%%%%
\begin{multi}{p3}
	El n\'umero de grados de libertad del modelo de elementos finitos es:
	\item* $32640$
	\item[fraction=-33.333] $11316$
	\item[fraction=-33.333] $7956$
	\item[fraction=-33.333] $26950$
	\item[fraction=0] Respuesta en blanco
\end{multi}
%%%%%%%%%%%%%%%%%%%%%%%%%%%%%%%%%%%%%%%%%%%%%%%%%%%%%%%%%%%%%%%%%%%%%%%%%%%%%
\begin{multi}{p4}
	El desplazamiento del nodo en el que est\'a aplicada la carga puntual,
	en la direcci\'on de dicha carga, vale:
	\item* $5.5$ mm
	\item[fraction=-33.333] $-2.8$ mm
	\item[fraction=-33.333] $1.4$ cm
	\item[fraction=-33.333] $-1.2$ cm
	\item[fraction=0] Respuesta en blanco
\end{multi}
%%%%%%%%%%%%%%%%%%%%%%%%%%%%%%%%%%%%%%%%%%%%%%%%%%%%%%%%%%%%%%%%%%%%%%%%%%%%%
\begin{multi}{p5}
	El valor m\'aximo del desplazamiento vertical vale:
	\item* $-5.5$ cm
	\item[fraction=-33.333] $-9.3$ cm
	\item[fraction=-33.333] $-2.7$ cm
	\item[fraction=-33.333] $-0.9$ cm
	\item[fraction=0] Respuesta en blanco
\end{multi}
%%%%%%%%%%%%%%%%%%%%%%%%%%%%%%%%%%%%%%%%%%%%%%%%%%%%%%%%%%%%%%%%%%%%%%%%%%%%%
\begin{multi}{p6}
	El valor m\'aximo de la tensi\'on de Von Mises vale:
	\item* $52.3$ MPa
	\item[fraction=-33.333] $38.4$ MPa
	\item[fraction=-33.333] $73.7$ MPa
	\item[fraction=-33.333] $101.3$ MPa
	\item[fraction=0] Respuesta en blanco
\end{multi}
%%%%%%%%%%%%%%%%%%%%%%%%%%%%%%%%%%%%%%%%%%%%%%%%%%%%%%%%%%%%%%%%%%%%%%%%%%%%%
\begin{multi}{p7}
	Visualizando los contornos de la tensi\'on de Von MisesPara evaluar 
	cualitativamente la posible rotura del perfil, se concluye que \'esta
	se producir\'ia:
	\item* Localmente en el punto de aplicaci\'on de la carga puntual
	\item[fraction=-33.333] En la zona inferior del empotramiento
	\item[fraction=-33.333] En la zona superior de la secci\'on
	situada en el extremo libre
	\item[fraction=-33.333] En una secci\'on situada entre el extremo libre
	y el extremo empotrado.
	\item[fraction=0] Respuesta en blanco
\end{multi}
%%%%%%%%%%%%%%%%%%%%%%%%%%%%%%%%%%%%%%%%%%%%%%%%%%%%%%%%%%%%%%%%%%%%%%%%%%%%%
\begin{multi}{p8}
	La reacci\'on vertical en el nodo situado en el origen de coordenadas vale:
	\item* $10$ kN
	\item[fraction=-33.333] $4$ kN
	\item[fraction=-33.333] $17$ kN
	\item[fraction=-33.333] $0$ kN
	\item[fraction=0] Respuesta en blanco
\end{multi}
%%%%%%%%%%%%%%%%%%%%%%%%%%%%%%%%%%%%%%%%%%%%%%%%%%%%%%%%%%%%%%%%%%%%%%%%%%%%%
\begin{multi}{p9}
La tensi\'on $\sigma_{xx}$ en el nodo situado en el origen de coordenadas
vale:
\item* $-7.8$ MPa
\item[fraction=-33.333] $-3.4$ MPa
\item[fraction=-33.333] $5.9$ MPa
\item[fraction=-33.333] $10$ MPa
\item[fraction=0] Respuesta en blanco
\end{multi}
%%%%%%%%%%%%%%%%%%%%%%%%%%%%%%%%%%%%%%%%%%%%%%%%%%%%%%%%%%%%%%%%%%%%%%%%%%%%%
\begin{multi}{p10}
La tensi\'on de Von Mises en el nodo situado en el origen de coordenadas est\'a
comprendida aproximadamente entre:
	\item* $4.4$ MPa y $8.7$ MPa
	\item[fraction=-33.333] $8.7$ MPa y $13.1$ MPa
	\item[fraction=-33.333] $13.1$ MPa y $17.4$ MPa
	\item[fraction=-33.333] $17.4$ MPa y $21.8$ MPa
	\item[fraction=0] Respuesta en blanco
\end{multi}

\end{quiz}

\end{document}
