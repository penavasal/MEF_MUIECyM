\documentclass[a4paper]{article}

\usepackage[utf8x]{inputenc}    
\usepackage[T1]{fontenc}
\usepackage[spanish]{babel}
\usepackage{multicol}

\usepackage{wrapfig}
\usepackage{graphicx}

\usepackage{bm}
\usepackage{amsxtra} 
\usepackage{amssymb}% to get the \mathbb alphabet
\usepackage{amsmath}

\usepackage[box,completemulti,separateanswersheet]{automultiplechoice}    
\def\AMCformQuestion#1{\vspace{\AMCformVSpace}\par {\sc Pregunta #1:} }    
\def\AMCbeginQuestion#1#2{\par\noindent{\bf Pregunta #1}#2\hspace*{1em}}
\def\AMCcleardoublepage{\ifodd\thepage\clearpage\mbox{}\fi\clearpage}

\begin{document}

\AMCrandomseed{1237893}

%%%%%%%%%%%%%%%%%%%%%%%%%%%%%%%%%%%%%%%%%%%%%%%%%%%%%%%%%%%%%%%%%%%%%%%%%%%%%
\element{test1}{
\begin{question}{p1}
El movimiento vertical máximo (en valor absoluto) vale:
\begin{multicols}{2}
\begin{choices}
	\correctchoice{$1.44$ mm}
	\wrongchoice{$2.55$ mm}
	\wrongchoice{$1.07$ mm}
	\wrongchoice{$2.25$ mm}
\end{choices}
\end{multicols}
\end{question}
}
%%%%%%%%%%%%%%%%%%%%%%%%%%%%%%%%%%%%%%%%%%%%%%%%%%%%%%%%%%%%%%%%%%%%%%%%%%%%%
\element{test1}{
\begin{question}{p2}
El movimiento horizontal máximo (en valor absoluto) vale:
\begin{multicols}{2}
\begin{choices}
	\correctchoice{$0.74$ mm}
	\wrongchoice{$1.52$ mm}
	\wrongchoice{$1.07$ mm}
	\wrongchoice{$0.45$ mm}
\end{choices}
\end{multicols}
\end{question}
}
%%%%%%%%%%%%%%%%%%%%%%%%%%%%%%%%%%%%%%%%%%%%%%%%%%%%%%%%%%%%%%%%%%%%%%%%%%%%%
\element{test1}{
\begin{question}{p3}
El movimiento horizontal del nodo $I$ vale:
\begin{multicols}{2}
\begin{choices}
	\correctchoice{$-0.41$ mm}
	\wrongchoice{$0.67$ mm}
	\wrongchoice{$-0.12$ mm}
	\wrongchoice{$0.22$ mm}
\end{choices}
\end{multicols}
\end{question}
}
%%%%%%%%%%%%%%%%%%%%%%%%%%%%%%%%%%%%%%%%%%%%%%%%%%%%%%%%%%%%%%%%%%%%%%%%%%%%%
\element{test1}{
\begin{question}{p4}
La reacción vertical en el nodo $A$ vale:
\begin{multicols}{2}
\begin{choices}
	\correctchoice{$206.25$ kN}
	\wrongchoice{$243.80$ kN}
	\wrongchoice{$-254.42$ kN}
	\wrongchoice{$-193.10$ kN}
\end{choices}
\end{multicols}
\end{question}
}
%%%%%%%%%%%%%%%%%%%%%%%%%%%%%%%%%%%%%%%%%%%%%%%%%%%%%%%%%%%%%%%%%%%%%%%%%%%%%
\element{test1}{
\begin{question}{p5}
La fuerza de tracción máxima absoluta vale:
\begin{multicols}{2}
\begin{choices}
	\correctchoice{$439.40$ kN}
	\wrongchoice{$344.20$ kN}
	\wrongchoice{$510.50$ kN}
	\wrongchoice{$156.60$ kN} 
\end{choices}
\end{multicols}
\end{question}
}
%%%%%%%%%%%%%%%%%%%%%%%%%%%%%%%%%%%%%%%%%%%%%%%%%%%%%%%%%%%%%%%%%%%%%%%%%%%%%
\element{test1}{
\begin{question}{p6}
La tensión en la barra $KB$ vale:
\begin{multicols}{2}
\begin{choices}
	\correctchoice{$4.04$ MPa (de tracción)}
	\wrongchoice{$2.42$ MPa (de tracción)}
	\wrongchoice{$9.13$ MPa (de compresión)}
	\wrongchoice{$2.85$ MPa (de compresión)}
\end{choices}
\end{multicols}
\end{question}
}
%%%%%%%%%%%%%%%%%%%%%%%%%%%%%%%%%%%%%%%%%%%%%%%%%%%%%%%%%%%%%%%%%%%%%%%%%%%%%
\element{test1}{
\begin{question}{p7}
La tensión en la barra $IE$ vale:
\begin{multicols}{2}
\begin{choices}
	\correctchoice{$5.00$ MPa (de compresión)}
	\wrongchoice{$9.45$ MPa (de tracción)}
	\wrongchoice{$7.50$ MPa (de compresión)}
	\wrongchoice{$0.00$ MPa}
\end{choices}
\end{multicols}
\end{question}
}
%%%%%%%%%%%%%%%%%%%%%%%%%%%%%%%%%%%%%%%%%%%%%%%%%%%%%%%%%%%%%%%%%%%%%%%%%%%%%
\element{test1}{
\begin{question}{p8}
El método de elementos finitos resuelve de forma aproximada:
\begin{multicols}{2}
\begin{choices}
\correctchoice{La formulación débil de un problema de contorno}
\wrongchoice{La formulación fuerte de un problema de contorno}
\wrongchoice{Un sistema lineal de ecuaciones algebráicas}
\wrongchoice{Únicamente modelos de mecánica estructural cuya solución
es imposible obtener por otros métodos exactos o aproximados}
\end{choices}
\end{multicols}
\end{question}
}
%%%%%%%%%%%%%%%%%%%%%%%%%%%%%%%%%%%%%%%%%%%%%%%%%%%%%%%%%%%%%%%%%%%%%%%%%%%%%
\element{test1}{
\begin{question}{p9}
Un elemento ``truss'' (barra articulada):
\begin{multicols}{2}
\begin{choices}
	\correctchoice{Sólo transmite esfuerzos axiles}
	\wrongchoice{Sólo transmite esfuerzos axiles y cortantes}
	\wrongchoice{Sólo transmite esfuerzos axiles y momentos flectores}
	\wrongchoice{Transmite esfuerzos axiles, cortantes y momentos flectores}
\end{choices}
\end{multicols}
\end{question}
}
%%%%%%%%%%%%%%%%%%%%%%%%%%%%%%%%%%%%%%%%%%%%%%%%%%%%%%%%%%%%%%%%%%%%%%%%%%%%%
\element{test1}{
\begin{question}{p10}
Un modelo de elementos finitos de una estructura plana de barras articulada
tiene $n$ nodos y $m$ movimientos impuestos de valor conocido. El número de
grados de libertad (número de incógnitas) del modelo es:
\begin{multicols}{2}
\begin{choices}
	\correctchoice{$2n-m$}
	\wrongchoice{$3n-m$}
	\wrongchoice{$n+m$}
	\wrongchoice{$n-m$}
\end{choices}
\end{multicols}
\end{question}
}
%%%%%%%%%%%%%%%%%%%%%%%%%%%%%%%%%%%%%%%%%%%%%

\scoringDefaultS{b=1,m=-1/(N-1)}

\onecopy{1}{    

%%% beginning of the test sheet header:    

\noindent{\large\bf Método de los Elementos Finitos  \hfill MUECYM \hfill TEST \# 1}

\vspace*{.5cm}
\begin{minipage}{.4\linewidth}
  \centering 26 enero 2023
\end{minipage}

\begin{center}\em
Tiempo: 60 minutos.

  %Está prohibido el uso de teléfonos móviles.

  %Se atribuirá puntuación negativa a las respuestas incorrectas.

\end{center}
\vspace{1ex}

%%% end of the header

\shufflegroup{test1}
\insertgroup{test1}

\AMCcleardoublepage    
%\clearpage

\AMCformBegin    

%%% beginning of the answer sheet header

\noindent\AMCcode{nummat}{2}\hspace*{\fill}
\begin{minipage}{.7\linewidth}
$\longleftarrow{}$ Escriba su número de matrícula marcando los dígitos
en los recuadros (con ceros a la izquierda si el número es de menos de dos dígitos) y el nombre y apellidos debajo.

\vspace{3ex}

\namefield{\fbox{
   \begin{minipage}{.9\linewidth}
     Apellidos, Nombre:

     \vspace*{.5cm}\dotfill
     \vspace*{1mm}
   \end{minipage}
 }}
\end{minipage}

\begin{center}
 \bf\em Debe dar las respuestas exclusivamente en esta hoja (las respuestas en las demás hojas no serán tenidas en cuenta).
\end{center}

%%% end of the answer sheet header


\AMCform    

\AMCcleardoublepage    

}  

\end{document}
