\documentclass[a4paper]{article}

\usepackage[utf8x]{inputenc}    
\usepackage[T1]{fontenc}
\usepackage[spanish]{babel}
\usepackage{multicol}

\usepackage{wrapfig}
\usepackage{graphicx}

\usepackage{bm}
\usepackage{amsxtra} 
\usepackage{amssymb}% to get the \mathbb alphabet
\usepackage{amsmath}

\usepackage[box,completemulti,separateanswersheet]{automultiplechoice}    
\def\AMCformQuestion#1{\vspace{\AMCformVSpace}\par {\sc Pregunta #1:} }    
\def\AMCbeginQuestion#1#2{\par\noindent{\bf Pregunta #1}#2\hspace*{1em}}
\def\AMCcleardoublepage{\ifodd\thepage\clearpage\mbox{}\fi\clearpage}

\begin{document}

\AMCrandomseed{1237893}

%%%%%%%%%%%%%%%%%%%%%%%%%%%%%%%%%%%%%%%%%%%%%%%%%%%%%%%%%%%%%%%%%%%%%%%%%%%%%
\element{test1}{
\begin{question}{p1}
La diferencia entre los valores máximo y mínimo de la altura piezométrica vale:
\begin{multicols}{2}
\begin{choices}
	\correctchoice{$6.0$ m}
	\wrongchoice{$3.5$ m}
	\wrongchoice{$12.0$ m}
	\wrongchoice{$2.0$ m}
\end{choices}
\end{multicols}
\end{question}
}
%%%%%%%%%%%%%%%%%%%%%%%%%%%%%%%%%%%%%%%%%%%%%%%%%%%%%%%%%%%%%%%%%%%%%%%%%%%%%
\element{test1}{
\begin{question}{p2}
El valor de la velocidad vertical en la zona próxima a la pantalla, justo en el contacto agua-terreno y situada a la derecha de la misma, vale aproximadamente:
\begin{multicols}{2}
\begin{choices}
	\correctchoice{$2.2 \cdot 10^{-3}$ cm/s en sentido ascendente}
	\wrongchoice{$12.2 \cdot 10^{-3}$ cm/s en sentido ascendente.}
	\wrongchoice{$2.2 \cdot 10^{-3}$ m/s en sentido descendente.}
	\wrongchoice{$12.2 \cdot 10^{-3}$ cm/s en sentido descendente.}
\end{choices}
\end{multicols}
\end{question}
}
%%%%%%%%%%%%%%%%%%%%%%%%%%%%%%%%%%%%%%%%%%%%%%%%%%%%%%%%%%%%%%%%%%%%%%%%%%%%%
\element{test1}{
\begin{question}{p3}
El valor máximo del módulo de la velocidad vale:
\begin{multicols}{2}
\begin{choices}
	\correctchoice{$7 \cdot 10^{-3}$ cm/s}
	\wrongchoice{$26 \cdot 10^{-3}$ cm/s}
	\wrongchoice{$2 \cdot 10^{-3}$ cm/s}
	\wrongchoice{$15 \cdot 10^{-3}$ cm/s}
\end{choices}
\end{multicols}
\end{question}
}
%%%%%%%%%%%%%%%%%%%%%%%%%%%%%%%%%%%%%%%%%%%%%%%%%%%%%%%%%%%%%%%%%%%%%%%%%%%%%
\element{test1}{
\begin{question}{p4}
Tomando como peso específico del agua $\gamma=10000$ N/m$^3$, el valor de la presión
del fluido en el punto más bajo de la pantalla vale aproximadamente:
\begin{multicols}{2}
\begin{choices}
	\correctchoice{$11.5$ kPa}
	\wrongchoice{$11.5$ Pa}
	\wrongchoice{$23.5$ Pa}
	\wrongchoice{$23.5$ kPa}
\end{choices}
\end{multicols}
\end{question}
}
%%%%%%%%%%%%%%%%%%%%%%%%%%%%%%%%%%%%%%%%%%%%%%%%%%%%%%%%%%%%%%%%%%%%%%%%%%%%%
\element{test1}{
\begin{question}{p5}
El número de nodos de la malla es:
\begin{multicols}{2}
\begin{choices}
	\correctchoice{$8830$}
	\wrongchoice{$752$}
	\wrongchoice{$526$}
	\wrongchoice{$1276$} 
\end{choices}
\end{multicols}
\end{question}
}
%%%%%%%%%%%%%%%%%%%%%%%%%%%%%%%%%%%%%%%%%%%%%%%%%%%%%%%%%%%%%%%%%%%%%%%%%%%%%
\element{test1}{
\begin{question}{p6}
El valor absoluto más alto del flujo vertical se obtiene:
\begin{multicols}{2}
\begin{choices}
	\correctchoice{En uno de los paramentos verticales de la pantalla}
	\wrongchoice{En el extremo inferior de la pantalla)}
	\wrongchoice{En el substrato rocoso}
	\wrongchoice{Bajo la capa de agua de $9$ m de profundidad)}
\end{choices}
\end{multicols}
\end{question}
}
%%%%%%%%%%%%%%%%%%%%%%%%%%%%%%%%%%%%%%%%%%%%%%%%%%%%%%%%%%%%%%%%%%%%%%%%%%%%%
\element{test1}{
\begin{question}{p7}
El caudal que se filtra, en régimen estacionario, bajo la pantalla impermeable
vale aproximadamente:
\begin{multicols}{2}
\begin{choices}
	\correctchoice{$0.30$ l/s}
	\wrongchoice{$0.20$ m$^3$ l/s}
	\wrongchoice{$0.45$ l/s}
	\wrongchoice{$0.10$ l/h}
\end{choices}
\end{multicols}
\end{question}
}
%%%%%%%%%%%%%%%%%%%%%%%%%%%%%%%%%%%%%%%%%%%%%%%%%%%%%%%%%%%%%%%%%%%%%%%%%%%%%
\element{test1}{
\begin{question}{p8}
En la formulación débil de un problema de contorno, la derivada del
campo incógnita que aparecen en dicha formulación:
\begin{multicols}{2}
	\begin{choices}
		\correctchoice{Es de un orden menor que su derivada en la formulación fuerte}
		\wrongchoice{Es de un orden mayor que su derivada en la formulación fuerte}
		\wrongchoice{Es del mismo orden que su derivada en la formulación fuerte}
		\wrongchoice{Depende del tamaño de la malla de elementos finitos}
\end{choices}
\end{multicols}
\end{question}
}
%%%%%%%%%%%%%%%%%%%%%%%%%%%%%%%%%%%%%%%%%%%%%%%%%%%%%%%%%%%%%%%%%%%%%%%%%%%%%
\element{test1}{
\begin{question}{p9}
En un problema de conducción de calor, la condición natural de contorno:
$$\bm{q} \cdot \bm{n} = \overline{q}, \textrm{ en } \partial_t \Omega$$
se interpreta como:
\begin{multicols}{2}
\begin{choices}
	\correctchoice{El valor impuesto, de tipo escalar, que corresponde al
	flujo de la temperatura dirección normal al contorno $\partial_t \Omega$}
	\wrongchoice{El valor impuesto, de tipo vectorial, que corresponde al
	flujo de la temperatura en el contorno $\partial_t \Omega$}
	\wrongchoice{El valor impuesto, de tipo escalar, que corresponde al
	gradiente de la temperatura en dirección normal al contorno $\partial_t \Omega$}
	\wrongchoice{El valor impuesto, de tipo vectorial, que corresponde al
	gradiente de la temperatura en el contorno $\partial_t \Omega$}
\end{choices}
\end{multicols}
\end{question}
}
%%%%%%%%%%%%%%%%%%%%%%%%%%%%%%%%%%%%%%%%%%%%%%%%%%%%%%%%%%%%%%%%%%%%%%%%%%%%%
\element{test1}{
\begin{question}{p10}
En los modelos de difusión, la ecuación constitutiva relaciona:
\begin{multicols}{2}
\begin{choices}
	\correctchoice{El vector flujo con el gradiente de la variable primaria}
	\wrongchoice{La tensión con la deformación}
	\wrongchoice{El flujo en dirección normal con la temperatura impuesta}
	\wrongchoice{Las deformaciones con los desplazamientos}
\end{choices}
\end{multicols}
\end{question}
}
%%%%%%%%%%%%%%%%%%%%%%%%%%%%%%%%%%%%%%%%%%%%%

\scoringDefaultS{b=1,m=-1/(N-1)}

\onecopy{1}{    

%%% beginning of the test sheet header:    

\noindent{\large\bf Método de los Elementos Finitos  \hfill MUECYM \hfill TEST \# 1}

\vspace*{.5cm}
\begin{minipage}{.4\linewidth}
  \centering 26 enero 2023
\end{minipage}

\begin{center}\em
Tiempo: 60 minutos.

  %Está prohibido el uso de teléfonos móviles.

  %Se atribuirá puntuación negativa a las respuestas incorrectas.

\end{center}
\vspace{1ex}

%%% end of the header

\shufflegroup{test1}
\insertgroup{test1}

\AMCcleardoublepage    
%\clearpage

\AMCformBegin    

%%% beginning of the answer sheet header

\noindent\AMCcode{nummat}{2}\hspace*{\fill}
\begin{minipage}{.7\linewidth}
$\longleftarrow{}$ Escriba su número de matrícula marcando los dígitos
en los recuadros (con ceros a la izquierda si el número es de menos de dos dígitos) y el nombre y apellidos debajo.

\vspace{3ex}

\namefield{\fbox{
   \begin{minipage}{.9\linewidth}
     Apellidos, Nombre:

     \vspace*{.5cm}\dotfill
     \vspace*{1mm}
   \end{minipage}
 }}
\end{minipage}

\begin{center}
 \bf\em Debe dar las respuestas exclusivamente en esta hoja (las respuestas en las demás hojas no serán tenidas en cuenta).
\end{center}

%%% end of the answer sheet header


\AMCform    

\AMCcleardoublepage    

}  

\end{document}
