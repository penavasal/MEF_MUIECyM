\documentclass[a4paper]{article}

\usepackage[utf8x]{inputenc}    
\usepackage[T1]{fontenc}
\usepackage[spanish]{babel}
\usepackage{multicol}

\usepackage{wrapfig}
\usepackage{graphicx}

\usepackage{bm}
\usepackage{amsxtra} 
\usepackage{amssymb}% to get the \mathbb alphabet
\usepackage{amsmath}

\usepackage[box,completemulti,separateanswersheet]{automultiplechoice}    
\def\AMCformQuestion#1{\vspace{\AMCformVSpace}\par {\sc Pregunta #1:} }    
\def\AMCbeginQuestion#1#2{\par\noindent{\bf Pregunta #1}#2\hspace*{1em}}
\def\AMCcleardoublepage{\ifodd\thepage\clearpage\mbox{}\fi\clearpage}

\begin{document}

\AMCrandomseed{1237893}

%%%%%%%%%%%%%%%%%%%%%%%%%%%%%%%%%%%%%%%%%%%%%%%%%%%%%%%%%%%%%%%%%%%%%%%%%%%%%
\element{test1}{
\begin{question}{p1}
El método de los elementos finitos proporciona la solución aproximada de:
\begin{multicols}{2}
\begin{choices}
	\correctchoice{La formulación débil de un problema de contorno}
	\wrongchoice{La formulación fuerte de un problema de contorno}
	\wrongchoice{Un sistema lineal de ecuaciones}
	\wrongchoice{Ninguna de las otras respuestas es correcta}
\end{choices}
\end{multicols}
\end{question}
}
%%%%%%%%%%%%%%%%%%%%%%%%%%%%%%%%%%%%%%%%%%%%%%%%%%%%%%%%%%%%%%%%%%%%%%%%%%%%%
\element{test1}{
\begin{question}{p2}
En un problema lineal de conducción de calor (modelo de difusión) discretizado con
una malla de $N$ nodos, con $n_u$ valores conocidos de la temperatura en los
nodos, y $n_t$ valores conocidos del flujo en direccion normal, el sistema
lineal de ecuaciones resultante de la formulación de elementos finitos:
\begin{multicols}{2}
\begin{choices}
	\correctchoice{Tiene $N-n_u$ ecuaciones}
	\wrongchoice{Tiene $N-n_t$ ecuaciones}
	\wrongchoice{Tiene $N-n_u-n_t$ ecuaciones}
	\wrongchoice{Tiene $N+n_u-n_t$ ecuaciones}
\end{choices}
\end{multicols}
\end{question}
}
%%%%%%%%%%%%%%%%%%%%%%%%%%%%%%%%%%%%%%%%%%%%%%%%%%%%%%%%%%%%%%%%%%%%%%%%%%%%%
\element{test1}{
\begin{question}{p3}
En la formulación débil de un problema de contorno, la derivada del
campo incógnita que aparecen en dicha formulación:
\begin{multicols}{2}
\begin{choices}
	\correctchoice{Es de un orden menor que su derivada en la formulación fuerte}
	\wrongchoice{Es de un orden mayor que su derivada en la formulación fuerte}
	\wrongchoice{Es del mismo orden que su derivada en la formulación fuerte}
	\wrongchoice{Depende del tamaño de la malla de elementos finitos}
\end{choices}
\end{multicols}
\end{question}
}
%%%%%%%%%%%%%%%%%%%%%%%%%%%%%%%%%%%%%%%%%%%%%%%%%%%%%%%%%%%%%%%%%%%%%%%%%%%%%
\element{test1}{
\begin{question}{p4}
El valor mínimo de la temperatura se alcanza en:
\begin{multicols}{2}
\begin{choices}
	\correctchoice{Los bordes del taladro}
	\wrongchoice{Los bordes exteriores de la chapa}
	\wrongchoice{En el interior de la chapa}
	\wrongchoice{Hay varias zonas en los bordes exteriores y los
bordes del taladro, en los que se alcanza la temperatura mínima}
\end{choices}
\end{multicols}
\end{question}
}
%%%%%%%%%%%%%%%%%%%%%%%%%%%%%%%%%%%%%%%%%%%%%%%%%%%%%%%%%%%%%%%%%%%%%%%%%%%%%
\element{test1}{
\begin{question}{p5}
El valor de la temperatura en la chapa:
\begin{multicols}{2}
\begin{choices}
	\correctchoice{Está comprendido entre $300^{\circ}$ K y
$473^{\circ}$ K}
	\wrongchoice{Está comprendido entre $350^{\circ}$ K y
$450^{\circ}$ K}
	\wrongchoice{Está comprendido entre $373^{\circ}$ K y
$473^{\circ}$ K}
	\wrongchoice{Está comprendido entre $0^{\circ}$ K y
$100^{\circ}$ K}
\end{choices}
\end{multicols}
\end{question}
}
%%%%%%%%%%%%%%%%%%%%%%%%%%%%%%%%%%%%%%%%%%%%%%%%%%%%%%%%%%%%%%%%%%%%%%%%%%%%%
\element{test1}{
\begin{question}{p6}
Los valores del flujo de calor en dirección horizontal están
comprendidos aproximadamente entre:
\begin{multicols}{2}
\begin{choices}
	\correctchoice{$40.8$ y $1111$ kW/m$^2$}
	\wrongchoice{$300$ y $473$ kW/m$^2$}
	\wrongchoice{$40.8$ y $1111$ W/m$^2$}
	\wrongchoice{$300$ y $473$ W/m$^2$}
\end{choices}
\end{multicols}
\end{question}
}
%%%%%%%%%%%%%%%%%%%%%%%%%%%%%%%%%%%%%%%%%%%%%%%%%%%%%%%%%%%%%%%%%%%%%%%%%%%%%
\element{test1}{
\begin{question}{p7}
Los valores del flujo de calor en dirección vertical están
comprendidos aproximadamente entre:
\begin{multicols}{2}
\begin{choices}
	\correctchoice{$-944.1$ y $-1.31$ kW/m$^2$}
	\wrongchoice{$-0.3$ y $37.5$ kW/m$^2$}
	\wrongchoice{$-944$ y $-1.31$ W/m$^2$}
	\wrongchoice{$-0.3$ y $37.5$ W/m$^2$}
\end{choices}
\end{multicols}
\end{question}
}
%%%%%%%%%%%%%%%%%%%%%%%%%%%%%%%%%%%%%%%%%%%%%%%%%%%%%%%%%%%%%%%%%%%%%%%%%%%%%
\element{test1}{
\begin{question}{p8}
Los valores más bajos del flujo en dirección vertical
se obtienen:
\begin{multicols}{2}
\begin{choices}
	\correctchoice{En lo bordes del taladro}
	\wrongchoice{En los bordes verticales de la chapa}
	\wrongchoice{El flujo en dirección vertical es constante}
	\wrongchoice{Ninguna de las respuestas anteriores es correcta}
\end{choices}
\end{multicols}
\end{question}
}
%%%%%%%%%%%%%%%%%%%%%%%%%%%%%%%%%%%%%%%%%%%%%%%%%%%%%%%%%%%%%%%%%%%%%%%%%%%%%
\element{test1}{
\begin{question}{p9}
En un problema general de conducción de calor, el flujo en dirección normal
a los ejes o planos de simetría:
\begin{multicols}{2}
\begin{choices}
	\correctchoice{Es nulo}
	\wrongchoice{Está indeterminado}
	\wrongchoice{Coincide con la divergencia de la temperatura}
	\wrongchoice{Ninguna de las respuestas anteriores es correcta}
\end{choices}
\end{multicols}
\end{question}
}
%%%%%%%%%%%%%%%%%%%%%%%%%%%%%%%%%%%%%%%%%%%%%%%%%%%%%%%%%%%%%%%%%%%%%%%%%%%%%
\element{test1}{
\begin{question}{p10}
En el punto medio del borde horizontal inferior de la malla (perteneciente
a un eje de simetría), la temperatura vale:
\begin{multicols}{2}
\begin{choices}
	\correctchoice{$403.3^{\circ}$ K}
	\wrongchoice{$450.1^{\circ}$ K}
	\wrongchoice{$320.6^{\circ}$ K}
	\wrongchoice{$386.5^{\circ}$ K}
\end{choices}
\end{multicols}
\end{question}
}
%%%%%%%%%%%%%%%%%%%%%%%%%%%%%%%%%%%%%%%%%%%%%

\scoringDefaultS{b=1,m=-1/(N-1)}

\onecopy{1}{    

%%% beginning of the test sheet header:    

\noindent{\large\bf Método de los Elementos Finitos  \hfill MUECYM \hfill TEST \# 2}

\vspace*{.5cm}
\begin{minipage}{.4\linewidth}
  \centering 11 oct 2019
\end{minipage}

\begin{center}\em
Tiempo: 60 minutos.

  %Está prohibido el uso de teléfonos móviles.

  %Se atribuirá puntuación negativa a las respuestas incorrectas.

\end{center}
\vspace{1ex}

%%% end of the header

\shufflegroup{test1}
\insertgroup{test1}

\AMCcleardoublepage    
%\clearpage

\AMCformBegin    

%%% beginning of the answer sheet header

\noindent\AMCcode{nummat}{2}\hspace*{\fill}
\begin{minipage}{.7\linewidth}
$\longleftarrow{}$ Escriba su número de matrícula marcando los dígitos
en los recuadros (con ceros a la izquierda si el número es de menos de tres dígitos) y el nombre y apellidos debajo.

\vspace{3ex}

\namefield{\fbox{
   \begin{minipage}{.9\linewidth}
     Apellidos, Nombre:

     \vspace*{.5cm}\dotfill
     \vspace*{1mm}
   \end{minipage}
 }}
\end{minipage}

\begin{center}
 \bf\em Debe dar las respuestas exclusivamente en esta hoja (las respuestas en las demás hojas no serán tenidas en cuenta).
\end{center}

%%% end of the answer sheet header


\AMCform    

\AMCcleardoublepage    

}  

\end{document}
