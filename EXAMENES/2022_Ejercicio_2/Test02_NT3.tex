\documentclass[a4paper]{article}

\usepackage[utf8x]{inputenc}    
\usepackage[T1]{fontenc}
\usepackage[spanish]{babel}
\usepackage{multicol}

\usepackage{wrapfig}
\usepackage{graphicx}

\usepackage{bm}
\usepackage{amsxtra} 
\usepackage{amssymb}% to get the \mathbb alphabet
\usepackage{amsmath}

\usepackage[box,completemulti,separateanswersheet]{automultiplechoice}    
\def\AMCformQuestion#1{\vspace{\AMCformVSpace}\par {\sc Pregunta #1:} }    
\def\AMCbeginQuestion#1#2{\par\noindent{\bf Pregunta #1}#2\hspace*{1em}}
\def\AMCcleardoublepage{\ifodd\thepage\clearpage\mbox{}\fi\clearpage}

\begin{document}

\AMCrandomseed{1237893}

%%%%%%%%%%%%%%%%%%%%%%%%%%%%%%%%%%%%%%%%%%%%%%%%%%%%%%%%%%%%%%%%%%%%%%%%%%%%%
\element{test1}{
\begin{question}{p1}
El caudal que se filtra, en régimen estacionario, a la salida del túnel vale aproximadamente:
\begin{multicols}{2}
\begin{choices}
	\correctchoice{$2.83$ l/s}
	\wrongchoice{$4.21$ l/s}
	\wrongchoice{$1.56$ l/s}
	\wrongchoice{$1.01$ l/s}
\end{choices}
\end{multicols}
\end{question}
}
%%%%%%%%%%%%%%%%%%%%%%%%%%%%%%%%%%%%%%%%%%%%%%%%%%%%%%%%%%%%%%%%%%%%%%%%%%%%%
\element{test1}{
\begin{question}{p2}
El valor mínimo de la altura piezométrica se alcanza en:
\begin{multicols}{2}
\begin{choices}
	\correctchoice{La salida del túnel}
	\wrongchoice{Un punto de contacto del dique con el suelo 2}
	\wrongchoice{La zona de contacto del agua con el medio permeable}
	\wrongchoice{La zona de contacto del suelo 1 con la pantalla izquierda}
\end{choices}
\end{multicols}
\end{question}
}
%%%%%%%%%%%%%%%%%%%%%%%%%%%%%%%%%%%%%%%%%%%%%%%%%%%%%%%%%%%%%%%%%%%%%%%%%%%%%
\element{test1}{
\begin{question}{p3}
El valor máximo de la altura piezométrica se alcanza en:
\begin{multicols}{2}
\begin{choices}
	\correctchoice{La zona del Suelo 1 situada entre la pantalla izquierda y el dique, en contacto con el agua}
	\wrongchoice{La zona del suelo 1 situada entre el dique y la pantalla derecha, en contacto con el agua}
	\wrongchoice{La parte superior de la salida del túnel}
	\wrongchoice{La parte inferior de la salida del túnel}
\end{choices}
\end{multicols}
\end{question}
}
%%%%%%%%%%%%%%%%%%%%%%%%%%%%%%%%%%%%%%%%%%%%%%%%%%%%%%%%%%%%%%%%%%%%%%%%%%%%%
\element{test1}{
\begin{question}{p4}
Los valores de la velocidad horizontal están comprendidos aproximadamente entre:
\begin{multicols}{2}
\begin{choices}
	\correctchoice{$0.0$ y $5.5$ mm/s}
	\wrongchoice{$-0.2$ y $3.5$ mm/s}
	\wrongchoice{$-0.5$ y $1.5$ mm/s}
	\wrongchoice{$-0.5$ y $0.2$ mm/s}
\end{choices}
\end{multicols}
\end{question}
}
%%%%%%%%%%%%%%%%%%%%%%%%%%%%%%%%%%%%%%%%%%%%%%%%%%%%%%%%%%%%%%%%%%%%%%%%%%%%%
\element{test1}{
\begin{question}{p5}
Los valores de la velocidad vertical están comprendidos aproximadamente entre:
\begin{multicols}{2}
\begin{choices}
	\correctchoice{$-4.5$ y $3.2$ mm/s}
	\wrongchoice{$-2.5$ y $5.5$ mm/s}
	\wrongchoice{$0.0$ y $5.5$ mm/s}
	\wrongchoice{$-0.5$ y $0.2$ mm/s}
\end{choices}
\end{multicols}
\end{question}
}
%%%%%%%%%%%%%%%%%%%%%%%%%%%%%%%%%%%%%%%%%%%%%%%%%%%%%%%%%%%%%%%%%%%%%%%%%%%%%
\element{test1}{
\begin{question}{p6}
Si retiramos de la visualización de resultados el suelo 2, la velocidad de flujo total varía entre:
\begin{multicols}{2}
\begin{choices}
	\correctchoice{$0.0$ y $4.3$ mm/s}
	\wrongchoice{$-0.2$ y $2.52$ mm/s}
	\wrongchoice{$1.45$ y $3.44$ mm/s}
	\wrongchoice{$-0.5$ y $3.42$ mm/s}
\end{choices}
\end{multicols}
\end{question}
}
%%%%%%%%%%%%%%%%%%%%%%%%%%%%%%%%%%%%%%%%%%%%%%%%%%%%%%%%%%%%%%%%%%%%%%%%%%%%%
\element{test1}{
\begin{question}{p7}
Si retiramos de la visualización de resultados el suelo 1, la velocidad de flujo vertical varía entre:
\begin{multicols}{2}
\begin{choices}
	\correctchoice{$-4.5$ y $3.5$ mm/s}
	\wrongchoice{$0.0$ y $4.5$ mm/s}
	\wrongchoice{$2.5$ y $5.5$ mm/s}
	\wrongchoice{$-0.5$ y $1.5$ mm/s}
\end{choices}
\end{multicols}
\end{question}
}
%%%%%%%%%%%%%%%%%%%%%%%%%%%%%%%%%%%%%%%%%%%%%%%%%%%%%%%%%%%%%%%%%%%%%%%%%%%%%
\element{test1}{
\begin{question}{p8}
El valor de la velocidad de flujo vertical en el nodo superior de la salida del túnel vale:
\begin{multicols}{2}
\begin{choices}
	\correctchoice{$-2.73$ mm/s}
	\wrongchoice{$-1.26$ mm/s}
	\wrongchoice{$2.45$ mm/s}
	\wrongchoice{$4.55$ mm/s}
\end{choices}
\end{multicols}
\end{question}
}
%%%%%%%%%%%%%%%%%%%%%%%%%%%%%%%%%%%%%%%%%%%%%%%%%%%%%%%%%%%%%%%%%%%%%%%%%%%%%
\element{test1}{
	\begin{question}{p9}
		En un problema de conducción de calor, la condición natural de contorno:
		$$\bm{q} \cdot \bm{n} = \overline{q}, \textrm{ en } \partial_t \Omega$$
		se interpreta como:
		\begin{multicols}{2}
			\begin{choices}
				\correctchoice{El valor impuesto, de tipo escalar, que corresponde al
					flujo de la temperatura dirección normal al contorno $\partial_t \Omega$}
				\wrongchoice{El valor impuesto, de tipo vectorial, que corresponde al
					flujo de la temperatura en el contorno $\partial_t \Omega$}
				\wrongchoice{El valor impuesto, de tipo escalar, que corresponde al
					gradiente de la temperatura en dirección normal al contorno $\partial_t \Omega$}
				\wrongchoice{El valor impuesto, de tipo vectorial, que corresponde al
					gradiente de la temperatura en el contorno $\partial_t \Omega$}
			\end{choices}
		\end{multicols}
	\end{question}
}
%%%%%%%%%%%%%%%%%%%%%%%%%%%%%%%%%%%%%%%%%%%%%%%%%%%%%%%%%%%%%%%%%%%%%%%%%%%%%
\element{test1}{
	\begin{question}{p10}
		El concepto de compatibilidad, en el ámbito del método de los elementos finitos,
		establece:
		\begin{multicols}{2}
			\begin{choices}
				\correctchoice{Un requisito de continuidad de las funciones de forma,
					entre elementos adyacentes}
				\wrongchoice{Un requisito de continuidad de las funciones de forma,
					en el interior de los elementos}
				\wrongchoice{Que cuando el tamaño de los elementos tiende a $0$,
					la solución aproximada prácticamente coincide con la solución exacta del
					problema de contorno}
				\wrongchoice{Que el valor de la suma de las funciones de forma de un
					elemento es igual a $1$}
			\end{choices}
		\end{multicols}
	\end{question}
}

%%%%%%%%%%%%%%%%%%%%%%%%%%%%%%%%%%%%%%%%%%%%%

\scoringDefaultS{b=1,m=-1/(N-1)}

\onecopy{1}{    

%%% beginning of the test sheet header:    

\noindent{\large\bf Método de los Elementos Finitos  \hfill MUECYM \hfill TEST \# 1}

\vspace*{.5cm}
\begin{minipage}{.4\linewidth}
  \centering 10 oct 2022
\end{minipage}

\begin{center}\em
Tiempo: 60 minutos.

  %Está prohibido el uso de teléfonos móviles.

  %Se atribuirá puntuación negativa a las respuestas incorrectas.

\end{center}
\vspace{1ex}

%%% end of the header

\shufflegroup{test1}
\insertgroup{test1}

\AMCcleardoublepage    
%\clearpage

\AMCformBegin    

%%% beginning of the answer sheet header

\noindent\AMCcode{nummat}{2}\hspace*{\fill}
\begin{minipage}{.7\linewidth}
$\longleftarrow{}$ Escriba su número de matrícula marcando los dígitos
en los recuadros (con ceros a la izquierda si el número es de menos de dos dígitos) y el nombre y apellidos debajo.

\vspace{3ex}

\namefield{\fbox{
   \begin{minipage}{.9\linewidth}
     Apellidos, Nombre:

     \vspace*{.5cm}\dotfill
     \vspace*{1mm}
   \end{minipage}
 }}
\end{minipage}

\begin{center}
 \bf\em Debe dar las respuestas exclusivamente en esta hoja (las respuestas en las demás hojas no serán tenidas en cuenta).
\end{center}

%%% end of the answer sheet header


\AMCform    

\AMCcleardoublepage    

}  

\end{document}
