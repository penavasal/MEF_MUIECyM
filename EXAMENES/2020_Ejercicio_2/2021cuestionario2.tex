\documentclass[a4paper]{article}

\usepackage[utf8]{inputenc}    
\usepackage[T1]{fontenc}
\usepackage[spanish]{babel}

\usepackage{bm}
\usepackage{amsxtra} 
\usepackage{amssymb}% to get the \mathbb alphabet
\usepackage{amsmath}

\usepackage{moodle}

\spanishdecimal{.}

\begin{document}
	
\begin{quiz}{MEF2-20-21}

%%%%%%%%%%%%%%%%%%%%%%%%%%%%%%%%%%%%%%%%%%%%%%%%%%%%%%%%%%%%%%%%%%%%%%%%%%%%%
\begin{multi}{p1}
El caudal que se filtra, en r\'egimen estacionario, a la salida del t\'unel vale
aproximadamente:
\item* $1.1$ l/s
\item[fraction=-33.333] $0.6$ l/s
\item[fraction=-33.333] $9.2$ l/s
\item[fraction=-33.333] $4.7$ l/s
\item[fraction=0] Respuesta en blanco
\end{multi}
%%%%%%%%%%%%%%%%%%%%%%%%%%%%%%%%%%%%%%%%%%%%%%%%%%%%%%%%%%%%%%%%%%%%%%%%%%%%%
\begin{multi}{p21}
El valor m\'inimo de la altura piezom\'etrica se alcanza en:
	\item* En la salida del t\'unel
\item[fraction=-33.333] En un punto de sustrato rocoso situado bajo el r\'io
\item[fraction=-33.333] En la zona de contacto del agua con el medio permeable
\item[fraction=-33.333] En el punto inferior de alguna de las pantallas
impermeables
\item[fraction=0] Respuesta en blanco
\end{multi}
%%%%%%%%%%%%%%%%%%%%%%%%%%%%%%%%%%%%%%%%%%%%%%%%%%%%%%%%%%%%%%%%%%%%%%%%%%%%%
\begin{multi}{p22}
El valor m\'aximo de la altura piezom\'etrica se alcanza en:
\item* En el medio permeable situado bajo el r\'io
\item[fraction=-33.333] En la salida del t\'unel
\item[fraction=-33.333] En la parte superior de la entrada del t\'unel
\item[fraction=-33.333] En la parte inferior de la entrada del t\'unel
\item[fraction=0] Respuesta en blanco
\end{multi}
%%%%%%%%%%%%%%%%%%%%%%%%%%%%%%%%%%%%%%%%%%%%%%%%%%%%%%%%%%%%%%%%%%%%%%%%%%%%%
\begin{multi}{p31}
Los valores de la velocidad horizontal est\'an
comprendidos aproximadamente entre:
\item* $0$ y $2.1$ mm/s
\item[fraction=-33.333] $-0.5$ y $1.5$ mm/s
\item[fraction=-33.333] $1.0$ y $3.0$ mm/s
\item[fraction=-33.333] $2.0$ y $3.5$ mm/s
\item[fraction=0] Respuesta en blanco
\end{multi}
%%%%%%%%%%%%%%%%%%%%%%%%%%%%%%%%%%%%%%%%%%%%%%%%%%%%%%%%%%%%%%%%%%%%%%%%%%%%%
\begin{multi}{p32}
Los valores de la velocidad en direcci\'on vertical est\'an
comprendidos aproximadamente entre:
\item* $-1.2$ y $1.2$ mm/s
\item[fraction=-33.333] $-1.2$ y $1.2$ cm/s
\item[fraction=-33.333] $0$ y $-1.5$ mm/s
\item[fraction=-33.333] $-0.1$ y $0.1$ m/s
\item[fraction=0] Respuesta en blanco
\end{multi}
%%%%%%%%%%%%%%%%%%%%%%%%%%%%%%%%%%%%%%%%%%%%%%%%%%%%%%%%%%%%%%%%%%%%%%%%%%%%%
\begin{multi}{p4}
Los valores m\'as altos de la velocidad en direcci\'on horizontal
se obtienen:
\item* En la salida del t\'unel
\item[fraction=-33.333] Bajo alguna de las pantallas impermeable
\item[fraction=-33.333] En el contacto del medio permeable con el agua del r\'io
\item[fraction=-33.333] Ninguna de las respuestas anteriores es correcta
\item[fraction=0] Respuesta en blanco
\end{multi}
%%%%%%%%%%%%%%%%%%%%%%%%%%%%%%%%%%%%%%%%%%%%%%%%%%%%%%%%%%%%%%%%%%%%%%%%%%%%%
\begin{multi}{p51}
El n\'umero de grados de libertad del modelo es:
\item* $1166$
\item[fraction=-33.333] $534$
\item[fraction=-33.333] $268$
\item[fraction=-33.333] $3524$
\item[fraction=0] Respuesta en blanco
\end{multi}
%%%%%%%%%%%%%%%%%%%%%%%%%%%%%%%%%%%%%%%%%%%%%%%%%%%%%%%%%%%%%%%%%%%%%%%%%%%%%
\begin{multi}{p52}
	El n\'umero de nodos (sin descontar los que se eliminan con el comando
	{\tt tie}) es:
\item* $1227$
\item[fraction=-33.333] $1125$
\item[fraction=-33.333] $1374$
\item[fraction=-33.333] $980$
\item[fraction=0] Respuesta en blanco
\end{multi}
%%%%%%%%%%%%%%%%%%%%%%%%%%%%%%%%%%%%%%%%%%%%%%%%%%%%%%%%%%%%%%%%%%%%%%%%%%%%%
\begin{multi}{p61}
	El valor m\'aximo de la altura piezom\'etrica $u$ en la entrada del t\'unel (secci\'on $AA'$) vale:
	\item* $7.3$ m
	\item[fraction=-33.333] $6.9$ m
	\item[fraction=-33.333] $9$ m
	\item[fraction=-33.333] $0.2$ m
	\item[fraction=0] Respuesta en blanco
\end{multi}
%%%%%%%%%%%%%%%%%%%%%%%%%%%%%%%%%%%%%%%%%%%%%%%%%%%%%%%%%%%%%%%%%%%%%%%%%%%%%
\begin{multi}{p62}
	El valor m\'inimo de la altura piezom\'etrica $u$ en la entrada del t\'unel (secci\'on $AA'$) vale:
	\item* $6.9$ m
	\item[fraction=-33.333] $7.3$ m
	\item[fraction=-33.333] $9$ m
	\item[fraction=-33.333] $0.2$ m
	\item[fraction=0] Respuesta en blanco
\end{multi}
%%%%%%%%%%%%%%%%%%%%%%%%%%%%%%%%%%%%%%%%%%%%%%%%%%%%%%%%%%%%%%%%%%%%%%%%%%%%%
\begin{multi}{p71}
	Tomando como peso espec\'ifico del agua $\gamma=10000$ N/m$^3$, la presi\'on en el punto
	m\'as alto de la entrada del t\'unel (secci\'on AA') vale:
	\item* $63$ kPa
	\item[fraction=-33.333] $7.3$ kPa
	\item[fraction=-33.333] $6.3$ kPa
	\item[fraction=-33.333] $73$ kPa
	\item[fraction=0] Respuesta en blanco
\end{multi}
%%%%%%%%%%%%%%%%%%%%%%%%%%%%%%%%%%%%%%%%%%%%%%%%%%%%%%%%%%%%%%%%%%%%%%%%%%%%%
\begin{multi}{p72}
	Tomando como peso espec\'ifico del agua $\gamma=10000$ N/m$^3$, la presi\'on en el punto
	m\'as bajo de la entrada del t\'unel (secci\'on AA') vale:
	\item* $69$ kPa
	\item[fraction=-33.333] $6.9$ kPa
	\item[fraction=-33.333] $63$ kPa
	\item[fraction=-33.333] $6.3$ kPa
	\item[fraction=0] Respuesta en blanco
\end{multi}
%%%%%%%%%%%%%%%%%%%%%%%%%%%%%%%%%%%%%%%%%%%%%%%%%%%%%%%%%%%%%%%%%%%%%%%%%%%%%
\begin{multi}{p8}
	En un modelo de conducci\'on de calor en el que existe un plano de simetr\'ia, la
	condici\'on de contorno a imponer en los nodos de dicho plano es:
	\item* Flujo en direcci\'on normal al plano igual a $0$
	\item[fraction=-33.333] Todas las componentes del vector flujo igual a $0$
	\item[fraction=-33.333] Flujo en direcci\'on paralela al plano igual a $0$
	\item[fraction=-33.333] Las dem\'as respuestas son incorrectas
	\item[fraction=0] Respuesta en blanco
\end{multi}
%%%%%%%%%%%%%%%%%%%%%%%%%%%%%%%%%%%%%%%%%%%%%%%%%%%%%%%%%%%%%%%%%%%%%%%%%%%%%
\begin{multi}{p9}
	En la formulaci\'on d\'ebil de un problema de contorno, la derivada del
campo inc\'ognita que aparece en dicha formulaci\'on es:
	\item* Es de un orden menor que su derivada en la formulaci\'on fuerte
	\item[fraction=-33.333] Es de un orden mayor que su derivada en la formulaci\'on fuerte
	\item[fraction=-33.333] Es del mismo orden que su derivada en la formulaci\'on fuerte
	\item[fraction=-33.333] Depende del tama\~{n}o de la malla de elementos finitos
	\item[fraction=0] Respuesta en blanco
\end{multi}
%%%%%%%%%%%%%%%%%%%%%%%%%%%%%%%%%%%%%%%%%%%%%%%%%%%%%%%%%%%%%%%%%%%%%%%%%%%%%
\begin{multi}{p101}
	El valor m\'aximo de la velocidad horizontal en la salida del t\'unel (secci\'on $BB'$) vale:
	\item* $2.0$ mm/s
	\item[fraction=-33.333] $0.4$ mm/s
	\item[fraction=-33.333] $1.0$ mm/s
	\item[fraction=-33.333] $3.1$ mm/s
	\item[fraction=0] Respuesta en blanco
\end{multi}
%%%%%%%%%%%%%%%%%%%%%%%%%%%%%%%%%%%%%%%%%%%%%%%%%%%%%%%%%%%%%%%%%%%%%%%%%%%%%
\begin{multi}{p102}
	El valor m\'inimo de la velocidad horizontal en la salida del t\'unel (secci\'on $BB'$) vale:
	\item* $0.4$ mm/s
	\item[fraction=-33.333] $2.0$ mm/s
	\item[fraction=-33.333] $1.1$ mm/s
	\item[fraction=-33.333] $3.4$ mm/s
	\item[fraction=0] Respuesta en blanco
\end{multi}
%%%%%%%%%%%%%%%%%%%%%%%%%%%%%%%%%%%%%%%%%%%%%%%%%%%%%%%%%%%%%%%%%%%%%%%%%%%%%




\end{quiz}

\end{document}
