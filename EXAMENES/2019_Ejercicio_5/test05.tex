\documentclass[a4paper]{article}

\usepackage[utf8x]{inputenc}    
\usepackage[T1]{fontenc}
\usepackage[spanish]{babel}
\usepackage{multicol}

\usepackage{wrapfig}
\usepackage{graphicx}

\usepackage{bm}
\usepackage{amsxtra} 
\usepackage{amssymb}% to get the \mathbb alphabet
\usepackage{amsmath}

\usepackage[box,completemulti,separateanswersheet]{automultiplechoice}    
\def\AMCformQuestion#1{\vspace{\AMCformVSpace}\par {\sc Pregunta #1:} }    
\def\AMCbeginQuestion#1#2{\par\noindent{\bf Pregunta #1}#2\hspace*{1em}}
\def\AMCcleardoublepage{\ifodd\thepage\clearpage\mbox{}\fi\clearpage}

\begin{document}

\AMCrandomseed{1237893}

%%%%%%%%%%%%%%%%%%%%%%%%%%%%%%%%%%%%%%%%%%%%%%%%%%%%%%%%%%%%%%%%%%%%%%%%%%%%%
\element{test1}{
\begin{question}{p1}
En la formulación de elementos finitos con integración selectiva-reducida:
\begin{choices}
	\correctchoice{La parte de la matriz de rigidez correspondiente al
comportamiento volumétrico se subintegra, y en el resto se realiza una
integración completa.}
	\wrongchoice{En las zonas de la malla que se deforman por flexión
las matrices de rigidez elementales se subintegran.}
	\wrongchoice{En una parte reducida de la malla, las matrices de
rigidez elementales se integran con una cuadratura selectiva.}
	\wrongchoice{Ninguna de las respuestas anteriores es correcta.}
\end{choices}
\end{question}
}
%%%%%%%%%%%%%%%%%%%%%%%%%%%%%%%%%%%%%%%%%%%%%%%%%%%%%%%%%%%%%%%%%%%%%%%%%%%%%
\element{test1}{
\begin{question}{p2}
La formulación $\overline{\bm{B}}$ de elementos finitos:
\begin{multicols}{2}
\begin{choices}
	\correctchoice{Se basa en modificar la matriz de interpolación de
deformaciones estándar}
	\wrongchoice{Se basa en remplazar la matriz de interpolación de
deformaciones por un vector de deformaciones libre de bloqueo}
	\wrongchoice{Evita el problema de bloqueo por flexión pero no el
bloqueo por incompresibilidad}
	\wrongchoice{Evita el problema de bloqueo, pero no pasa la prueba de
la parcela}
\end{choices}
\end{multicols}
\end{question}
}
%%%%%%%%%%%%%%%%%%%%%%%%%%%%%%%%%%%%%%%%%%%%%%%%%%%%%%%%%%%%%%%%%%%%%%%%%%%%%
\element{test1}{
\begin{question}{p3}
El fenómeno del ``hourglassing'':
\begin{multicols}{2}
\begin{choices}
	\correctchoice{Puede aparecer cuando las matrices de rigidez
se calculan con una regla de integración reducida}
	\wrongchoice{Consiste en el cambio del signo de la presión en elementos
adyacentes, denominándose también problema del ``tablero de damas'' o 
``checkerboard''}
	\wrongchoice{Puede aparecer en problemas de incompresibilidad,
con independecia del orden que tenga la regla de integración (o cuadratura)
empleada}
	\wrongchoice{Sólo puede aparecer cuando el material es incompresible}
\end{choices}
\end{multicols}
\end{question}
}
%%%%%%%%%%%%%%%%%%%%%%%%%%%%%%%%%%%%%%%%%%%%%%%%%%%%%%%%%%%%%%%%%%%%%%%%%%%%%
\element{test1}{
\begin{question}{p4}
El desplazamiento bajo la carga puntual en dirección de la misma, obtenido con los elementos
isoparamétricos, vale
\begin{multicols}{2}
\begin{choices}
	\correctchoice{$5.0 \cdot 10^{-2}$}
	\wrongchoice{$3.1 \cdot 10^{-2}$}
	\wrongchoice{$2.5 \cdot 10^{-2}$}
	\wrongchoice{$9.7 \cdot 10^{-2}$}
\end{choices}
\end{multicols}
\end{question}
}
%%%%%%%%%%%%%%%%%%%%%%%%%%%%%%%%%%%%%%%%%%%%%%%%%%%%%%%%%%%%%%%%%%%%%%%%%%%%%
\element{test1}{
\begin{question}{p5}
El desplazamiento bajo la carga puntual en dirección de la misma, obtenido con los elementos
mixtos, vale
\begin{multicols}{2}
\begin{choices}
	\correctchoice{$5.4 \cdot 10^{-2}$}
	\wrongchoice{$3.5 \cdot 10^{-2}$}
	\wrongchoice{$2.9 \cdot 10^{-2}$}
	\wrongchoice{$1.1 \cdot 10^{-1}$}
\end{choices}
\end{multicols}
\end{question}
}
%%%%%%%%%%%%%%%%%%%%%%%%%%%%%%%%%%%%%%%%%%%%%%%%%%%%%%%%%%%%%%%%%%%%%%%%%%%%%
\element{test1}{
\begin{question}{p6}
El desplazamiento bajo la carga puntual, obtenido con los elementos
de modos incompatibles, vale
\begin{multicols}{2}
\begin{choices}
        \correctchoice{$1.5 \cdot 10^{-1}$}
        \wrongchoice{$9.2 \cdot 10^{-2}$}
        \wrongchoice{$3.2 \cdot 10^{-1}$}
        \wrongchoice{$8.4 \cdot 10^{-2}$}
\end{choices}
\end{multicols}
\end{question}
}
%%%%%%%%%%%%%%%%%%%%%%%%%%%%%%%%%%%%%%%%%%%%%%%%%%%%%%%%%%%%%%%%%%%%%%%%%%%%%
\element{test1}{
\begin{question}{p7}
Al dibujar los contornos de la tensión de Von Mises,
el valor máximo se obtiene:
\begin{multicols}{2}
\begin{choices}
	\correctchoice{Bajo la carga puntual}
	\wrongchoice{En el punto más alto del empotramiento}
	\wrongchoice{En el punto más bajo del empotramiento}
	\wrongchoice{En algún punto de la generatriz que es un eje de simetría
y no contiene a la carga puntual}
\end{choices}
\end{multicols}
\end{question}
}
%%%%%%%%%%%%%%%%%%%%%%%%%%%%%%%%%%%%%%%%%%%%%%%%%%%%%%%%%%%%%%%%%%%%%%%%%%%%%
\element{test1}{
\begin{question}{p8}
El valor máximo de la tensión de Von Mises, calculado con los elementos
isoparamétricos, vale:
\begin{multicols}{2}
\begin{choices}
	\correctchoice{$3.9 \cdot 10^{-2}$}
	\wrongchoice{$4.7 \cdot 10^{-2}$}
	\wrongchoice{$5.1 \cdot 10^{-2}$}
	\wrongchoice{$1.6 \cdot 10^{-2}$}
\end{choices}
\end{multicols}
\end{question}
}
%%%%%%%%%%%%%%%%%%%%%%%%%%%%%%%%%%%%%%%%%%%%%%%%%%%%%%%%%%%%%%%%%%%%%%%%%%%%%
\element{test1}{
\begin{question}{p9}
El valor máximo de la tensión de Von Mises, calculado con los elementos
mixtos, vale:
\begin{multicols}{2}
\begin{choices}
	\correctchoice{$4.2 \cdot 10^{-2}$}
	\wrongchoice{$5.1 \cdot 10^{-2}$}
	\wrongchoice{$5.6 \cdot 10^{-2}$}
	\wrongchoice{$1.9 \cdot 10^{-2}$}
\end{choices}
\end{multicols}
\end{question}
}
%%%%%%%%%%%%%%%%%%%%%%%%%%%%%%%%%%%%%%%%%%%%%%%%%%%%%%%%%%%%%%%%%%%%%%%%%%%%%
\element{test1}{
\begin{question}{p10}
El valor máximo de la tensión de Von Mises, calculado con los elementos
incompatibles, vale:
\begin{multicols}{2}
\begin{choices}
	\correctchoice{$5.0 \cdot 10^{-2}$}
	\wrongchoice{$4.7 \cdot 10^{-2}$}
	\wrongchoice{$5.6 \cdot 10^{-2}$}
	\wrongchoice{$2.7 \cdot 10^{-2}$}
\end{choices}
\end{multicols}
\end{question}
}
%%%%%%%%%%%%%%%%%%%%%%%%%%%%%%%%%%%%%%%%%%%%%

\scoringDefaultS{b=1,m=-1/(N-1)}

\onecopy{1}{    

%%% beginning of the test sheet header:    

\noindent{\large\bf Método de los Elementos Finitos  \hfill MUECYM \hfill TEST \# 1}

\vspace*{.5cm}
\begin{minipage}{.4\linewidth}
  \centering 2 oct 2015
\end{minipage}

\begin{center}\em
Tiempo: 60 minutos.

  %Está prohibido el uso de teléfonos móviles.

  %Se atribuirá puntuación negativa a las respuestas incorrectas.

\end{center}
\vspace{1ex}

%%% end of the header

\shufflegroup{test1}
\insertgroup{test1}

\AMCcleardoublepage    
%\clearpage

\AMCformBegin    

%%% beginning of the answer sheet header

\noindent\AMCcode{nummat}{4}\hspace*{\fill}
\begin{minipage}{.7\linewidth}
$\longleftarrow{}$ Escriba su número de matrícula marcando los dígitos
en los recuadros (con ceros a la izquierda si el número es de menos de tres dígitos) y el nombre y apellidos debajo.

\vspace{3ex}

\namefield{\fbox{
   \begin{minipage}{.9\linewidth}
     Apellidos, Nombre:

     \vspace*{.5cm}\dotfill
     \vspace*{1mm}
   \end{minipage}
 }}
\end{minipage}

\begin{center}
 \bf\em Debe dar las respuestas exclusivamente en esta hoja (las respuestas en las demás hojas no serán tenidas en cuenta).
\end{center}

%%% end of the answer sheet header


\AMCform    

\AMCcleardoublepage    

}  

\end{document}
