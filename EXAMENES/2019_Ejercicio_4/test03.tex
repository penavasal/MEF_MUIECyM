\documentclass[a4paper]{article}

\usepackage[utf8x]{inputenc}    
\usepackage[T1]{fontenc}
\usepackage[spanish]{babel}
\usepackage{multicol}

\usepackage{wrapfig}
\usepackage{graphicx}

\usepackage{bm}
\usepackage{amsxtra} 
\usepackage{amssymb}% to get the \mathbb alphabet
\usepackage{amsmath}

\usepackage[box,completemulti,separateanswersheet]{automultiplechoice}    
\def\AMCformQuestion#1{\vspace{\AMCformVSpace}\par {\sc Pregunta #1:} }    
\def\AMCbeginQuestion#1#2{\par\noindent{\bf Pregunta #1}#2\hspace*{1em}}
\def\AMCcleardoublepage{\ifodd\thepage\clearpage\mbox{}\fi\clearpage}

\begin{document}

\AMCrandomseed{1237893}

%%%%%%%%%%%%%%%%%%%%%%%%%%%%%%%%%%%%%%%%%%%%%%%%%%%%%%%%%%%%%%%%%%%%%%%%%%%%%
\element{test1}{
\begin{question}{p1}
La flecha en el punto en el que está aplicada la carga, calculada con la Resistencia de Materiales, vale aproximadamente:
\begin{multicols}{2}
\begin{choices}
	\correctchoice{$2.0$ mm}
	\wrongchoice{$0.7$ mm}
	\wrongchoice{$2.6$ mm}
	\wrongchoice{$1.4$ mm}
\end{choices}
\end{multicols}
\end{question}
}
%%%%%%%%%%%%%%%%%%%%%%%%%%%%%%%%%%%%%%%%%%%%%%%%%%%%%%%%%%%%%%%%%%%%%%%%%%%%%
\element{test1}{
\begin{question}{p2}
En un problema plano de elasticidad lineal con la hipótesis de deformación plana,
en general:
\begin{multicols}{2}
\begin{choices}
	\correctchoice{Las deformaciones perpendiculares al plano del sólido son
nulas}
	\wrongchoice{Las tensiones perpendiculares al plano del sólido son 
nulas}
	\wrongchoice{Las tensiones y las deformaciones perpendiculares al plano
del sólido son nulas}
	\wrongchoice{Ninguna de las otras respuestas es correcta}
\end{choices}
\end{multicols}
\end{question}
}
%%%%%%%%%%%%%%%%%%%%%%%%%%%%%%%%%%%%%%%%%%%%%%%%%%%%%%%%%%%%%%%%%%%%%%%%%%%%%
\element{test1}{
\begin{question}{p3}
La formulación débil del problema del sólido elastico se interpreta como:
\begin{multicols}{2}
\begin{choices}
	\correctchoice{El principio de los trabajos virtuales}
	\wrongchoice{El equilibrio de fuerzas en cada punto del sólido}
	\wrongchoice{No tiene interpretación física}
	\wrongchoice{Un requisito de convergencia del método de elementos
                     finitos}
\end{choices}
\end{multicols}
\end{question}
}
%%%%%%%%%%%%%%%%%%%%%%%%%%%%%%%%%%%%%%%%%%%%%%%%%%%%%%%%%%%%%%%%%%%%%%%%%%%%%
\element{test1}{
\begin{question}{p4}
Los elementos isoparamétricos se caracterizan por:
\begin{choices}
	\correctchoice{Las funciones de interpolación de las coordenadas
son las mismas que las funciones de interpolación de los desplazamientos}
	\wrongchoice{Las integrales que permiten calcular la matriz de
rigidez y el vector de fuerzas nodales han de evaluarse mediante la
cuadratura de Gauss}
	\wrongchoice{La dimensión de la matriz de rigidez del elemento
es $n \times n$, siendo $n$ el número de nodos de dicho elemento}
	\wrongchoice{Los nodos del elemento necesariamente deben estar
situados en los lados del mismo}
\end{choices}
\end{question}
}
%%%%%%%%%%%%%%%%%%%%%%%%%%%%%%%%%%%%%%%%%%%%%%%%%%%%%%%%%%%%%%%%%%%%%%%%%%%%%
\element{test1}{
\begin{question}{p5}
La flecha en el punto en el que está aplicada la carga, calculada con cuadriláteros de $4$ nodos, vale aproximadamente:
\begin{multicols}{2}
\begin{choices}
	\correctchoice{Un $57 \%$ de la flecha teórica}
	\wrongchoice{Un $87 \%$ de la flecha teórica}
	\wrongchoice{Un $73 \%$ de la flecha teórica}
	\wrongchoice{Un $36 \%$ de la flecha teórica}
\end{choices}
\end{multicols}
\end{question}
}
%%%%%%%%%%%%%%%%%%%%%%%%%%%%%%%%%%%%%%%%%%%%%%%%%%%%%%%%%%%%%%%%%%%%%%%%%%%%%
\element{test1}{
\begin{question}{p6}
El número de grados de libertad del modelo con elementos cuadriláteros es:
\begin{multicols}{2}
\begin{choices}
	\correctchoice{57}
	\wrongchoice{43}
	\wrongchoice{64}
	\wrongchoice{72}
\end{choices}
\end{multicols}
\end{question}
}
%%%%%%%%%%%%%%%%%%%%%%%%%%%%%%%%%%%%%%%%%%%%%%%%%%%%%%%%%%%%%%%%%%%%%%%%%%%%%
\element{test1}{
\begin{question}{p7}
La flecha en el punto en el que está aplicada la carga, calculada con triángulos de $3$ nodos, vale aproximadamente:
\begin{multicols}{2}
\begin{choices}
	\correctchoice{Un $28 \%$ de la flecha teórica}
	\wrongchoice{Un $36 \%$ de la flecha teórica}
	\wrongchoice{Un $56 \%$ de la flecha teórica}
	\wrongchoice{Un $44 \%$ de la flecha teórica}
\end{choices}
\end{multicols}
\end{question}
}
%%%%%%%%%%%%%%%%%%%%%%%%%%%%%%%%%%%%%%%%%%%%%%%%%%%%%%%%%%%%%%%%%%%%%%%%%%%%%
\element{test1}{
\begin{question}{p8}
El número de grados de libertad del modelo con elementos triangulares es:
\begin{multicols}{2}
\begin{choices}
	\correctchoice{57}
	\wrongchoice{43}
	\wrongchoice{64}
	\wrongchoice{108}
\end{choices}
\end{multicols}
\end{question}
}
%%%%%%%%%%%%%%%%%%%%%%%%%%%%%%%%%%%%%%%%%%%%%%%%%%%%%%%%%%%%%%%%%%%%%%%%%%%%%
\element{test1}{
\begin{question}{p9}
La reacción vertical en el apoyo derecho:
\begin{multicols}{2}
\begin{choices}
	\correctchoice{Vale lo mismo en ambas mallas}
	\wrongchoice{Es mayor en la malla de elementos triangulares}
	\wrongchoice{Es mayor en la malla de elementos cuadriláteros}
	\wrongchoice{Ninguna de las otras respuestas es correcta}
\end{choices}
\end{multicols}
\end{question}
}
%%%%%%%%%%%%%%%%%%%%%%%%%%%%%%%%%%%%%%%%%%%%%%%%%%%%%%%%%%%%%%%%%%%%%%%%%%%%%
\element{test1}{
\begin{question}{p10}
La suma de las reacciones verticales en los apoyos coincide con la
carga aplicada con un error inferior al $0.1$\%:
\begin{multicols}{2}
\begin{choices}
	\correctchoice{Con todos los elementos}
	\wrongchoice{Sólo con elementos cuadriláteros}
	\wrongchoice{Sólo con elementos triangulares}
	\wrongchoice{En ningún caso}
\end{choices}
\end{multicols}
\end{question}
}
%%%%%%%%%%%%%%%%%%%%%%%%%%%%%%%%%%%%%%%%%%%%%

\scoringDefaultS{b=1,m=-1/(N-1)}

\onecopy{1}{    

%%% beginning of the test sheet header:    

\noindent{\large\bf Método de los Elementos Finitos  \hfill MUECYM \hfill TEST \# 3}

\vspace*{.5cm}
\begin{minipage}{.4\linewidth}
  \centering 31 oct 2019
\end{minipage}

\begin{center}\em
Tiempo: 60 minutos.

  %Está prohibido el uso de teléfonos móviles.

  %Se atribuirá puntuación negativa a las respuestas incorrectas.

\end{center}
\vspace{1ex}

%%% end of the header

\shufflegroup{test1}
\insertgroup{test1}

\AMCcleardoublepage    
%\clearpage

\AMCformBegin    

%%% beginning of the answer sheet header

\noindent\AMCcode{nummat}{2}\hspace*{\fill}
\begin{minipage}{.7\linewidth}
$\longleftarrow{}$ Escriba su número de matrícula marcando los dígitos
en los recuadros (con ceros a la izquierda si el número es de menos de dos dígitos) y el nombre y apellidos debajo.

\vspace{3ex}

\namefield{\fbox{
   \begin{minipage}{.9\linewidth}
     Apellidos, Nombre:

     \vspace*{.5cm}\dotfill
     \vspace*{1mm}
   \end{minipage}
 }}
\end{minipage}

\begin{center}
 \bf\em Debe dar las respuestas exclusivamente en esta hoja (las respuestas en las demás hojas no serán tenidas en cuenta).
\end{center}

%%% end of the answer sheet header


\AMCform    

\AMCcleardoublepage    

}  

\end{document}
