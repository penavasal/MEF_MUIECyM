\documentclass[a4paper]{article}

\usepackage[utf8]{inputenc}    
\usepackage[T1]{fontenc}
\usepackage[spanish]{babel}

\usepackage{bm}
\usepackage{amsxtra} 
\usepackage{amssymb}% to get the \mathbb alphabet
\usepackage{amsmath}

\usepackage{moodle}

\spanishdecimal{.}

\begin{document}

\begin{quiz}{MEF3-20-21}

%%%%%%%%%%%%%%%%%%%%%%%%%%%%%%%%%%%%%%%%%%%%%%%%%%%%%%%%%%%%%%%%%%%%%%%%%%%%%
\begin{multi}{p1}
La flecha en el extremo libre, calculada con la Resistencia de Materiales,
vale aproximadamente:
	\item* $3.0$ mm
	\item[fraction=-33.333] $2.0$ mm
	\item[fraction=-33.333] $2.5$ mm
	\item[fraction=-33.333] $3.5$ mm
	\item[fraction=0] Respuesta en blanco
\end{multi}
%%%%%%%%%%%%%%%%%%%%%%%%%%%%%%%%%%%%%%%%%%%%%%%%%%%%%%%%%%%%%%%%%%%%%%%%%%%%%
\begin{multi}{p2}
	El modelo con menor n\'umero de grados de libertad es el de:
	\item* Cuadril\'ateros de $8$ nodos
	\item[fraction=-33.333]  Tri\'angulos de $3$ nodos
	\item[fraction=-33.333] Tri\'angulos de $6$ nodos
	\item[fraction=-33.333] Cuadril\'ateros de $4$ nodos
	\item[fraction=0] Respuesta en blanco
\end{multi}
%%%%%%%%%%%%%%%%%%%%%%%%%%%%%%%%%%%%%%%%%%%%%%%%%%%%%%%%%%%%%%%%%%%%%%%%%%%%%
\begin{multi}{p3}
En la malla de cuadril\'ateros de $4$ nodos, la flecha en el extremo
en el que est\'a aplicada la carga vale:
\item* $1.8$ mm
\item[fraction=-33.333] $0.5$ mm
\item[fraction=-33.333] $1.5$ mm
\item[fraction=-33.333] $2.2$ mm
\item[fraction=0] Respuesta en blanco
\end{multi}
%%%%%%%%%%%%%%%%%%%%%%%%%%%%%%%%%%%%%%%%%%%%%%%%%%%%%%%%%%%%%%%%%%%%%%%%%%%%%
\begin{multi}{p4}
En la malla de cuadril\'ateros de $8$ nodos, la flecha en el extremo
en el que est\'a aplicada la carga vale:
\item* $3.0$ mm
\item[fraction=-33.333] $2.5$ mm
\item[fraction=-33.333] $3.5$ mm
\item[fraction=-33.333] $0.9$ mm
\item[fraction=0] Respuesta en blanco
\end{multi}
%%%%%%%%%%%%%%%%%%%%%%%%%%%%%%%%%%%%%%%%%%%%%%%%%%%%%%%%%%%%%%%%%%%%%%%%%%%%%
\begin{multi}{p5}
En la malla de tri\'angulos de $6$ nodos, la flecha en el extremo
en el que est\'a aplicada la carga vale:
\item* $3.0$ mm
\item[fraction=-33.333] $0.9$ mm
\item[fraction=-33.333] $2.8$ mm
\item[fraction=-33.333] $3.2$ mm
\item[fraction=0] Respuesta en blanco
\end{multi}
%%%%%%%%%%%%%%%%%%%%%%%%%%%%%%%%%%%%%%%%%%%%%%%%%%%%%%%%%%%%%%%%%%%%%%%%%%%%%
\begin{multi}{p6}
En la malla de tri\'angulos de $3$ nodos, la flecha en el extremo
en el que est\'a aplicada la carga vale:
\item* $0.9$ mm
\item[fraction=-33.333] $3.0$ mm
\item[fraction=-33.333] $2.0$ mm
\item[fraction=-33.333] $1.3$ mm
\item[fraction=0] Respuesta en blanco
\end{multi}
%%%%%%%%%%%%%%%%%%%%%%%%%%%%%%%%%%%%%%%%%%%%%%%%%%%%%%%%%%%%%%%%%%%%%%%%%%%%%
\begin{multi}{p7}
	El valor de la reacci\'on horizontal en la esquina inferior izquierda, calculada con
	la malla de cuadril\'ateros de $4$ nodos vale:
	\item* $115.8$ kN
	\item[fraction=-33.333]  $75.4$ kN
	\item[fraction=-33.333] $215.4$ kN
	\item[fraction=-33.333] $165.4$ kN
	\item[fraction=0] Respuesta en blanco
\end{multi}
%%%%%%%%%%%%%%%%%%%%%%%%%%%%%%%%%%%%%%%%%%%%%%%%%%%%%%%%%%%%%%%%%%%%%%%%%%%%%
\begin{multi}{p8}
	El valor de la reacci\'on horizontal en la esquina inferior izquierda, calculada con
	la malla de cuadril\'ateros de $8$ nodos vale:
	\item* $117.1$ kN
	\item[fraction=-33.333] $73.2$ kN
	\item[fraction=-33.333] $211.6$ kN
	\item[fraction=-33.333] $163.3$ kN
	\item[fraction=0] Respuesta en blanco
\end{multi}
%%%%%%%%%%%%%%%%%%%%%%%%%%%%%%%%%%%%%%%%%%%%%%%%%%%%%%%%%%%%%%%%%%%%%%%%%%%%%
\begin{multi}{p9}
	El valor de la reacci\'on horizontal en la esquina inferior izquierda, calculada con
	la malla de tri\'angulos de $3$ nodos vale:
	\item* $118.3$ kN
	\item[fraction=-33.333] $77.2$ kN
	\item[fraction=-33.333] $217.7$ kN
	\item[fraction=-33.333] $167.2$ kN
	\item[fraction=0] Respuesta en blanco
\end{multi}
%%%%%%%%%%%%%%%%%%%%%%%%%%%%%%%%%%%%%%%%%%%%%%%%%%%%%%%%%%%%%%%%%%%%%%%%%%%%%
\begin{multi}{p10}
La suma de las reacciones verticales en el empotramiento coincide con la
carga aplicada el extremo con un error inferior al $0.1$\%:
	\item* Con todos los elementos
	\item[fraction=-33.333] S\'olo con el cuadril\'atero de $8$ nodos
	\item[fraction=-33.333] S\'olo con el tri\'angulo de $6$ nodos
	\item[fraction=-33.333] S\'olo con el cuadril\'atero de $8$ nodos y con el tri\'angulo
                     de $6$ nodos
                     \item[fraction=0] Respuesta en blanco
\end{multi}
%%%%%%%%%%%%%%%%%%%%%%%%%%%%%%%%%%%%%%%%%%%%%%%%%%%%%%%%%%%%%%%%%%%%%%%%%%%%%


\end{quiz}

\end{document}
