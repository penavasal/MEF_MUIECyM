\documentclass[a4paper]{article}

\usepackage[utf8x]{inputenc}    
\usepackage[T1]{fontenc}
\usepackage[spanish]{babel}
\usepackage{multicol}

\usepackage{wrapfig}
\usepackage{graphicx}

\usepackage{bm}
\usepackage{amsxtra} 
\usepackage{amssymb}% to get the \mathbb alphabet
\usepackage{amsmath}

\usepackage[box,completemulti,separateanswersheet]{automultiplechoice}    
\def\AMCformQuestion#1{\vspace{\AMCformVSpace}\par {\sc Pregunta #1:} }    
\def\AMCbeginQuestion#1#2{\par\noindent{\bf Pregunta #1}#2\hspace*{1em}}
\def\AMCcleardoublepage{\ifodd\thepage\clearpage\mbox{}\fi\clearpage}

\begin{document}

\AMCrandomseed{1237893}

%%%%%%%%%%%%%%%%%%%%%%%%%%%%%%%%%%%%%%%%%%%%%%%%%%%%%%%%%%%%%%%%%%%%%%%%%%%%%
\element{test1}{
\begin{question}{p1}
El caudal que se filtra, en régimen estacionario, a la derecha de la pantalla impermeable (contorno $CD$) vale:
\begin{multicols}{2}
\begin{choices}
	\correctchoice{$7.67$ l/s}
	\wrongchoice{$4.21$ l/s}
	\wrongchoice{$1.56$ l/s}
	\wrongchoice{$1.01$ l/s}
\end{choices}
\end{multicols}
\end{question}
}
%%%%%%%%%%%%%%%%%%%%%%%%%%%%%%%%%%%%%%%%%%%%%%%%%%%%%%%%%%%%%%%%%%%%%%%%%%%%%
\element{test1}{
\begin{question}{p2}
El valor máximo de la altura piezométrica en el contorno $FG$ vale:
\begin{multicols}{2}
\begin{choices}
	\correctchoice{$15,93$ m}
	\wrongchoice{$18,01$ m}
	\wrongchoice{$13,54$ m}
	\wrongchoice{$9,96$ m}
\end{choices}
\end{multicols}
\end{question}
}
%%%%%%%%%%%%%%%%%%%%%%%%%%%%%%%%%%%%%%%%%%%%%%%%%%%%%%%%%%%%%%%%%%%%%%%%%%%%%
\element{test1}{
\begin{question}{p3} 
El valor de la velocidad en el punto $C$ vale:
\begin{multicols}{2}
\begin{choices}
	\correctchoice{$0.10$ mm/s}
	\wrongchoice{$0.22$ mm/s}
	\wrongchoice{$0.03$ mm/s}
	\wrongchoice{$0.41$ mm/s}
\end{choices}
\end{multicols}
\end{question}
}
%%%%%%%%%%%%%%%%%%%%%%%%%%%%%%%%%%%%%%%%%%%%%%%%%%%%%%%%%%%%%%%%%%%%%%%%%%%%%
\element{test1}{
\begin{question}{p4}
El máximo valor de la velocidad horizontal en el contorno $BE$ es:
\begin{multicols}{2}
\begin{choices}
	\correctchoice{$0.14$ mm/s}
	\wrongchoice{$0.23$ mm/s}
	\wrongchoice{$0.87$ mm/s}
	\wrongchoice{$1.11$ mm/s}
\end{choices}
\end{multicols}
\end{question}
}
%%%%%%%%%%%%%%%%%%%%%%%%%%%%%%%%%%%%%%%%%%%%%%%%%%%%%%%%%%%%%%%%%%%%%%%%%%%%%
\element{test1}{
\begin{question}{p5}
La altura piezométrica en el punto $B$ vale:
\begin{multicols}{2}
\begin{choices}
	\correctchoice{$17,29$ m}
	\wrongchoice{$19,98$ m}
	\wrongchoice{$13,21$ m}
	\wrongchoice{$16,53$ m}
\end{choices}
\end{multicols}
\end{question}
}
%%%%%%%%%%%%%%%%%%%%%%%%%%%%%%%%%%%%%%%%%%%%%%%%%%%%%%%%%%%%%%%%%%%%%%%%%%%%%
\element{test1}{
\begin{question}{p6}
El valor de la presión intersticial en el punto $A$ vale:
\begin{multicols}{2}
\begin{choices}
	\correctchoice{$56.63$ kPa}
	\wrongchoice{$18.63$ kPa}
	\wrongchoice{$17,11$ kPa}
	\wrongchoice{$1,21$ MPa}
\end{choices}
\end{multicols}
\end{question}
}
%%%%%%%%%%%%%%%%%%%%%%%%%%%%%%%%%%%%%%%%%%%%%%%%%%%%%%%%%%%%%%%%%%%%%%%%%%%%%
\element{test1}{
\begin{question}{p7} 
La altura piezométrica en el punto $H$ vale:
\begin{multicols}{2}
\begin{choices}
	\correctchoice{$20$ m}
	\wrongchoice{$11$ m}
	\wrongchoice{$15$ m}
	\wrongchoice{$17,78$ m}
\end{choices}
\end{multicols}
\end{question}
}
%%%%%%%%%%%%%%%%%%%%%%%%%%%%%%%%%%%%%%%%%%%%%%%%%%%%%%%%%%%%%%%%%%%%%%%%%%%%%
\element{test1}{
\begin{question}{p8}
El máximo valor de la velocidad en todo el dominio vale:
\begin{multicols}{2}
\begin{choices}
	\correctchoice{$0.25$ mm/s}
	\wrongchoice{$0.36$ mm/s}
	\wrongchoice{$0.41$ mm/s}
	\wrongchoice{$0.94$ mm/s}
\end{choices}
\end{multicols}
\end{question}
}
%%%%%%%%%%%%%%%%%%%%%%%%%%%%%%%%%%%%%%%%%%%%%%%%%%%%%%%%%%%%%%%%%%%%%%%%%%%%%
\element{test1}{
	\begin{question}{p9}
		En un problema de conducción de calor, la condición natural de contorno:
		$$\bm{q} \cdot \bm{n} = \overline{q}, \textrm{ en } \partial_t \Omega$$
		se interpreta como:
		\begin{multicols}{2}
			\begin{choices}
				\correctchoice{El valor impuesto, de tipo escalar, que corresponde al
					flujo de la temperatura dirección normal al contorno $\partial_t \Omega$}
				\wrongchoice{El valor impuesto, de tipo vectorial, que corresponde al
					flujo de la temperatura en el contorno $\partial_t \Omega$}
				\wrongchoice{El valor impuesto, de tipo escalar, que corresponde al
					gradiente de la temperatura en dirección normal al contorno $\partial_t \Omega$}
				\wrongchoice{El valor impuesto, de tipo vectorial, que corresponde al
					gradiente de la temperatura en el contorno $\partial_t \Omega$}
			\end{choices}
		\end{multicols}
	\end{question}
}
%%%%%%%%%%%%%%%%%%%%%%%%%%%%%%%%%%%%%%%%%%%%%%%%%%%%%%%%%%%%%%%%%%%%%%%%%%%%%
\element{test1}{
	\begin{question}{p10}
		El concepto de compatibilidad, en el ámbito del método de los elementos finitos,
		establece:
		\begin{multicols}{2}
			\begin{choices}
				\correctchoice{Un requisito de continuidad de las funciones de forma,
					entre elementos adyacentes}
				\wrongchoice{Un requisito de continuidad de las funciones de forma,
					en el interior de los elementos}
				\wrongchoice{Que cuando el tamaño de los elementos tiende a $0$,
					la solución aproximada prácticamente coincide con la solución exacta del
					problema de contorno}
				\wrongchoice{Que el valor de la suma de las funciones de forma de un
					elemento es igual a $1$}
			\end{choices}
		\end{multicols}
	\end{question}
}

%%%%%%%%%%%%%%%%%%%%%%%%%%%%%%%%%%%%%%%%%%%%%

\scoringDefaultS{b=1,m=-1/(N-1)}

\onecopy{1}{    

%%% beginning of the test sheet header:    

\noindent{\large\bf Método de los Elementos Finitos  \hfill MUECYM \hfill TEST \# 2}

\vspace*{.5cm}
\begin{minipage}{.4\linewidth}
  \centering 5 oct 2023
\end{minipage}

\begin{center}\em
Tiempo: 60 minutos.

  %Está prohibido el uso de teléfonos móviles.

  %Se atribuirá puntuación negativa a las respuestas incorrectas.

\end{center}
\vspace{1ex}

%%% end of the header

\shufflegroup{test1}
\insertgroup{test1}

\AMCcleardoublepage    
%\clearpage

\AMCformBegin    

%%% beginning of the answer sheet header

\noindent\AMCcode{nummat}{2}\hspace*{\fill}
\begin{minipage}{.7\linewidth}
$\longleftarrow{}$ Escriba su número de matrícula marcando los dígitos
en los recuadros (con ceros a la izquierda si el número es de menos de dos dígitos) y el nombre y apellidos debajo.

\vspace{3ex}

\namefield{\fbox{
   \begin{minipage}{.9\linewidth}
     Apellidos, Nombre:

     \vspace*{.5cm}\dotfill
     \vspace*{1mm}
   \end{minipage}
 }}
\end{minipage}

\begin{center}
 \bf\em Debe dar las respuestas exclusivamente en esta hoja (las respuestas en las demás hojas no serán tenidas en cuenta).
\end{center}

%%% end of the answer sheet header


\AMCform    

\AMCcleardoublepage    

}  

\end{document}
