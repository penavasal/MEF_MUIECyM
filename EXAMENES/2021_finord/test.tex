\documentclass[a4paper]{article}

%\usepackage[utf8x]{inputenc}
\usepackage[utf8x]{inputenc}
\usepackage[T1]{fontenc}
\usepackage[spanish]{babel}
\usepackage{multicol}

\usepackage{wrapfig}
\usepackage{graphicx}

\usepackage{bm}
\usepackage{amsxtra}
\usepackage{amssymb}% to get the \mathbb alphabet
\usepackage{amsmath}

\usepackage[box,completemulti,separateanswersheet]{automultiplechoice}
\def\AMCformQuestion#1{\vspace{\AMCformVSpace}\par {\sc Pregunta #1:} }
\def\AMCbeginQuestion#1#2{\par\noindent{\bf Pregunta #1}#2\hspace*{1em}}
\def\AMCcleardoublepage{\ifodd\thepage\clearpage\mbox{}\fi\clearpage}

\spanishdecimal{.}

\begin{document}

\AMCrandomseed{1237893}

%%%%%%%%%%%%%%%%%%%%%%%%%%%%%%%%%%%%%%%%%%%%%%%%%%%%%%%%%%%%%%%%%%%%%%%%%%%%%
\element{bloque1}{
\begin{question}{11}
En un modelo 3D de conducción de calor, el número de grados de libertad de un nodo
es:
\begin{multicols}{2}
\begin{choices}
     \correctchoice{$1$}
     \wrongchoice{$2$}
     \wrongchoice{$3$}
     \wrongchoice{Depende del número de nodos del elemento utilizado en la malla}
    \end{choices}
\end{multicols}
\end{question}
}
%%%%%%%%%%%%%%%%%%%%%%%%%%%%%%%%%%%%%%%%%%%%%%%%%%%%%%%%%%%%%%%%%%%%%%%%%%%%%%
\element{bloque2}{
\begin{question}{21}
En un  modelo de flujo en medios porosos, la interpretación de la variable primaria,
$u$, es:
\begin{multicols}{2}
\begin{choices}
     \correctchoice{La altura piezométrica, que es la suma de la altura geométrica
     y la altura de presión ($p/\gamma$)}
     \wrongchoice{La presión intersticial $p$}
     \wrongchoice{La velocidad}
     \wrongchoice{El gradiente de presión}
    \end{choices}
\end{multicols}
\end{question}
}
%%%%%%%%%%%%%%%%%%%%%%%%%%%%%%%%%%%%%%%%%%%%%%%%%%%%%%%%%%%%%%%%%%%%%%%%%%%%%%
\element{bloque3}{
\begin{question}{31}
La formulación de elementos finitos proporciona:
\begin{multicols}{2} 
\begin{choices}
     \correctchoice{La solución aproximada de la formulación débil de un problema
     de contorno}
     \wrongchoice{La solución aproximada de un sistema lineal de ecuaciones}
     \wrongchoice{La solución exacta de un sistema de ecuaciones diferenciales ordinarias}
     \wrongchoice{La solución exacta de un sistema de ecuaciones diferenciales en derivadas parciales}
    \end{choices}
\end{multicols}
\end{question}
}
%%%%%%%%%%%%%%%%%%%%%%%%%%%%%%%%%%%%%%%%%%%%%%%%%%%%%%%%%%%%%%%%%%%%%%%%%%%%%%
\element{bloque4}{
	\begin{question}{41}
		El número de nodos de la malla es:
		\begin{multicols}{2} 
			\begin{choices}
				\correctchoice{$857$}
				\wrongchoice{$634$}
				\wrongchoice{$431$}
				\wrongchoice{$912$}
			\end{choices}
		\end{multicols}
	\end{question}
}
%%%%%%%%%%%%%%%%%%%%%%%%%%%%%%%%%%%%%%%%%%%%%%%%%%%%%%%%%%%%%%%%%%%%%%%%%%%%%
\element{bloque5}{
	\begin{question}{51}
		El número de elementos de la malla es:
		\begin{multicols}{2}
			\begin{choices}
				\correctchoice{$804$}
				\wrongchoice{$582$}
				\wrongchoice{$404$}
				\wrongchoice{$902$}
			\end{choices}
		\end{multicols}
	\end{question}
}
%%%%%%%%%%%%%%%%%%%%%%%%%%%%%%%%%%%%%%%%%%%%%%%%%%%%%%%%%%%%%%%%%%%%%%%%%%%%%%
\element{bloque6}{
\begin{question}{61}
La diferencia entre el valor máximo y el valor mínimo de la variable primaria $u$ vale:
\begin{multicols}{2}
\begin{choices}
     \correctchoice{$9$}
     \wrongchoice{$6$}
     \wrongchoice{$10$}
     \wrongchoice{$1$}
    \end{choices}
\end{multicols}
\end{question}
}
%%%%%%%%%%%%%%%%%%%%%%%%%%%%%%%%%%%%%%%%%%%%%%%%%%%%%%%%%%%%%%%%%%%%%%%%%%%%%%
\element{bloque7}{
\begin{question}{71}
El valor máximo del módulo de la velocidad vale:
\begin{multicols}{2}
\begin{choices}
     \correctchoice{$7.22 \cdot 10^{-5}$ N}
     \wrongchoice{$1.4 \cdot 10^{-5}$ kN}
     \wrongchoice{$3.2 \cdot 10^{-5}$ N}
     \wrongchoice{$5.7 \cdot 10^{-5}$ kN}
    \end{choices}
\end{multicols}
\end{question}
}
%%%%%%%%%%%%%%%%%%%%%%%%%%%%%%%%%%%%%%%%%%%%%%%%%%%%%%%%%%%%%%%%%%%%%%%%%%%%%%
\element{bloque8}{
\begin{question}{81}
Tomando como peso específico del agua $\gamma=10000$ N/m$^3$, el valor más aproximado
para la presión en el punto más bajo de la pantalla es:
\begin{multicols}{2}
\begin{choices}
     \correctchoice{$10^5$ Pa}
     \wrongchoice{$3 \cdot 10^3$ Pa}
     \wrongchoice{$15$ Pa}
     \wrongchoice{$21$ MPa}
    \end{choices}
\end{multicols}
\end{question}
}
%%%%%%%%%%%%%%%%%%%%%%%%%%%%%%%%%%%%%%%%%%%%%%%%%%%%%%%%%%%%%%%%%%%%%%%%%%%%%
\element{bloque9}{
\begin{question}{91}
La velocidad ascendente, a la derecha de la pantalla, justo en el contacto agua terreno vale aproximadamente:
\begin{multicols}{2} 
\begin{choices}
     \correctchoice{$3.5 \cdot 10^{-5}$ m/s}
     \wrongchoice{$1.3 \cdot 10^{-3}$ m/s}
     \wrongchoice{$4.6 \cdot 10^{-4}$ m/s}
     \wrongchoice{$0.3$ m/s}
    \end{choices}
\end{multicols}
\end{question}
}
%%%%%%%%%%%%%%%%%%%%%%%%%%%%%%%%%%%%%%%%%%%%%%%%%%%%%%%%%%%%%%%%%%%%%%%%%%%%%%
\element{bloque10}{
\begin{question}{10}
El valor máximo del flujo en dirección horizontal vale:
\begin{multicols}{2} 
\begin{choices}
     \correctchoice{$5.9 \cdot 10^{-5}$ m/s}
     \wrongchoice{$8.3 \cdot 10^{-4}$ m/s}
     \wrongchoice{$3.4 \cdot 10^{-3}$ m/s}
     \wrongchoice{$1.7 \cdot 10^{-5}$ m/s}
    \end{choices}
\end{multicols}
\end{question}
}
%%%%%%%%%%%%%%%%%%%%%%%%%%%%%%%%%%%%%%%%%%%%%%%%%%%%%%%%%%%%%%%%%%%%%%%%%%%%%%

\scoringDefaultS{b=1,m=-1/(N-1)}

\onecopy{1}{

%%% beginning of the test sheet header:    

\noindent{\large\bf Método de los Elementos Finitos  \hfill MUIECYM}

\vspace*{.5cm}
\begin{minipage}{.4\linewidth}
  \centering 27 de enero 2022
\end{minipage}

\begin{center}\em
Tiempo: 60 minutos.

%  No se permite ninguna otra hoja sobre la mesa. Está prohibido el uso de calculadoras programables y teléfonos móviles.

  Se atribuirá puntuación negativa a las respuestas incorrectas.

\end{center}
\vspace{1ex}

%%% end of the header

\vspace{4mm}

%\includegraphics{ej1_graf.pdf}
%\includegraphics{ej1a_graf.pdf}
%\begin{center}
%\includegraphics[width=0.5\textwidth]{pizarra1.pdf}
%\end{center}


%%\shufflegroup{tema01}
%%\insertgroup{tema01}

\cleargroup{all}
\copygroup[1]{bloque1}{all}
\copygroup[1]{bloque2}{all}
\copygroup[1]{bloque3}{all}
\copygroup[1]{bloque4}{all}
\copygroup[1]{bloque5}{all}
\copygroup[1]{bloque6}{all}
\copygroup[1]{bloque7}{all}
\copygroup[1]{bloque8}{all}
\copygroup[1]{bloque9}{all}
\copygroup[1]{bloque10}{all}
%\shufflegroup{bloque1}\copygroup[1]{bloque1}{all}
%\shufflegroup{bloque2}\copygroup[1]{bloque2}{all}
%\shufflegroup{bloque3}\copygroup[1]{bloque3}{all}
%\shufflegroup{bloque4}\copygroup[1]{bloque4}{all}
%\shufflegroup{bloque5}\copygroup[1]{bloque5}{all}
%\shufflegroup{bloque6}\copygroup[1]{bloque6}{all}
%\shufflegroup{bloque7}\copygroup[1]{bloque7}{all}
%\shufflegroup{bloque8}\copygroup[1]{bloque8}{all}
%\shufflegroup{bloque9}\copygroup[1]{bloque9}{all}
%\shufflegroup{bloque10}\copygroup[1]{bloque10}{all}
%\shufflegroup{all}
\insertgroup{all}

%\AMCcleardoublepage
\clearpage

\AMCformBegin

%%% beginning of the answer sheet header

\noindent\AMCcode{nummat}{4}\hspace*{\fill}
\begin{minipage}{.7\linewidth}
$\longleftarrow{}$ Escriba su número de matrícula marcando los dígitos
en los recuadros (con ceros a la izquierda si el número es de menos de cuatro dígitos) y el nombre y apellidos debajo.

\vspace{3ex}

\namefield{\fbox{
   \begin{minipage}{.9\linewidth}
     Apellidos, Nombre:

     \vspace*{.5cm}\dotfill
     \vspace*{1mm}
   \end{minipage}
 }}
\end{minipage}

\begin{center}
 \bf\em Debe dar las respuestas exclusivamente en esta hoja (las respuestas en las demás hojas no serán tenidas en cuenta).
\end{center}

%%% end of the answer sheet header


\AMCform
\AMCcleardoublepage
}

\end{document}

