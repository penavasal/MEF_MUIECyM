\documentclass[a4paper]{article}

\usepackage[utf8]{inputenc}    
\usepackage[T1]{fontenc}
\usepackage[spanish]{babel}

\usepackage{bm}
\usepackage{amsxtra} 
\usepackage{amssymb}% to get the \mathbb alphabet
\usepackage{amsmath}

\usepackage{moodle}

\spanishdecimal{.}

\begin{document}
	
\begin{quiz}{MEF2-21-22}

%%%%%%%%%%%%%%%%%%%%%%%%%%%%%%%%%%%%%%%%%%%%%%%%%%%%%%%%%%%%%%%%%%%%%%%%%%%%%
\begin{multi}{p1}
En un problema lineal de conducci\'on de calor (modelo de difusi\'on) discretizado con
una malla de $N$ nodos, con $n_u$ valores conocidos de la temperatura en los
nodos, y $n_t$ valores conocidos del flujo en direcci\'on normal, el sistema
lineal de ecuaciones resultante de la formulaci\'on de elementos finitos:
	\item* Tiene $N-n_u$ ecuaciones
	\item[fraction=-33.333] Tiene $N-n_t$ ecuaciones
	\item[fraction=-33.333] Tiene $N-n_u-n_t$ ecuaciones
	\item[fraction=-33.333]Tiene $N+n_u-n_t$ ecuaciones
\end{multi}
%%%%%%%%%%%%%%%%%%%%%%%%%%%%%%%%%%%%%%%%%%%%%%%%%%%%%%%%%%%%%%%%%%%%%%%%%%%%%
\begin{multi}{p2}
En los modelos de difusi\'on, la ecuaci\'on constitutiva relaciona:
	\item* El vector flujo con el gradiente de la variable primaria
	\item[fraction=-33.333] La tensi\'on con la deformaci\'on
	\item[fraction=-33.333] El flujo en direcci\'on normal con la temperatura impuesta
	\item[fraction=-33.333] Las deformaciones con los desplazamientos
\end{multi}
%%%%%%%%%%%%%%%%%%%%%%%%%%%%%%%%%%%%%%%%%%%%%%%%%%%%%%%%%%%%%%%%%%%%%%%%%%%%%
\begin{multi}{p3}
Para garantizar la convergencia del m\'etodo de los elementos finitos es
necesario:
	\item* Verificar el requisito de complitud
	\item[fraction=-33.333] Que las funciones de forma sean compatibles
    \item[fraction=-33.333] Que todos los elementos tengan el mismo n\'umero de nodos
	\item[fraction=-33.333] Ninguna de las respuestas es correcta
\end{multi}
%%%%%%%%%%%%%%%%%%%%%%%%%%%%%%%%%%%%%%%%%%%%%%%%%%%%%%%%%%%%%%%%%%%%%%%%%%%%%
\begin{multi}{p4}
	La diferencia entre los valores m\'aximo y m\'inimo de la altura piezom\'etrica
	vale:
	\item* $6$ m.
	\item[fraction=-33.333] $3.5$ m.
	\item[fraction=-33.333] $12$ m.
	\item[fraction=-33.333] $2$ m.
\end{multi}
%%%%%%%%%%%%%%%%%%%%%%%%%%%%%%%%%%%%%%%%%%%%%%%%%%%%%%%%%%%%%%%%%%%%%%%%%%%%%
\begin{multi}{p5}
	El valor de la velocidad vertical en la zona pr\'oxima a la pantalla, justo en el contacto agua-terreno y situada a la derecha de la misma, vale aproximadamente:
	\item* $2.2 \cdot 10^{-3}$ cm/s en sentido ascendente.
	\item[fraction=-33.333] $12.2 \cdot 10^{-3}$ cm/s en sentido ascendente.
	\item[fraction=-33.333] $2.2 \cdot 10^{-3}$ m/s en sentido descendente.
	\item[fraction=-33.333] $12.2 \cdot 10^{-3}$ cm/s en sentido descendente.
\end{multi}
%%%%%%%%%%%%%%%%%%%%%%%%%%%%%%%%%%%%%%%%%%%%%%%%%%%%%%%%%%%%%%%%%%%%%%%%%%%%%
\begin{multi}{p6}
El valor m\'aximo del m\'odulo de la velocidad vale:
\item*  $7 \cdot 10^{-3}$ cm/s
\item[fraction=-33.333] $26 \cdot 10^{-3}$ cm/s
\item[fraction=-33.333] $2 \cdot 10^{-3}$ cm/s
\item[fraction=-33.333] $15 \cdot 10^{-3}$ cm/s
\end{multi}
%%%%%%%%%%%%%%%%%%%%%%%%%%%%%%%%%%%%%%%%%%%%%%%%%%%%%%%%%%%%%%%%%%%%%%%%%%%%%
\begin{multi}{p7}
	Tomando como peso espec\'ifico del agua $\gamma=10000$ N/m$^3$, el valor de la presi\'on
	del fluido en el punto m\'as bajo de la pantalla, vale aproximadamente:
	\item* $11.5$ kPa
	\item[fraction=-33.333] $11.5$ Pa
	\item[fraction=-33.333] $23.5$ Pa
	\item[fraction=-33.333] $23.5$ kPa
\end{multi}
%%%%%%%%%%%%%%%%%%%%%%%%%%%%%%%%%%%%%%%%%%%%%%%%%%%%%%%%%%%%%%%%%%%%%%%%%%%%%
\begin{multi}{p8}
	El n\'umero de nodos de la malla es:
	\item* $883$
	\item[fraction=-33.333] $752$
	\item[fraction=-33.333] $526$
	\item[fraction=-33.333] $1276$
\end{multi}
%%%%%%%%%%%%%%%%%%%%%%%%%%%%%%%%%%%%%%%%%%%%%%%%%%%%%%%%%%%%%%%%%%%%%%%%%%%%%
\begin{multi}{p9}
	El valor absoluto m\'as alto del flujo vertical se obtiene:
	\item* En uno de los paramentos verticales de la pantalla
	\item[fraction=-33.333] En el extremo inferior de la pantalla
	\item[fraction=-33.333] En el substrato rocoso
	\item[fraction=-33.333] Bajo la capa de agua de $9$ m de profundidad
\end{multi}
%%%%%%%%%%%%%%%%%%%%%%%%%%%%%%%%%%%%%%%%%%%%%%%%%%%%%%%%%%%%%%%%%%%%%%%%%%%%%
\begin{multi}{p10}
El caudal que se filtra, en r\'egimen estacionario, bajo la pantalla impermeable
vale aproximadamente:
	\item* $0.3$ l/s
	\item[fraction=-33.333] $0.2$ m$^3$/s
	\item[fraction=-33.333] $0.35$ l/h
	\item[fraction=-33.333] $9.6$ l/s
\end{multi}
%%%%%%%%%%%%%%%%%%%%%%%%%%%%%%%%%%%%%%%%%%%%%%%%%%%%%%%%%%%%%%%%%%%%%%%%%%%%%
\end{quiz}
\end{document}
