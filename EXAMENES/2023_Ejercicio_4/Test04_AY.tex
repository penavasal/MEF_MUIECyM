\documentclass[a4paper]{article}

\usepackage[utf8x]{inputenc}    
\usepackage[T1]{fontenc}
\usepackage[spanish]{babel}
\usepackage{multicol}
\spanishdecimal{.}

\usepackage{wrapfig}
\usepackage{graphicx}

\usepackage{bm}
\usepackage{amsxtra} 
\usepackage{amssymb}% to get the \mathbb alphabet
\usepackage{amsmath}

\usepackage[box,completemulti,separateanswersheet]{automultiplechoice}    
\def\AMCformQuestion#1{\vspace{\AMCformVSpace}\par {\sc Pregunta #1:} }    
\def\AMCbeginQuestion#1#2{\par\noindent{\bf Pregunta #1}#2\hspace*{1em}}
\def\AMCcleardoublepage{\ifodd\thepage\clearpage\mbox{}\fi\clearpage}

\begin{document}

\AMCrandomseed{1237893}

%%%%%%%%%%%%%%%%%%%%%%%%%%%%%%%%%%%%%%%%%%%%%%%%%%%%%%%%%%%%%%%%%%%%%%%%%%%%%
\element{test1}{
\begin{question}{p1}
Con la malla de elementos C3D8, el desplazamiento vertical máximo en el extremo libre de la viga vale:
\begin{multicols}{2}
\begin{choices}
	\correctchoice{$0.51$ m}
	\wrongchoice{$0.31$ m}
	\wrongchoice{$0.15$ m}
	\wrongchoice{$0.10$ m}
\end{choices}
\end{multicols}
\end{question}
}
%%%%%%%%%%%%%%%%%%%%%%%%%%%%%%%%%%%%%%%%%%%%%%%%%%%%%%%%%%%%%%%%%%%%%%%%%%%%%
\element{test1}{
\begin{question}{p2}
Con la malla de elementos C3D20, el desplazamiento vertical máximo en el extremo libre de la viga vale:
\begin{multicols}{2}
\begin{choices}
	\correctchoice{$0.29$ m}
	\wrongchoice{$0.61$ m}
	\wrongchoice{$0.01$ m}
	\wrongchoice{$0.47$ m}
\end{choices}
\end{multicols}
\end{question}
}
%%%%%%%%%%%%%%%%%%%%%%%%%%%%%%%%%%%%%%%%%%%%%%%%%%%%%%%%%%%%%%%%%%%%%%%%%%%%%
\element{test1}{
\begin{question}{p3} 
Con la malla de elementos C3D8R, el desplazamiento vertical máximo en el extremo libre de la viga vale:
\begin{multicols}{2}
\begin{choices}
	\correctchoice{$0.59$ m}
	\wrongchoice{$0.23$ m}
	\wrongchoice{$1.25$ m}
	\wrongchoice{$0.42$ m}
\end{choices}
\end{multicols}
\end{question}
}
%%%%%%%%%%%%%%%%%%%%%%%%%%%%%%%%%%%%%%%%%%%%%%%%%%%%%%%%%%%%%%%%%%%%%%%%%%%%%
\element{test1}{
\begin{question}{p4}
El grado de coincidencia en el valor del desplazamiento vertical máximo en el extremo libre de la viga para la malla de elementos C3D8I y C3D20 es, aproximadamente, del:
\begin{multicols}{2}
\begin{choices}
	\correctchoice{$100$\%}
	\wrongchoice{$50$\%}
	\wrongchoice{$70$\%}
	\wrongchoice{$85$\%}
\end{choices}
\end{multicols}
\end{question}
}
%%%%%%%%%%%%%%%%%%%%%%%%%%%%%%%%%%%%%%%%%%%%%%%%%%%%%%%%%%%%%%%%%%%%%%%%%%%%%
\element{test1}{
\begin{question}{p5}
Con la malla de elementos C3D20, el valor máximo del módulo de la reacción en el modelo vale:
\begin{multicols}{2}
\begin{choices}
	\correctchoice{$17.5$ MN}
	\wrongchoice{$0.26$ GN}
	\wrongchoice{$13.4$ MN}
	\wrongchoice{$2.22$ MN}
\end{choices}
\end{multicols}
\end{question}
}
%%%%%%%%%%%%%%%%%%%%%%%%%%%%%%%%%%%%%%%%%%%%%%%%%%%%%%%%%%%%%%%%%%%%%%%%%%%%%
\element{test1}{
\begin{question}{p6}
Con la malla de elementos C3D8I, el mayor valor (absoluto) de la tensión principal mínima vale:
\begin{multicols}{2}
\begin{choices}
	\correctchoice{$0.96$ GPa}
	\wrongchoice{$18.63$ kPa}
	\wrongchoice{$34.11$ kPa}
	\wrongchoice{$28.63$ MPa}
\end{choices}
\end{multicols}
\end{question}
}
%%%%%%%%%%%%%%%%%%%%%%%%%%%%%%%%%%%%%%%%%%%%%%%%%%%%%%%%%%%%%%%%%%%%%%%%%%%%%
\element{test1}{
\begin{question}{p7} 
El número de nodos en el modelo para la malla de elementos C3D8 es:
\begin{multicols}{2}
\begin{choices}
	\correctchoice{$126$}
	\wrongchoice{$120$}
	\wrongchoice{$158$}
	\wrongchoice{$164$}
\end{choices}
\end{multicols}
\end{question}
}

%%%%%%%%%%%%%%%%%%%%%%%%%%%%%%%%%%%%%%%%%%%%%

\scoringDefaultS{b=1,m=-1/(N-1)}

\onecopy{1}{    

%%% beginning of the test sheet header:    

\noindent{\large\bf Método de los Elementos Finitos  \hfill MUECYM \hfill TEST \# 4}

\vspace*{.5cm}
\begin{minipage}{.4\linewidth}
  \centering 10 nov 2023
\end{minipage}

\begin{center}\em
Tiempo: 60 minutos.

  %Está prohibido el uso de teléfonos móviles.

  %Se atribuirá puntuación negativa a las respuestas incorrectas.

\end{center}
\vspace{1ex}

%%% end of the header

\shufflegroup{test1}
\insertgroup{test1}

\AMCcleardoublepage    
%\clearpage

\AMCformBegin    

%%% beginning of the answer sheet header

\noindent\AMCcode{nummat}{2}\hspace*{\fill}
\begin{minipage}{.7\linewidth}
$\longleftarrow{}$ Escriba su número de matrícula marcando los dígitos
en los recuadros (con ceros a la izquierda si el número es de menos de dos dígitos) y el nombre y apellidos debajo.

\vspace{3ex}

\namefield{\fbox{
   \begin{minipage}{.9\linewidth}
     Apellidos, Nombre:

     \vspace*{.5cm}\dotfill
     \vspace*{1mm}
   \end{minipage}
 }}
\end{minipage}

\begin{center}
 \bf\em Debe dar las respuestas exclusivamente en esta hoja (las respuestas en las demás hojas no serán tenidas en cuenta).
\end{center}

%%% end of the answer sheet header


\AMCform    

\AMCcleardoublepage    

}  

\end{document}
