\documentclass[a4paper]{article}

\usepackage[utf8]{inputenc}    
\usepackage[T1]{fontenc}
\usepackage[spanish]{babel}

\usepackage{bm}
\usepackage{amsxtra} 
\usepackage{amssymb}% to get the \mathbb alphabet
\usepackage{amsmath}

\usepackage{moodle}

\spanishdecimal{.}

\begin{document}

\begin{quiz}{MEF4-20-21}

%%%%%%%%%%%%%%%%%%%%%%%%%%%%%%%%%%%%%%%%%%%%%%%%%%%%%%%%%%%%%%%%%%%%%%%%%%%%%
\begin{multi}{p11}
El valor del desplazamiento en direcci\'on $X$ del nodo $A$,
calculado con los elementos isoparam\'etricos, vale aproximadamente:
	\item* $-12.1$ mm
	\item[fraction=-33.333] $-7.3$ mm
	\item[fraction=-33.333] $-0.3$ mm
	\item[fraction=-33.333] $23.6$ mm
	\item[fraction=0] Respuesta en blanco
\end{multi}
%%%%%%%%%%%%%%%%%%%%%%%%%%%%%%%%%%%%%%%%%%%%%%%%%%%%%%%%%%%%%%%%%%%%%%%%%%%%%
\begin{multi}{p12}
	El valor del desplazamiento en direcci\'on $Y$ del nodo $B$,
	calculado con los elementos isoparam\'etricos, vale aproximadamente:
	\item* $12.1$ mm
	\item[fraction=-33.333] $7.8$ mm
	\item[fraction=-33.333] $21.2$ mm
	\item[fraction=-33.333] $1.2$ mm
	\item[fraction=0] Respuesta en blanco
\end{multi}
%%%%%%%%%%%%%%%%%%%%%%%%%%%%%%%%%%%%%%%%%%%%%%%%%%%%%%%%%%%%%%%%%%%%%%%%%%%%%
\begin{multi}{p21}
	El valor del desplazamiento en direcci\'on $X$ del nodo $A$,
	calculado con los elementos mixtos, vale aproximadamente:
	\item* $-15.8$ mm
	\item[fraction=-33.333] $-6.3$ mm
	\item[fraction=-33.333] $-23.2$ mm
	\item[fraction=-33.333] $-0.9$ mm
	\item[fraction=0] Respuesta en blanco
\end{multi}
%%%%%%%%%%%%%%%%%%%%%%%%%%%%%%%%%%%%%%%%%%%%%%%%%%%%%%%%%%%%%%%%%%%%%%%%%%%%%
\begin{multi}{p22}
	El valor del desplazamiento en direcci\'on $Y$ del nodo $B$,
	calculado con los elementos mixtos, vale aproximadamente:
	\item* $15.8$ mm
	\item[fraction=-33.333] $3.3$ mm
	\item[fraction=-33.333] $21.4$ mm
	\item[fraction=-33.333] $0.6$ mm
	\item[fraction=0] Respuesta en blanco
\end{multi}
%%%%%%%%%%%%%%%%%%%%%%%%%%%%%%%%%%%%%%%%%%%%%%%%%%%%%%%%%%%%%%%%%%%%%%%%%%%%%
\begin{multi}{p31}
	El valor del desplazamiento en direcci\'on $X$ del nodo $A$,
	calculado con los elementos de deformaciones mejoradas supuestas,
	vale aproximadamente:
	\item* $-21.3$ mm
	\item[fraction=-33.333] $-11.3$ mm
	\item[fraction=-33.333] $-2.7$ mm
	\item[fraction=-33.333] $-35.6$ mm
	\item[fraction=0] Respuesta en blanco
\end{multi}
%%%%%%%%%%%%%%%%%%%%%%%%%%%%%%%%%%%%%%%%%%%%%%%%%%%%%%%%%%%%%%%%%%%%%%%%%%%%%
\begin{multi}{p32}
	El valor del desplazamiento en direcci\'on $Y$ del nodo $B$,
	calculado con los elementos de deformaciones mejoradas supuestas,
	vale aproximadamente:
	\item* $21.3$ mm
	\item[fraction=-33.333] $33.4$ mm
	\item[fraction=-33.333] $13.4$ mm
	\item[fraction=-33.333] $7.6$ mm
	\item[fraction=0] Respuesta en blanco
\end{multi}
%%%%%%%%%%%%%%%%%%%%%%%%%%%%%%%%%%%%%%%%%%%%%%%%%%%%%%%%%%%%%%%%%%%%%%%%%%%%%
\begin{multi}{p4}
	Para evaluar la posible rotura del perfil, se visualizan 
	los contornos de la tensi\'on de Von Mises ({\tt pstr,6}).
	Para los elementos isoparam\'etricos, el valor m\'aximo es aproximadamente:
	\item* $27$ MPa
	\item[fraction=-33.333] $12$ MPa 
	\item[fraction=-33.333] $53$ MPa
	\item[fraction=-33.333] $6$ MPa
	\item[fraction=0] Respuesta en blanco
\end{multi}
%%%%%%%%%%%%%%%%%%%%%%%%%%%%%%%%%%%%%%%%%%%%%%%%%%%%%%%%%%%%%%%%%%%%%%%%%%%%%
\begin{multi}{p5}
	Para evaluar la posible rotura del perfil, se visualizan 
	los contornos de la tensi\'on de Von Mises ({\tt pstr,6}).
	Para los elementos mixtos, el valor m\'aximo es aproximadamente:
	\item* $29$ MPa
	\item[fraction=-33.333] $15$ MPa
	\item[fraction=-33.333] $48$ MPa
	\item[fraction=-33.333] $9$ MPa
	\item[fraction=0] Respuesta en blanco
\end{multi}
%%%%%%%%%%%%%%%%%%%%%%%%%%%%%%%%%%%%%%%%%%%%%%%%%%%%%%%%%%%%%%%%%%%%%%%%%%%%%
\begin{multi}{p6}
	Para evaluar la posible rotura del perfil, se visualizan 
	los contornos de la tensi\'on de Von Mises ({\tt pstr,6}).
	Para los elementos de deformaciones mejoradas supuestas, el valor m\'aximo
	es aproximadamente:
	\item* $32$ MPa
	\item[fraction=-33.333] $17$ MPa
	\item[fraction=-33.333] $49$ MPa
	\item[fraction=-33.333] $3$ MPa
	\item[fraction=0] Respuesta en blanco
\end{multi}
%%%%%%%%%%%%%%%%%%%%%%%%%%%%%%%%%%%%%%%%%%%%%%%%%%%%%%%%%%%%%%%%%%%%%%%%%%%%%
\begin{multi}{p7}
De acuerdo con los valores de los contornos de la tensi\'on de
Von Mises, la posible zona de rotura del perfil es:
\item* Con los tres modelos, en los puntos en que se aplican las cargas
\item[fraction=-33.333] Con el modelo de elementos con deformaciones mejoradas
supuestas, en los puntos en que se aplican las cargas. Con los modelos de
elementos mixtos e isoparam\'etricos, en la zona central de la cara interna
\item[fraction=-33.333] Con los modelos de elementos mixtos e isoparam\'etricos,
en los puntos en que se aplican las cargas. Con el modelo de elementos con
deformaciones mejoradas supuestas, en la zona central de la cara interna
\item[fraction=-33.333] Con los tres modelos, en la zona central de la cara
interna
\item[fraction=0] Respuesta en blanco
\end{multi}
%%%%%%%%%%%%%%%%%%%%%%%%%%%%%%%%%%%%%%%%%%%%%%%%%%%%%%%%%%%%%%%%%%%%%%%%%%%%%
\begin{multi}{p81}
	Para los elementos isoparam\'etricos, el valor m\'aximo del desplazamiento
	seg\'un $Oz$ es aproximadamente:
	\item* $0.9$ mm
	\item[fraction=-33.333] $1.8$ mm
	\item[fraction=-33.333] $8.3$ mm
	\item[fraction=-33.333] $17.2$ mm
	\item[fraction=0] Respuesta en blanco
\end{multi}
%%%%%%%%%%%%%%%%%%%%%%%%%%%%%%%%%%%%%%%%%%%%%%%%%%%%%%%%%%%%%%%%%%%%%%%%%%%%%
\begin{multi}{p82}
	Para los elementos mixtos, el valor m\'aximo del desplazamiento
	seg\'un $Oz$ es aproximadamente:
	\item* $0.9$ mm
	\item[fraction=-33.333] $2.6$ mm
	\item[fraction=-33.333] $9.4$ mm
	\item[fraction=-33.333] $14.3$ mm
	\item[fraction=0] Respuesta en blanco
\end{multi}
%%%%%%%%%%%%%%%%%%%%%%%%%%%%%%%%%%%%%%%%%%%%%%%%%%%%%%%%%%%%%%%%%%%%%%%%%%%%%
\begin{multi}{p83}
	Para los elementos de deformaciones mejoradas supuestas, el valor m\'aximo
	del desplazamiento seg\'un $Oz$ es aproximadamente:
	\item* $1.2$ mm
	\item[fraction=-33.333] $0.3$ mm
	\item[fraction=-33.333] $7.8$ mm
	\item[fraction=-33.333] $13.2$ mm
	\item[fraction=0] Respuesta en blanco
\end{multi}
%%%%%%%%%%%%%%%%%%%%%%%%%%%%%%%%%%%%%%%%%%%%%%%%%%%%%%%%%%%%%%%%%%%%%%%%%%%%%
\begin{multi}{p91}
	El valor de $\sigma_{xx}$ en el nodo $A$,
	calculado con los elementos isoparam\'etricos, vale aproximadamente:
	\item* $-20$ MPa
	\item[fraction=-33.333] $-7.1$ MPa
	\item[fraction=-33.333] $-45.8$ MPa
	\item[fraction=-33.333] $-101.3$ MPa
	\item[fraction=0] Respuesta en blanco
	\end{multi}
%%%%%%%%%%%%%%%%%%%%%%%%%%%%%%%%%%%%%%%%%%%%%%%%%%%%%%%%%%%%%%%%%%%%%%%%%%%%%
\begin{multi}{p92}
	El valor de $\sigma_{xx}$ en el nodo $A$,
	calculado con los elementos mixtos, vale aproximadamente:
	\item* $-18.6$ MPa
	\item[fraction=-33.333] $-4.6$ MPa
	\item[fraction=-33.333] $-43.2$ MPa
	\item[fraction=-33.333] $-98.3$ MPa
	\item[fraction=0] Respuesta en blanco
\end{multi}
%%%%%%%%%%%%%%%%%%%%%%%%%%%%%%%%%%%%%%%%%%%%%%%%%%%%%%%%%%%%%%%%%%%%%%%%%%%%%
\begin{multi}{p93}
	El valor de $\sigma_{xx}$ en el nodo $A$,
	calculado con los elementos de deformaciones mejoradas supuestas,
	vale aproximadamente:
	\item* $-25.3$ MPa
	\item[fraction=-33.333] $-3.2$ MPa
	\item[fraction=-33.333] $-52.3$ MPa
	\item[fraction=-33.333] $-88.4$ MPa
	\item[fraction=0] Respuesta en blanco
\end{multi}
%%%%%%%%%%%%%%%%%%%%%%%%%%%%%%%%%%%%%%%%%%%%%%%%%%%%%%%%%%%%%%%%%%%%%%%%%%%%%
\begin{multi}{p101}
	El valor de $\sigma_{yy}$ en el nodo $B$,
	calculado con los elementos isoparam\'etricos, vale aproximadamente:
	\item* $20$ MPa
	\item[fraction=-33.333] $11$ MPa
	\item[fraction=-33.333] $6$ MPa
	\item[fraction=-33.333] $43$ MPa
	\item[fraction=0] Respuesta en blanco
\end{multi}
%%%%%%%%%%%%%%%%%%%%%%%%%%%%%%%%%%%%%%%%%%%%%%%%%%%%%%%%%%%%%%%%%%%%%%%%%%%%%
\begin{multi}{p102}
	El valor de $\sigma_{yy}$ en el nodo $B$,
	calculado con los elementos mixtos, vale aproximadamente:
	\item* $18.6$ MPa
	\item[fraction=-33.333] $12$ MPa
	\item[fraction=-33.333] $9$ MPa
	\item[fraction=-33.333] $41$ MPa
	\item[fraction=0] Respuesta en blanco
\end{multi}
%%%%%%%%%%%%%%%%%%%%%%%%%%%%%%%%%%%%%%%%%%%%%%%%%%%%%%%%%%%%%%%%%%%%%%%%%%%%%
\begin{multi}{p103}
	El valor de $\sigma_{yy}$ en el nodo $B$,
	calculado con los elementos de deformaciones mejoradas supuestas,
	vale aproximadamente:
	\item* $25.3$ MPa
	\item[fraction=-33.333] $16$ MPa
	\item[fraction=-33.333] $8$ MPa
	\item[fraction=-33.333] $47$ MPa
	\item[fraction=0] Respuesta en blanco
\end{multi}
%%%%%%%%%%%%%%%%%%%%%%%%%%%%%%%%%%%%%%%%%%%%%%%%%%%%%%%%%%%%%%%%%%%%%%%%%%%%%

\end{quiz}

\end{document}
