\documentclass[a4paper]{article}

\usepackage[utf8]{inputenc}    
\usepackage[T1]{fontenc}
\usepackage[spanish]{babel}

\usepackage{bm}
\usepackage{amsxtra} 
\usepackage{amssymb}% to get the \mathbb alphabet
\usepackage{amsmath}

\usepackage{moodle}

\spanishdecimal{.}

\begin{document}
	
%%%%%%%%%%%%%%%%%%%%%%%%%%%%%%%%%%%%%%%%%%%%%%%%%%%%%%%%%%%%%%%%%%%%%%%%%%%%%
\begin{quiz}{MEF1-21-22}
%%%%%%%%%%%%%%%%%%%%%%%%%%%%%%%%%%%%%%%%%%%%%%%%%%%%%%%%%%%%%%%%%%%%%%%%%%%%%

%%%%%%%%%%%%%%%%%%%%%%%%%%%%%%%%%%%%%%%%%%%%%%%%%%%%%%%%%%%%%%%%%%%%%%%%%%%%%
\begin{multi}{p1}
El movimiento horizontal m\'aximo (en valor absoluto) vale:
	\item* $1.5$ cm
	\item[fraction=-33.333]  $3.1$ cm
	\item[fraction=-33.333]  $1.5$ mm
	\item[fraction=-33.333]  $3.1$ mm
\end{multi}
%%%%%%%%%%%%%%%%%%%%%%%%%%%%%%%%%%%%%%%%%%%%%%%%%%%%%%%%%%%%%%%%%%%%%%%%%%%%%
\begin{multi}{p2}
El movimiento vertical m\'aximo (en valor absoluto) vale:
	\item* $3.8$ cm
	\item[fraction=-33.333] $1.1$ cm
	\item[fraction=-33.333] $1.1$ mm
	\item[fraction=-33.333] $3.8$ mm
\end{multi}
%%%%%%%%%%%%%%%%%%%%%%%%%%%%%%%%%%%%%%%%%%%%%%%%%%%%%%%%%%%%%%%%%%%%%%%%%%%%%
\begin{multi}{p3}
El movimiento horizontal del nodo $7$ vale:
	\item* $5.8$ mm
	\item[fraction=-33.333] $2.3$ mm
	\item[fraction=-33.333] $1.6$ mm
	\item[fraction=-33.333] $1.2$ cm
\end{multi}
%%%%%%%%%%%%%%%%%%%%%%%%%%%%%%%%%%%%%%%%%%%%%%%%%%%%%%%%%%%%%%%%%%%%%%%%%%%%%
\begin{multi}{p4}
La reacci\'on vertical en el nodo $2$ vale:
	\item* $20.0$ kN
	\item[fraction=-33.333] $60.0$ kN
	\item[fraction=-33.333] $30.0$ kN
	\item[fraction=-33.333] $40.0$ kN
\end{multi}
%%%%%%%%%%%%%%%%%%%%%%%%%%%%%%%%%%%%%%%%%%%%%%%%%%%%%%%%%%%%%%%%%%%%%%%%%%%%%
\begin{multi}{p5}
El n\'umero de grados de libertad del modelo de elementos finitos es:
	\item* $12$
	\item[fraction=-33.333] $6$
	\item[fraction=-33.333] $16$
	\item[fraction=-33.333] $13$
\end{multi}
%%%%%%%%%%%%%%%%%%%%%%%%%%%%%%%%%%%%%%%%%%%%%%%%%%%%%%%%%%%%%%%%%%%%%%%%%%%%%
\begin{multi}{p6}
En el modelo de barras articuladas, la formulaci\'on d\'ebil del problema
de contorno se puede interpretar como:
	\item* El principio de los trabajos virtuales
	\item[fraction=-33.333] La nulidad de la suma de fuerzas y la suma de momentos
	\item[fraction=-33.333] La formulaci\'on de Galerkin expresada en forma integral
	\item[fraction=-33.333] La relaciones tensi\'on-deformaci\'on y deformaci\'on-desplazamiento
\end{multi}
%%%%%%%%%%%%%%%%%%%%%%%%%%%%%%%%%%%%%%%%%%%%%%%%%%%%%%%%%%%%%%%%%%%%%%%%%%%%%
\begin{multi}{p7}
El esfuerzo de tracci\'on m\'axima es:
	\item* $89$ kN
	\item[fraction=-33.333] $50.$ kN
	\item[fraction=-33.333] $102$ kN
	\item[fraction=-33.333] No hay barras con esfuerzos de tracci\'on
\end{multi}
%%%%%%%%%%%%%%%%%%%%%%%%%%%%%%%%%%%%%%%%%%%%%%%%%%%%%%%%%%%%%%%%%%%%%%%%%%%%%
\begin{multi}{p8}
El esfuerzo de compresi\'on m\'axima es:
	\item* $-113$ kN
	\item[fraction=-33.333] $-430$ kN
	\item[fraction=-33.333] $-265$ kN
	\item[fraction=-33.333] $-531$ kN
\end{multi}
%%%%%%%%%%%%%%%%%%%%%%%%%%%%%%%%%%%%%%%%%%%%%%%%%%%%%%%%%%%%%%%%%%%%%%%%%%%%%
\begin{multi}{p9}
La tensi\'on en la barra $4$ vale:
	\item* $81$ N/mm$^2$ (de compresi\'on)
	\item[fraction=-33.333] $81$ N/mm$^2$ (de tracci\'on)
	\item[fraction=-33.333] $40$ N/mm$^2$ (de tracci\'on)
	\item[fraction=-33.333] $40$ N/mm$^2$ (de compresi\'on)
\end{multi}
%%%%%%%%%%%%%%%%%%%%%%%%%%%%%%%%%%%%%%%%%%%%%%%%%%%%%%%%%%%%%%%%%%%%%%%%%%%%%
\begin{multi}{p10}
La tensi\'on en la barra $7$ vale:
	\item* $182$ N/mm$^2$ (de tracci\'on)
	\item[fraction=-33.333] $182$ N/mm$^2$ (de compresi\'on)
	\item[fraction=-33.333] $900$ N/mm$^2$ (de tracci\'on)
	\item[fraction=-33.333] $900$ N/mm$^2$ (de compresi\'on)
\end{multi}

\end{quiz}
\end{document}
