\documentclass[a4paper]{article}

\usepackage[utf8x]{inputenc}    
\usepackage[T1]{fontenc}
\usepackage[spanish]{babel}
\usepackage{multicol}

\usepackage{wrapfig}
\usepackage{graphicx}

\usepackage{bm}
\usepackage{amsxtra} 
\usepackage{amssymb}% to get the \mathbb alphabet
\usepackage{amsmath}

\usepackage[box,completemulti,separateanswersheet]{automultiplechoice}    
\def\AMCformQuestion#1{\vspace{\AMCformVSpace}\par {\sc Pregunta #1:} }    
\def\AMCbeginQuestion#1#2{\par\noindent{\bf Pregunta #1}#2\hspace*{1em}}
\def\AMCcleardoublepage{\ifodd\thepage\clearpage\mbox{}\fi\clearpage}

\begin{document}

\AMCrandomseed{1237893}

%%%%%%%%%%%%%%%%%%%%%%%%%%%%%%%%%%%%%%%%%%%%%%%%%%%%%%%%%%%%%%%%%%%%%%%%%%%%%
\element{test1}{
\begin{question}{p1}
El movimiento horizontal máximo (en valor absoluto) vale:
\begin{multicols}{2}
\begin{choices}
	\correctchoice{$1.5$ cm}
	\wrongchoice{$3.1$ cm}
	\wrongchoice{$1.5$ mm}
	\wrongchoice{$3.1$ mm}
\end{choices}
\end{multicols}
\end{question}
}
%%%%%%%%%%%%%%%%%%%%%%%%%%%%%%%%%%%%%%%%%%%%%%%%%%%%%%%%%%%%%%%%%%%%%%%%%%%%%
\element{test1}{
\begin{question}{p2}
El movimiento vertical máximo (en valor absoluto) vale:
\begin{multicols}{2}
\begin{choices}
	\correctchoice{$3.8$ cm}
	\wrongchoice{$1.1$ cm}
	\wrongchoice{$1.1$ mm}
	\wrongchoice{$3.8$ mm}
\end{choices}
\end{multicols}
\end{question}
}
%%%%%%%%%%%%%%%%%%%%%%%%%%%%%%%%%%%%%%%%%%%%%%%%%%%%%%%%%%%%%%%%%%%%%%%%%%%%%
\element{test1}{
\begin{question}{p3}
El movimiento horizontal del nodo $7$ vale:
\begin{multicols}{2}
\begin{choices}
	\correctchoice{$5.8$ mm}
	\wrongchoice{$2.3$ mm}
	\wrongchoice{$1.6$ mm}
	\wrongchoice{$1.2$ cm}
\end{choices}
\end{multicols}
\end{question}
}
%%%%%%%%%%%%%%%%%%%%%%%%%%%%%%%%%%%%%%%%%%%%%%%%%%%%%%%%%%%%%%%%%%%%%%%%%%%%%
\element{test1}{
\begin{question}{p4}
La reacción vertical en el nodo $2$ vale:
\begin{multicols}{2}
\begin{choices}
	\correctchoice{$20.0$ kN}
	\wrongchoice{$60.0$ kN}
	\wrongchoice{$30.0$ kN}
	\wrongchoice{$40.0$ kN}
\end{choices}
\end{multicols}
\end{question}
}
%%%%%%%%%%%%%%%%%%%%%%%%%%%%%%%%%%%%%%%%%%%%%%%%%%%%%%%%%%%%%%%%%%%%%%%%%%%%%
\element{test1}{
\begin{question}{p5}
Con las cargas impuestas, las barras que no trabajan son:
\begin{multicols}{2}
\begin{choices}
	\correctchoice{$2$, $3$, $6$, $10$ y $11$}
	\wrongchoice{$1$ y $4$}
	\wrongchoice{$2$, $3$ y $11$}
	\wrongchoice{Todas las barras están en tracción o en compresión}
\end{choices}
\end{multicols}
\end{question}
}
%%%%%%%%%%%%%%%%%%%%%%%%%%%%%%%%%%%%%%%%%%%%%%%%%%%%%%%%%%%%%%%%%%%%%%%%%%%%%
\element{test1}{
\begin{question}{p6}
Las barras que están en tracción son:
\begin{multicols}{2}
\begin{choices}
	\correctchoice{$5$, $7$, $8$ y $13$}
	\wrongchoice{$2$, $3$ y $11$}
	\wrongchoice{$1$, $4$, $6$ $9$ y $12$}
	\wrongchoice{Todas las barras trabajan a compresión}
\end{choices}
\end{multicols}
\end{question}
}
%%%%%%%%%%%%%%%%%%%%%%%%%%%%%%%%%%%%%%%%%%%%%%%%%%%%%%%%%%%%%%%%%%%%%%%%%%%%%
\element{test1}{
\begin{question}{p7}
El esfuerzo de tracción máxima es:
\begin{multicols}{2}
\begin{choices}
	\correctchoice{$89$ kN}
	\wrongchoice{$50.$ kN}
	\wrongchoice{$102$ kN}
	\wrongchoice{No hay barras con esfuerzos de tracción}
\end{choices}
\end{multicols}
\end{question}
}
%%%%%%%%%%%%%%%%%%%%%%%%%%%%%%%%%%%%%%%%%%%%%%%%%%%%%%%%%%%%%%%%%%%%%%%%%%%%%
\element{test1}{
\begin{question}{p8}
El esfuerzo de compresión máxima es:
\begin{multicols}{2}
\begin{choices}
	\correctchoice{$-113$ kN}
	\wrongchoice{$-430$ kN}
	\wrongchoice{$-265$ kN}
	\wrongchoice{$-531$ kN}
\end{choices}
\end{multicols}
\end{question}
}
%%%%%%%%%%%%%%%%%%%%%%%%%%%%%%%%%%%%%%%%%%%%%%%%%%%%%%%%%%%%%%%%%%%%%%%%%%%%%
\element{test1}{
\begin{question}{p9}
La tensión en la barra $4$ vale:
\begin{multicols}{2}
\begin{choices}
	\correctchoice{$81$ N/mm$^2$ (de compresión)}
	\wrongchoice{$81$ N/mm$^2$ (de tracción)}
	\wrongchoice{$40$ N/mm$^2$ (de tracción)}
	\wrongchoice{$40$ N/mm$^2$ (de compresión)}
\end{choices}
\end{multicols}
\end{question}
}
%%%%%%%%%%%%%%%%%%%%%%%%%%%%%%%%%%%%%%%%%%%%%%%%%%%%%%%%%%%%%%%%%%%%%%%%%%%%%
\element{test1}{
\begin{question}{p10}
La tensión en la barra $7$ vale:
\begin{multicols}{2}
\begin{choices}
	\correctchoice{$182$ N/mm$^2$ (de tracción)}
	\wrongchoice{$182$ N/mm$^2$ (de compresión)}
	\wrongchoice{$900$ N/mm$^2$ (de tracción)}
	\wrongchoice{$900$ N/mm$^2$ (de compresión)}
\end{choices}
\end{multicols}
\end{question}
}
%%%%%%%%%%%%%%%%%%%%%%%%%%%%%%%%%%%%%%%%%%%%%

\scoringDefaultS{b=1,m=-1/(N-1)}

\onecopy{1}{    

%%% beginning of the test sheet header:    

\noindent{\large\bf Método de los Elementos Finitos  \hfill MUECYM \hfill TEST \# 1}

\vspace*{.5cm}
\begin{minipage}{.4\linewidth}
  \centering 26 sep 2019
\end{minipage}

\begin{center}\em
Tiempo: 60 minutos.

  %Está prohibido el uso de teléfonos móviles.

  %Se atribuirá puntuación negativa a las respuestas incorrectas.

\end{center}
\vspace{1ex}

%%% end of the header

\shufflegroup{test1}
\insertgroup{test1}

\AMCcleardoublepage    
%\clearpage

\AMCformBegin    

%%% beginning of the answer sheet header

\noindent\AMCcode{nummat}{2}\hspace*{\fill}
\begin{minipage}{.7\linewidth}
$\longleftarrow{}$ Escriba su número de matrícula marcando los dígitos
en los recuadros (con ceros a la izquierda si el número es de menos de dos dígitos) y el nombre y apellidos debajo.

\vspace{3ex}

\namefield{\fbox{
   \begin{minipage}{.9\linewidth}
     Apellidos, Nombre:

     \vspace*{.5cm}\dotfill
     \vspace*{1mm}
   \end{minipage}
 }}
\end{minipage}

\begin{center}
 \bf\em Debe dar las respuestas exclusivamente en esta hoja (las respuestas en las demás hojas no serán tenidas en cuenta).
\end{center}

%%% end of the answer sheet header


\AMCform    

\AMCcleardoublepage    

}  

\end{document}
