\documentclass[a4paper]{article}

%%%%%%%%%%%%%%%%%%%%%%%%%%%%%%%%%%%%%%%%%%%%%%%%%%%%%%%%%%%%%%%%%%%%%%%%%%%%%
%   CARGO LOS PAQUETES DE LATEX QUE VOY A NECESITAR - EL SEGUNDO GRUPO PUEDE VARIAR
%%%%%%%%%%%%%%%%%%%%%%%%%%%%%%%%%%%%%%%%%%%%%%%%%%%%%%%%%%%%%%%%%%%%%%%%%%%%%
\usepackage[utf8x]{inputenc}    
\usepackage[T1]{fontenc}
\usepackage[spanish]{babel}
\usepackage{multicol}

% Estos pueden variar segun tus necesidades
\usepackage{wrapfig}
\usepackage{graphicx}
\usepackage{bm}
\usepackage{amsxtra} 
\usepackage{amssymb}
\usepackage{amsmath}


%%%%%%%%%%%%%%%%%%%%%%%%%%%%%%%%%%%%%%%%%%%%%%%%%%%%%%%%%%%%%%%%%%%%%%%%%%%%%
%   CARGO EL PAQUETE automultiplechoice CON SUS OPCIONES 
%%%%%%%%%%%%%%%%%%%%%%%%%%%%%%%%%%%%%%%%%%%%%%%%%%%%%%%%%%%%%%%%%%%%%%%%%%%%%

\usepackage[box,completemulti,lang=ES,separateanswersheet]{automultiplechoice}  
  
\def\AMCcleardoublepage{\ifodd\thepage\clearpage\mbox{}\fi\clearpage}
\AMCtext{none}{Ninguna de estas respuestas es correcta}

%%%%%%%%%%%%%%%%%%%%%%%%%%%%%%%%%%%%%%%%%%%%%%%%%%%%%%%%%%%%%%%%%%%%%%%%%%%%%
%  INICIO EL DOCUMENTO
%%%%%%%%%%%%%%%%%%%%%%%%%%%%%%%%%%%%%%%%%%%%%%%%%%%%%%%%%%%%%%%%%%%%%%%%%%%%%

\begin{document}

% Numero semilla que uso para aleatorizar las preguntas
\AMCrandomseed{1237893}


%%%%%%%%%%%%%%%%%%%%%%%%%%%%%%%%%%%%%%%%%%%%%%%%%%%%%%%%%%%%%%%%%%%%%%%%%%%%%
% CARGO LAS PREGUNTAS DE LAS BIBLIOTECAS QUE VOY A USAR
%%%%%%%%%%%%%%%%%%%%%%%%%%%%%%%%%%%%%%%%%%%%%%%%%%%%%%%%%%%%%%%%%%%%%%%%%%%%%
% \element{test1}{
\begin{question}{p1}
El caudal que se filtra, en régimen estacionario, a la derecha de la pantalla impermeable (contorno $CD$) vale:
\begin{multicols}{2}
\begin{choices}
	\correctchoice{$0.85$ l/s}
	\wrongchoice{$4.21$ l/s}
	\wrongchoice{$1.56$ l/s}
	\wrongchoice{$0.04$ l/s}
\end{choices}
\end{multicols}
\end{question}
}
%%%%%%%%%%%%%%%%%%%%%%%%%%%%%%%%%%%%%%%%%%%%%%%%%%%%%%%%%%%%%%%%%%%%%%%%%%%%%
\element{test1}{
\begin{question}{p2}
El valor máximo de la altura piezométrica en el contorno $FG$ vale:
\begin{multicols}{2}
\begin{choices}
	\correctchoice{$15,45$ m}
	\wrongchoice{$20,00$ m}
	\wrongchoice{$13,54$ m}
	\wrongchoice{$9,96$ m}
\end{choices}
\end{multicols}
\end{question}
}
%%%%%%%%%%%%%%%%%%%%%%%%%%%%%%%%%%%%%%%%%%%%%%%%%%%%%%%%%%%%%%%%%%%%%%%%%%%%%
\element{test1}{
\begin{question}{p3} 
El valor de la velocidad en el punto $C$ vale:
\begin{multicols}{2}
\begin{choices}
	\correctchoice{$0.09$ mm/s}
	\wrongchoice{$0.22$ mm/s}
	\wrongchoice{$0.03$ mm/s}
	\wrongchoice{$0.41$ mm/s}
\end{choices}
\end{multicols}
\end{question}
}
%%%%%%%%%%%%%%%%%%%%%%%%%%%%%%%%%%%%%%%%%%%%%%%%%%%%%%%%%%%%%%%%%%%%%%%%%%%%%
\element{test1}{
\begin{question}{p4}
El máximo valor de la velocidad horizontal en el contorno $BE$ es:
\begin{multicols}{2}
\begin{choices}
	\correctchoice{$0.15$ mm/s}
	\wrongchoice{$0.02$ mm/s}
	\wrongchoice{$3.33$ mm/s}
	\wrongchoice{$5.21$ mm/s}
\end{choices}
\end{multicols}
\end{question}
}
%%%%%%%%%%%%%%%%%%%%%%%%%%%%%%%%%%%%%%%%%%%%%%%%%%%%%%%%%%%%%%%%%%%%%%%%%%%%%
\element{test1}{
\begin{question}{p5}
La altura piezométrica en el punto $B$ vale:
\begin{multicols}{2}
\begin{choices}
	\correctchoice{$17,03$ m}
	\wrongchoice{$19,99$ m}
	\wrongchoice{$14,21$ m}
	\wrongchoice{$16,51$ m}
\end{choices}
\end{multicols}
\end{question}
}
%%%%%%%%%%%%%%%%%%%%%%%%%%%%%%%%%%%%%%%%%%%%%%%%%%%%%%%%%%%%%%%%%%%%%%%%%%%%%
\element{test1}{
\begin{question}{p6}
El valor de la presión intersticial en el punto $A$ vale:
\begin{multicols}{2}
\begin{choices}
	\correctchoice{$54.44$ kPa}
	\wrongchoice{$19.62$ kPa}
	\wrongchoice{$27,14$ kPa}
	\wrongchoice{$1,21$ MPa}
\end{choices}
\end{multicols}
\end{question}
}
%%%%%%%%%%%%%%%%%%%%%%%%%%%%%%%%%%%%%%%%%%%%%%%%%%%%%%%%%%%%%%%%%%%%%%%%%%%%%
\element{test1}{
\begin{question}{p7} 
En la Formulación Débil, la función de peso (o función de prueba) $\delta u$ debe
        cumplir obligatoriamente la siguiente restricción:
        \begin{multicols}{2}
            \begin{choices}
                \correctchoice{Debe anularse ($\delta u = 0$) en el contorno esencial (Dirichlet)}
                \wrongchoice{Debe anularse ($\delta u = 0$) en el contorno natural (Neumann)}
                \wrongchoice{Debe ser igual al flujo impuesto ($\delta u = \overline{q}$) en
                    el contorno de Neumann}
                \wrongchoice{Debe ser igual a la solución $u$ en todo el dominio}
\end{choices}
\end{multicols}
\end{question}
}
%%%%%%%%%%%%%%%%%%%%%%%%%%%%%%%%%%%%%%%%%%%%%%%%%%%%%%%%%%%%%%%%%%%%%%%%%%%%%
\element{test1}{
\begin{question}{p8}
El máximo valor de la velocidad en todo el dominio vale:
\begin{multicols}{2}
\begin{choices}
	\correctchoice{$0.27$ mm/s}
	\wrongchoice{$0.35$ mm/s}
	\wrongchoice{$0.42$ mm/s}
	\wrongchoice{$0.97$ mm/s}
\end{choices}
\end{multicols}
\end{question}
}
%%%%%%%%%%%%%%%%%%%%%%%%%%%%%%%%%%%%%%%%%%%%%%%%%%%%%%%%%%%%%%%%%%%%%%%%%%%%%
\element{test1}{
	\begin{question}{p9}
		En un problema de flujo en medios porosos (Ley de Darcy), la condición esencial de contorno:
        $$u = \overline{u}, \textrm{ en } \partial_u \Omega$$
        se interpreta como:
        \begin{multicols}{2}
            \begin{choices}
                \correctchoice{El valor impuesto, de tipo escalar, que corresponde al
                    potencial $u$ (altura piezométrica) en el contorno $\partial_u \Omega$}
                \wrongchoice{El valor impuesto, de tipo escalar, que corresponde al
                    flujo en dirección normal al contorno $\partial_u \Omega$}
                \wrongchoice{El valor impuesto, de tipo vectorial, que corresponde al
                    potencial $u$ (altura piezométrica) en el contorno $\partial_u \Omega$}
                \wrongchoice{El valor impuesto, de tipo vectorial, que corresponde al
                    gradiente del potencial en el contorno $\partial_u \Omega$}
			\end{choices}
		\end{multicols}
	\end{question}
}
%%%%%%%%%%%%%%%%%%%%%%%%%%%%%%%%%%%%%%%%%%%%%%%%%%%%%%%%%%%%%%%%%%%%%%%%%%%%%
\element{test1}{
	\begin{question}{p10}
		En un problema lineal de conducci\'on de calor (modelo de difusi\'on) discretizado con
una malla de $N$ nodos, con $n_u$ valores conocidos de la temperatura en los
nodos, y $n_t$ valores conocidos del flujo en direcci\'on normal, el sistema
lineal de ecuaciones resultante de la formulaci\'on de elementos finitos:
		\begin{multicols}{2}
			\begin{choices}
				\correctchoice{Tiene $N-n_u$ ecuaciones}
				\wrongchoice{Tiene $N-n_t$ ecuaciones}
				\wrongchoice{Tiene $N-n_u-n_t$ ecuaciones}
				\wrongchoice{Tiene $N+n_u-n_t$ ecuaciones}
			\end{choices}
		\end{multicols}
	\end{question}
}




\element{test1}{
\begin{question}{p1}
El caudal que se filtra, en régimen estacionario, a la derecha de la pantalla impermeable (contorno $CD$) vale:
\begin{multicols}{2}
\begin{choices}
	\correctchoice{$0.85$ l/s}
	\wrongchoice{$4.21$ l/s}
	\wrongchoice{$1.56$ l/s}
	\wrongchoice{$0.04$ l/s}
\end{choices}
\end{multicols}
\end{question}
}
%%%%%%%%%%%%%%%%%%%%%%%%%%%%%%%%%%%%%%%%%%%%%%%%%%%%%%%%%%%%%%%%%%%%%%%%%%%%%
\element{test1}{
\begin{question}{p2}
El valor máximo de la altura piezométrica en el contorno $FG$ vale:
\begin{multicols}{2}
\begin{choices}
	\correctchoice{$15,45$ m}
	\wrongchoice{$20,00$ m}
	\wrongchoice{$13,54$ m}
	\wrongchoice{$9,96$ m}
\end{choices}
\end{multicols}
\end{question}
}
%%%%%%%%%%%%%%%%%%%%%%%%%%%%%%%%%%%%%%%%%%%%%%%%%%%%%%%%%%%%%%%%%%%%%%%%%%%%%
\element{test1}{
\begin{question}{p3} 
El valor de la velocidad en el punto $C$ vale:
\begin{multicols}{2}
\begin{choices}
	\correctchoice{$0.09$ mm/s}
	\wrongchoice{$0.22$ mm/s}
	\wrongchoice{$0.03$ mm/s}
	\wrongchoice{$0.41$ mm/s}
\end{choices}
\end{multicols}
\end{question}
}
%%%%%%%%%%%%%%%%%%%%%%%%%%%%%%%%%%%%%%%%%%%%%%%%%%%%%%%%%%%%%%%%%%%%%%%%%%%%%
\element{test1}{
\begin{question}{p4}
El máximo valor de la velocidad horizontal en el contorno $BE$ es:
\begin{multicols}{2}
\begin{choices}
	\correctchoice{$0.15$ mm/s}
	\wrongchoice{$0.02$ mm/s}
	\wrongchoice{$3.33$ mm/s}
	\wrongchoice{$5.21$ mm/s}
\end{choices}
\end{multicols}
\end{question}
}
%%%%%%%%%%%%%%%%%%%%%%%%%%%%%%%%%%%%%%%%%%%%%%%%%%%%%%%%%%%%%%%%%%%%%%%%%%%%%
\element{test1}{
\begin{question}{p5}
La altura piezométrica en el punto $B$ vale:
\begin{multicols}{2}
\begin{choices}
	\correctchoice{$17,03$ m}
	\wrongchoice{$19,99$ m}
	\wrongchoice{$14,21$ m}
	\wrongchoice{$16,51$ m}
\end{choices}
\end{multicols}
\end{question}
}
%%%%%%%%%%%%%%%%%%%%%%%%%%%%%%%%%%%%%%%%%%%%%%%%%%%%%%%%%%%%%%%%%%%%%%%%%%%%%
\element{test1}{
\begin{question}{p6}
El valor de la presión intersticial en el punto $A$ vale:
\begin{multicols}{2}
\begin{choices}
	\correctchoice{$54.44$ kPa}
	\wrongchoice{$19.62$ kPa}
	\wrongchoice{$27,14$ kPa}
	\wrongchoice{$1,21$ MPa}
\end{choices}
\end{multicols}
\end{question}
}
%%%%%%%%%%%%%%%%%%%%%%%%%%%%%%%%%%%%%%%%%%%%%%%%%%%%%%%%%%%%%%%%%%%%%%%%%%%%%
\element{test1}{
\begin{question}{p7} 
En la Formulación Débil, la función de peso (o función de prueba) $\delta u$ debe
        cumplir obligatoriamente la siguiente restricción:
        \begin{multicols}{2}
            \begin{choices}
                \correctchoice{Debe anularse ($\delta u = 0$) en el contorno esencial (Dirichlet)}
                \wrongchoice{Debe anularse ($\delta u = 0$) en el contorno natural (Neumann)}
                \wrongchoice{Debe ser igual al flujo impuesto ($\delta u = \overline{q}$) en
                    el contorno de Neumann}
                \wrongchoice{Debe ser igual a la solución $u$ en todo el dominio}
\end{choices}
\end{multicols}
\end{question}
}
%%%%%%%%%%%%%%%%%%%%%%%%%%%%%%%%%%%%%%%%%%%%%%%%%%%%%%%%%%%%%%%%%%%%%%%%%%%%%
\element{test1}{
\begin{question}{p8}
El máximo valor de la velocidad en todo el dominio vale:
\begin{multicols}{2}
\begin{choices}
	\correctchoice{$0.27$ mm/s}
	\wrongchoice{$0.35$ mm/s}
	\wrongchoice{$0.42$ mm/s}
	\wrongchoice{$0.97$ mm/s}
\end{choices}
\end{multicols}
\end{question}
}
%%%%%%%%%%%%%%%%%%%%%%%%%%%%%%%%%%%%%%%%%%%%%%%%%%%%%%%%%%%%%%%%%%%%%%%%%%%%%
\element{test1}{
	\begin{question}{p9}
		En un problema de flujo en medios porosos (Ley de Darcy), la condición esencial de contorno:
        $$u = \overline{u}, \textrm{ en } \partial_u \Omega$$
        se interpreta como:
        \begin{multicols}{2}
            \begin{choices}
                \correctchoice{El valor impuesto, de tipo escalar, que corresponde al
                    potencial $u$ (altura piezométrica) en el contorno $\partial_u \Omega$}
                \wrongchoice{El valor impuesto, de tipo escalar, que corresponde al
                    flujo en dirección normal al contorno $\partial_u \Omega$}
                \wrongchoice{El valor impuesto, de tipo vectorial, que corresponde al
                    potencial $u$ (altura piezométrica) en el contorno $\partial_u \Omega$}
                \wrongchoice{El valor impuesto, de tipo vectorial, que corresponde al
                    gradiente del potencial en el contorno $\partial_u \Omega$}
			\end{choices}
		\end{multicols}
	\end{question}
}
%%%%%%%%%%%%%%%%%%%%%%%%%%%%%%%%%%%%%%%%%%%%%%%%%%%%%%%%%%%%%%%%%%%%%%%%%%%%%
\element{test1}{
	\begin{question}{p10}
		En un problema lineal de conducci\'on de calor (modelo de difusi\'on) discretizado con
una malla de $N$ nodos, con $n_u$ valores conocidos de la temperatura en los
nodos, y $n_t$ valores conocidos del flujo en direcci\'on normal, el sistema
lineal de ecuaciones resultante de la formulaci\'on de elementos finitos:
		\begin{multicols}{2}
			\begin{choices}
				\correctchoice{Tiene $N-n_u$ ecuaciones}
				\wrongchoice{Tiene $N-n_t$ ecuaciones}
				\wrongchoice{Tiene $N-n_u-n_t$ ecuaciones}
				\wrongchoice{Tiene $N+n_u-n_t$ ecuaciones}
			\end{choices}
		\end{multicols}
	\end{question}
}




%%%%%%%%%%%%%%%%%%%%%%%%%%%%%%%%%%%%%%%%%%%%%%%%%%%%%%%%%%%%%%%%%%%%%%%%%%%%%


%%%%%%%%%%%%%%%%%%%%%%%%%%%%%%%%%%%%%%%%%%%%%%%%%%%%%%%%%%%%%%%%%%%%%%%%%%%%%
% DEFINO LA ESTRATEGIA GLOBAL DE PUNTUACIÓN
%%%%%%%%%%%%%%%%%%%%%%%%%%%%%%%%%%%%%%%%%%%%%%%%%%%%%%%%%%%%%%%%%%%%%%%%%%%%%
% Estrategia para las preguntas con una respuesta correcta no
% respuesta 0, bien +1, mal -1/(N-1)
\scoringDefaultS{b=1,m=-1/(N-1)}
% bien +1, mal 0 \scoringDefaultS{b=1,m=0} Estrategia para las
% preguntas con multiples respuestas correctas bien +1, mal 0
\scoringDefaultM{b=1,m=0}
% bien +1, mal -1/(N-1) \scoringDefaultM{b=1,m=-1/(N-1)}


%%%%%%%%%%%%%%%%%%%%%%%%%%%%%%%%%%%%%%%%%%%%%%%%%%%%%%%%%%%%%%%%%%%%%%%%%%%%%
% INDICO EL NUMERO DE COPIAS A REALIZAR
%%%%%%%%%%%%%%%%%%%%%%%%%%%%%%%%%%%%%%%%%%%%%%%%%%%%%%%%%%%%%%%%%%%%%%%%%%%%%
\onecopy{1}{ \setcounter{figure}{0}

%%%%%%%%%%%%%%%%%%%%%%%%%%%%%%%%%%%%%%%%%%%%%%%%%%%%%%%%%%%%%%%%%%%%%%%%%%%%%
  % DEFINO EL ENCABEZADO DE LA HOJA DE PREGUNTAS
%%%%%%%%%%%%%%%%%%%%%%%%%%%%%%%%%%%%%%%%%%%%%%%%%%%%%%%%%%%%%%%%%%%%%%%%%%%%%

{\centering
{\small\sc Escuela Técnica Superior de Ingenieros de Caminos, Canales y
Puertos (Madrid)}\\*[4mm]
{\Large\bf Método de los Elementos Finitos (Curso 25-26)}\\*[4mm]
Ejercicio 2: Ecuación de difusión \\*[4mm]
}
\begin{center}\em
  \textbf{Tiempo: 60 minutos.}

  Está prohibido el uso de internet en el equipo y de teléfonos
  móviles.

  Todas las  preguntas tienen una única respuesta. Elija aquella que se
  aproxime más a su solución.

  Se atribuirá puntuación negativa a las respuestas incorrectas (-$1/3$ del valor de la pregunta).

\end{center}
\vspace{3mm}

%%%%%
\noindent

Se dispone una presa sobre un terreno tal que aguas arriba se almacena el agua a una cierta altura. Aguas abajo de la presa el nivel de agua coincide con el terreno natural. Se instaló una pantalla impermeable ($EAC$) de espesor igual a 0.05m en la zona de aguas abajo, como muestra la figura.   
El terreno está formado por tres materiales distintos, cuyas permeabilidades son respectivamente: $K_1$ = $1\cdot 10^{-4}$ m/s, $K_2$ = $1\cdot 10^{-3}$ m/s y $K_3$ = $5\cdot 10^{-3}$ m/s.

Se asumirá que los contornos laterales del terreno, así como la base del mismo, son impermeables. Las unidades de la figura son en metros. 


\vspace{0.2cm}

\begin{center}
\includegraphics[width=0.70\textwidth]{Ejer2._2025_cropped.pdf}
\end{center}
\vspace{0.2cm}

Téngase en cuenta las siguientes consideraciones:

\begin{itemize}
\item 
  Se usarán preferentemente elementos cuadriláteros (\emph{Quad dominated}) para la creación de la malla y la técnica de mallado \emph{Free}, con tamaño global de malla 1 m. e interpolación lineal (Elemento \emph{DC2D4}).
\item Se considerará que el fluido es agua dulce con densidad $\rho_w=1000$
  kg/m$^3$ y que el valor de la aceleración de la gravedad vale $g=9.81$ m/s$^2$.

\end{itemize}
\vspace{0.3cm}
% vacio el grupo miprueba

\cleargroup{miprueba}
% baraja el grupo historia y copia las cuatro primeras preguntas en
% miprueba
\copygroup{test1}{miprueba}
% baraja el grupo geografia y copia las tres primeras preguntas en
% miprueba baraja "miprueba"
\shufflegroup{miprueba}
% inserta las preguntas seleccionadas en "miprueba"
\insertgroup{miprueba}
%%%%%%%%%%%%%%%%%%%%%%%%%%%%%%%%%%%%%%%%%%%%%%%%%%%%%%%%%%%%%%%%%%%%%%%%%%%%%
% FINALIZO LA HOJA DE PREGUNTAS
%%%%%%%%%%%%%%%%%%%%%%%%%%%%%%%%%%%%%%%%%%%%%%%%%%%%%%%%%%%%%%%%%%%%%%%%%%%%%

\AMCcleardoublepage % Si quiero que la hoja de respuestas sea impar (nuevo folio)
% \clearpage % Si quiero que la hoja de respuestas este justo detras de
% la ultima hoja de preguntas

%%%%%%%%%%%%%%%%%%%%%%%%%%%%%%%%%%%%%%%%%%%%%%%%%%%%%%%%%%%%%%%%%%%%%%%%%%%%%
% INICIO LA HOJA DE RESPUESTAS
%%%%%%%%%%%%%%%%%%%%%%%%%%%%%%%%%%%%%%%%%%%%%%%%%%%%%%%%%%%%%%%%%%%%%%%%%%%%%

\AMCformBegin

%%%%%%%%%%%%%%%%%%%%%%%%%%%%%%%%%%%%%%%%%%%%%%%%%%%%%%%%%%%%%%%%%%%%%%%%%%%%%
% INCLUYO UN CODIGO PARA ASIGNAR ALUMNOS CON HOJA DE EXAMEN
%%%%%%%%%%%%%%%%%%%%%%%%%%%%%%%%%%%%%%%%%%%%%%%%%%%%%%%%%%%%%%%%%%%%%%%%%%%%%

\noindent\AMCcode{nummat}{4}\hspace*{\fill}
\begin{minipage}{.7\linewidth}
  $\longleftarrow{}$ Escriba su número de matrícula marcando los
  dígitos en los recuadros (con ceros a la izquierda si el número es
  de menos de cuatro dígitos) y el nombre y apellidos debajo.

  \vspace{3ex}


%%%%%%%%%%%%%%%%%%%%%%%%%%%%%%%%%%%%%%%%%%%%%%%%%%%%%%%%%%%%%%%%%%%%%%%%%%%%%
  % INCLUYO UN ESPACIO PARA QUE LOS ALUMNOS FIRMEN
%%%%%%%%%%%%%%%%%%%%%%%%%%%%%%%%%%%%%%%%%%%%%%%%%%%%%%%%%%%%%%%%%%%%%%%%%%%%%

  \namefield{\fbox{
      \begin{minipage}{.9\linewidth}
        Apellidos, Nombre:

        \vspace*{.5cm}\dotfill \vspace*{1mm}
      \end{minipage}
    }}
\end{minipage}

%%%%%%%%%%%%%%%%%%%%%%%%%%%%%%%%%%%%%%%%%%%%%%%%%%%%%%%%%%%%%%%%%%%%%%%%%%%%%
% INCLUYO UN COMENTARIO A LA HOJA DE RESPUESTAS
%%%%%%%%%%%%%%%%%%%%%%%%%%%%%%%%%%%%%%%%%%%%%%%%%%%%%%%%%%%%%%%%%%%%%%%%%%%%%

\begin{center}
  \bf\em Debe dar las respuestas exclusivamente en esta hoja (las
  respuestas en las demás hojas no serán tenidas en cuenta).
\end{center}


%%%%%%%%%%%%%%%%%%%%%%%%%%%%%%%%%%%%%%%%%%%%%%%%%%%%%%%%%%%%%%%%%%%%%%%%%%%%%
% FINALIZO LA HOJA DE RESPUESTAS
%%%%%%%%%%%%%%%%%%%%%%%%%%%%%%%%%%%%%%%%%%%%%%%%%%%%%%%%%%%%%%%%%%%%%%%%%%%%%

\AMCform
  
\AMCcleardoublepage % Si quiero que la siguiente hoja de preguntas sea
% impar (nuevo folio)
% \clearpage % Si quiero que la siguiente hoja de preguntas este
% justo detras de la ultima hoja de respuestas


}

\end{document}


