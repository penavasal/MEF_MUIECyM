\element{test1}{
\begin{question}{p1}
El caudal que se filtra, en régimen estacionario, a la derecha de la pantalla impermeable (contorno $CD$) vale:
\begin{multicols}{2}
\begin{choices}
	\correctchoice{$0.85$ l/s}
	\wrongchoice{$4.21$ l/s}
	\wrongchoice{$1.56$ l/s}
	\wrongchoice{$0.04$ l/s}
\end{choices}
\end{multicols}
\end{question}
}
%%%%%%%%%%%%%%%%%%%%%%%%%%%%%%%%%%%%%%%%%%%%%%%%%%%%%%%%%%%%%%%%%%%%%%%%%%%%%
\element{test1}{
\begin{question}{p2}
El valor máximo de la altura piezométrica en el contorno $FG$ vale:
\begin{multicols}{2}
\begin{choices}
	\correctchoice{$15,45$ m}
	\wrongchoice{$20,00$ m}
	\wrongchoice{$13,54$ m}
	\wrongchoice{$9,96$ m}
\end{choices}
\end{multicols}
\end{question}
}
%%%%%%%%%%%%%%%%%%%%%%%%%%%%%%%%%%%%%%%%%%%%%%%%%%%%%%%%%%%%%%%%%%%%%%%%%%%%%
\element{test1}{
\begin{question}{p3} 
El valor de la velocidad en el punto $C$ vale:
\begin{multicols}{2}
\begin{choices}
	\correctchoice{$0.09$ mm/s}
	\wrongchoice{$0.22$ mm/s}
	\wrongchoice{$0.03$ mm/s}
	\wrongchoice{$0.41$ mm/s}
\end{choices}
\end{multicols}
\end{question}
}
%%%%%%%%%%%%%%%%%%%%%%%%%%%%%%%%%%%%%%%%%%%%%%%%%%%%%%%%%%%%%%%%%%%%%%%%%%%%%
\element{test1}{
\begin{question}{p4}
El máximo valor de la velocidad horizontal en el contorno $BE$ es:
\begin{multicols}{2}
\begin{choices}
	\correctchoice{$0.15$ mm/s}
	\wrongchoice{$0.02$ mm/s}
	\wrongchoice{$3.33$ mm/s}
	\wrongchoice{$5.21$ mm/s}
\end{choices}
\end{multicols}
\end{question}
}
%%%%%%%%%%%%%%%%%%%%%%%%%%%%%%%%%%%%%%%%%%%%%%%%%%%%%%%%%%%%%%%%%%%%%%%%%%%%%
\element{test1}{
\begin{question}{p5}
La altura piezométrica en el punto $B$ vale:
\begin{multicols}{2}
\begin{choices}
	\correctchoice{$17,03$ m}
	\wrongchoice{$19,99$ m}
	\wrongchoice{$14,21$ m}
	\wrongchoice{$16,51$ m}
\end{choices}
\end{multicols}
\end{question}
}
%%%%%%%%%%%%%%%%%%%%%%%%%%%%%%%%%%%%%%%%%%%%%%%%%%%%%%%%%%%%%%%%%%%%%%%%%%%%%
\element{test1}{
\begin{question}{p6}
El valor de la presión intersticial en el punto $A$ vale:
\begin{multicols}{2}
\begin{choices}
	\correctchoice{$54.44$ kPa}
	\wrongchoice{$19.62$ kPa}
	\wrongchoice{$27,14$ kPa}
	\wrongchoice{$1,21$ MPa}
\end{choices}
\end{multicols}
\end{question}
}
%%%%%%%%%%%%%%%%%%%%%%%%%%%%%%%%%%%%%%%%%%%%%%%%%%%%%%%%%%%%%%%%%%%%%%%%%%%%%
\element{test1}{
\begin{question}{p7} 
En la Formulación Débil, la función de peso (o función de prueba) $\delta u$ debe
        cumplir obligatoriamente la siguiente restricción:
        \begin{multicols}{2}
            \begin{choices}
                \correctchoice{Debe anularse ($\delta u = 0$) en el contorno esencial (Dirichlet)}
                \wrongchoice{Debe anularse ($\delta u = 0$) en el contorno natural (Neumann)}
                \wrongchoice{Debe ser igual al flujo impuesto ($\delta u = \overline{q}$) en
                    el contorno de Neumann}
                \wrongchoice{Debe ser igual a la solución $u$ en todo el dominio}
\end{choices}
\end{multicols}
\end{question}
}
%%%%%%%%%%%%%%%%%%%%%%%%%%%%%%%%%%%%%%%%%%%%%%%%%%%%%%%%%%%%%%%%%%%%%%%%%%%%%
\element{test1}{
\begin{question}{p8}
El máximo valor de la velocidad en todo el dominio vale:
\begin{multicols}{2}
\begin{choices}
	\correctchoice{$0.27$ mm/s}
	\wrongchoice{$0.35$ mm/s}
	\wrongchoice{$0.42$ mm/s}
	\wrongchoice{$0.97$ mm/s}
\end{choices}
\end{multicols}
\end{question}
}
%%%%%%%%%%%%%%%%%%%%%%%%%%%%%%%%%%%%%%%%%%%%%%%%%%%%%%%%%%%%%%%%%%%%%%%%%%%%%
\element{test1}{
	\begin{question}{p9}
		En un problema de flujo en medios porosos (Ley de Darcy), la condición esencial de contorno:
        $$u = \overline{u}, \textrm{ en } \partial_u \Omega$$
        se interpreta como:
        \begin{multicols}{2}
            \begin{choices}
                \correctchoice{El valor impuesto, de tipo escalar, que corresponde al
                    potencial $u$ (altura piezométrica) en el contorno $\partial_u \Omega$}
                \wrongchoice{El valor impuesto, de tipo escalar, que corresponde al
                    flujo en dirección normal al contorno $\partial_u \Omega$}
                \wrongchoice{El valor impuesto, de tipo vectorial, que corresponde al
                    potencial $u$ (altura piezométrica) en el contorno $\partial_u \Omega$}
                \wrongchoice{El valor impuesto, de tipo vectorial, que corresponde al
                    gradiente del potencial en el contorno $\partial_u \Omega$}
			\end{choices}
		\end{multicols}
	\end{question}
}
%%%%%%%%%%%%%%%%%%%%%%%%%%%%%%%%%%%%%%%%%%%%%%%%%%%%%%%%%%%%%%%%%%%%%%%%%%%%%
\element{test1}{
	\begin{question}{p10}
		En un problema lineal de conducci\'on de calor (modelo de difusi\'on) discretizado con
una malla de $N$ nodos, con $n_u$ valores conocidos de la temperatura en los
nodos, y $n_t$ valores conocidos del flujo en direcci\'on normal, el sistema
lineal de ecuaciones resultante de la formulaci\'on de elementos finitos:
		\begin{multicols}{2}
			\begin{choices}
				\correctchoice{Tiene $N-n_u$ ecuaciones}
				\wrongchoice{Tiene $N-n_t$ ecuaciones}
				\wrongchoice{Tiene $N-n_u-n_t$ ecuaciones}
				\wrongchoice{Tiene $N+n_u-n_t$ ecuaciones}
			\end{choices}
		\end{multicols}
	\end{question}
}



