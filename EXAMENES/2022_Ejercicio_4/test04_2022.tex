\documentclass[a4paper]{article}

\usepackage[utf8x]{inputenc}    
\usepackage[T1]{fontenc}
\usepackage[spanish]{babel}
\usepackage{multicol}

\usepackage{wrapfig}
\usepackage{graphicx}

\usepackage{bm}
\usepackage{amsxtra} 
\usepackage{amssymb}% to get the \mathbb alphabet
\usepackage{amsmath}

\usepackage[box,completemulti,separateanswersheet]{automultiplechoice}    
\def\AMCformQuestion#1{\vspace{\AMCformVSpace}\par {\sc Pregunta #1:} }    
\def\AMCbeginQuestion#1#2{\par\noindent{\bf Pregunta #1}#2\hspace*{1em}}
\def\AMCcleardoublepage{\ifodd\thepage\clearpage\mbox{}\fi\clearpage}

\begin{document}

\AMCrandomseed{1237893}

%%%%%%%%%%%%%%%%%%%%%%%%%%%%%%%%%%%%%%%%%%%%%%%%%%%%%%%%%%%%%%%%%%%%%%%%%%%%%
\element{test1}{
\begin{question}{p1}
El desplazamiento absoluto vertical en el punto D es:
\begin{multicols}{2}
\begin{choices}
	\correctchoice{$8.454$ mm}
	\wrongchoice{$8.462$ mm}
	\wrongchoice{$7.354$ mm}
	\wrongchoice{$7.652$ mm}
\end{choices}
\end{multicols}
\end{question}
}
%%%%%%%%%%%%%%%%%%%%%%%%%%%%%%%%%%%%%%%%%%%%%%%%%%%%%%%%%%%%%%%%%%%%%%%%%%%%%
\element{test1}{
\begin{question}{p2}
La reacción vertical total es:
\begin{multicols}{2}
\begin{choices}
	\correctchoice{$2914.04$ kN}
	\wrongchoice{$2750.00$ kN}
	\wrongchoice{$2875.02$ kN}
	\wrongchoice{$3005.03$ kN}
\end{choices}
\end{multicols}
\end{question}
}
%%%%%%%%%%%%%%%%%%%%%%%%%%%%%%%%%%%%%%%%%%%%%%%%%%%%%%%%%%%%%%%%%%%%%%%%%%%%%
\element{test1}{
\begin{question}{p3}
La reacción vertical por peso propio es:
\begin{multicols}{2}
\begin{choices}
	\correctchoice{$164.04$ kN}
	\wrongchoice{$2750.04$ kN}
	\wrongchoice{$2914.04$ kN}
	\wrongchoice{$184.04$ kN}
\end{choices}
\end{multicols}
\end{question}
}
%%%%%%%%%%%%%%%%%%%%%%%%%%%%%%%%%%%%%%%%%%%%%%%%%%%%%%%%%%%%%%%%%%%%%%%%%%%%%
\element{test1}{
\begin{question}{p4}
La deformación principal máxima en valor absoluto es:
\begin{multicols}{2}
\begin{choices}
	\correctchoice{$2.801 \cdot 10^{-4}$ mm/mm}
	\wrongchoice{$2.696 \cdot 10^{-4}$ mm/mm}
	\wrongchoice{$7.907 \cdot 10^{-4}$ mm/mm}
	\wrongchoice{$7.698 \cdot 10^{-4}$ mm/mm}
\end{choices}
\end{multicols}
\end{question}
}
%%%%%%%%%%%%%%%%%%%%%%%%%%%%%%%%%%%%%%%%%%%%%%%%%%%%%%%%%%%%%%%%%%%%%%%%%%%%%
\element{test1}{
\begin{question}{p5}
La tensión principal máxima en el punto F es:
\begin{multicols}{2}
\begin{choices}
	\correctchoice{$3.685$ MPa}
	\wrongchoice{$1.5725$ MPa}
	\wrongchoice{$4.293$ MPa}
	\wrongchoice{$1.684$ MPa}
\end{choices}
\end{multicols}
\end{question}
}
%%%%%%%%%%%%%%%%%%%%%%%%%%%%%%%%%%%%%%%%%%%%%%%%%%%%%%%%%%%%%%%%%%%%%%%%%%%%%
\element{test1}{
\begin{question}{p6}
Considerando el tipo de elemento C3D8R (sub-integración), el desplazamiento absoluto vertical en el punto C es:
\begin{multicols}{2}
	\begin{choices}
		\correctchoice{$2.626$ mm}
		\wrongchoice{$2.9002$ mm}
		\wrongchoice{$2.494$ mm}
		\wrongchoice{$2.855$ mm}
\end{choices}
\end{multicols}
\end{question}
}
%%%%%%%%%%%%%%%%%%%%%%%%%%%%%%%%%%%%%%%%%%%%%%%%%%%%%%%%%%%%%%%%%%%%%%%%%%%%%
\element{test1}{
\begin{question}{p7}
Considerando el tipo de elemento C3D8R (sub-integración), el desplazamiento máximo absoluto horizontal es:
\begin{multicols}{2}
\begin{choices}
		\correctchoice{$0.819$ mm}
		\wrongchoice{$0.868$ mm}
		\wrongchoice{$0.791$ mm}
		\wrongchoice{$0.789$ mm}
\end{choices}
\end{multicols}
\end{question}
}
%%%%%%%%%%%%%%%%%%%%%%%%%%%%%%%%%%%%%%%%%%%%%%%%%%%%%%%%%%%%%%%%%%%%%%%%%%%%%
\element{test1}{
\begin{question}{p8}
Considerando el tipo de elemento C3D8I (modos incompatibles), el desplazamiento máximo absoluto horizontal en el nudo F es:
\begin{multicols}{2}
\begin{choices}
		\correctchoice{$0.696$ mm}
		\wrongchoice{$0.819$ mm}
		\wrongchoice{$0.797$ mm}
		\wrongchoice{$0.704$ mm}
\end{choices}
\end{multicols}
\end{question}
}
%%%%%%%%%%%%%%%%%%%%%%%%%%%%%%%%%%%%%%%%%%%%%%%%%%%%%%%%%%%%%%%%%%%%%%%%%%%%%
\element{test1}{
\begin{question}{p9}
Considerando el tipo de elemento C3D8RI (modos incompatibles), la deformación principal mínima en valor absoluto es:
\begin{multicols}{2}
\begin{choices}
		\correctchoice{$8.429 \cdot 10^{-4}$ mm/mm}
		\wrongchoice{$6.808 \cdot 10^{-4}$ mm/mm}
		\wrongchoice{$2.696 \cdot 10^{-4}$ mm/mm}
		\wrongchoice{$2.190 \cdot 10^{-4}$ mm/mm}
\end{choices}
\end{multicols}
\end{question}
}
%%%%%%%%%%%%%%%%%%%%%%%%%%%%%%%%%%%%%%%%%%%%%%%%%%%%%%%%%%%%%%%%%%%%%%%%%%%%%
\element{test1}{
\begin{question}{p10}
La formulación dxxxxxxxxx:
\begin{multicols}{2}
\begin{choices}
        \correctchoice{El principio de los trabajos virtuales}
        \wrongchoice{El equilibrio de fuerzas en cada punto del sólido}
        \wrongchoice{No tiene interpretación física}
        \wrongchoice{Un requisito de convergencia del método de elementos
                     finitos}
\end{choices}
\end{multicols}
\end{question}
}
%%%%%%%%%%%%%%%%%%%%%%%%%%%%%%%%%%%%%%%%%%%%%

\scoringDefaultS{b=1,m=-1/(N-1)}

\onecopy{1}{    

%%% beginning of the test sheet header:    

\noindent{\large\bf Método de los Elementos Finitos  \hfill MUECYM \hfill TEST \# 3}

\vspace*{.5cm}
\begin{minipage}{.4\linewidth}
  \centering 21 oct 2022
\end{minipage}

\begin{center}\em
Tiempo: 60 minutos.

  %Está prohibido el uso de teléfonos móviles.

  %Se atribuirá puntuación negativa a las respuestas incorrectas.

\end{center}
\vspace{1ex}

%%% end of the header

\shufflegroup{test1}
\insertgroup{test1}

\AMCcleardoublepage    
%\clearpage

\AMCformBegin    

%%% beginning of the answer sheet header

\noindent\AMCcode{nummat}{2}\hspace*{\fill}
\begin{minipage}{.7\linewidth}
$\longleftarrow{}$ Escriba su número de matrícula marcando los dígitos
en los recuadros (con ceros a la izquierda si el número es de menos de dos dígitos) y el nombre y apellidos debajo.

\vspace{3ex}

\namefield{\fbox{
   \begin{minipage}{.9\linewidth}
     Apellidos, Nombre:

     \vspace*{.5cm}\dotfill
     \vspace*{1mm}
   \end{minipage}
 }}
\end{minipage}

\begin{center}
 \bf\em Debe dar las respuestas exclusivamente en esta hoja (las respuestas en las demás hojas no serán tenidas en cuenta).
\end{center}

%%% end of the answer sheet header


\AMCform    

\AMCcleardoublepage    

}  

\end{document}
